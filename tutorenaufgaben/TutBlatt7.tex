\documentclass[10pt,oneside,onecolumn,a4paper,german,titlepage]{article}
\usepackage[utf8]{inputenc}
\usepackage[german]{babel}
\usepackage{german}
\usepackage{amsmath,amsfonts,amssymb,latexsym,textcomp,stmaryrd}
\usepackage{tikz}
\usetikzlibrary{automata,positioning}
\pagestyle{plain}
\pagenumbering{arabic}
\parskip0.5ex plus0.1ex minus0.1ex
\parindent0pt
\voffset-1.04cm
\topmargin0pt
\headheight0pt
\headsep0pt
\topskip0pt
\textheight26.7cm
\footskip20pt
\hoffset-1.04cm
\oddsidemargin0pt
\evensidemargin0pt
\textwidth18cm
\marginparsep0pt
\marginparwidth0pt
\begin{document}

\section*{Tutorien-"Ubungsblatt 7}

\subsection*{Aufgabe 1}
\begin{enumerate}
\item Beweisen Sie, dass $K(x)$ nicht berechenbar ist!
\item Beweisen Sie, dass die Menge der nichtkomprimierbaren Strings $\mathcal{L}$
nicht rekursiv aufz"ahlbar ist!
\item Geben Sie eine m"oglichst gute obere Schranke f"ur die Kolmogorow-Komplexit"at von $0^n$ an!
\item Geben Sie eine m"oglichst gute obere Schranke f"ur die Kolmogorow-Komplexit"at der
Bin"ardarstellung der $n$-ten\\ Primzahl $p$ an!
\item Sei $x$ ein Palindrom. Geben sie eine m"oglichst gute obere Schranke f"ur $K(x)$ an!
\item Sei $\pi_n$ die Kreiszahl $\pi$ bis zur $n$-ten Nachkommastelle entwickelt. Geben Sie eine m"oglichst gute obere Schranke f"ur $\pi_n$ an.
\end{enumerate}

\subsection*{Aufgabe 2}
Geben Sue f"ur folgendende Formeln an ob diese in den besagten Theorien liegen
\begin{enumerate}
\item Ist $\phi_1 = \forall x \exists y \forall z: x + y = z$ in $\text{Th}(\mathbb{N,+})$?
\item Ist $\phi_2 = \forall x \exists y \forall z \exists w: (x + z = w ) \wedge (x + y = w)$  in $\text{Th}(\mathbb{N},+)$?
\item Ist $\phi_3 = \forall x \forall y \forall z \forall w \forall v \exists s: \neg(x + w = y) \vee \neg(y + v = z) \vee (x + s = z)$ in $\text{Th}(\mathbb{N},+)$?
\item Sei $\text{Th}(\mathbb{N},<)$ die Theorie der nat"urlichen Zahlen mit der Relation "`echt kleiner"'. Zeigen Sie: $\text{Th}(\mathbb{N},<)$ ist entscheidbar.
\end{enumerate}

\subsection*{Aufgabe 3}
Geben Sie Modelle f"ur die folgenden pr"adikatenlogischen Formeln an! Geben Sie dazu
jeweils ein Universum $\mathcal{U}$\\
und eine Interpretation der Relationszeichen $R_i$ an!
\begin{enumerate}
\item $\phi_1 =$ \hspace*{0.2cm} $\forall \; x \; (R_1(x,x))$ \hspace*{5.3cm}[K1.1]\\
\hspace*{0.85cm} $\wedge \forall \; x,y \; (R_1(x,y) \leftrightarrow R_1(y,x))$
\hspace*{3.2cm} [K1.2]\\
\hspace*{0.85cm} $\wedge \forall \; x,y,z \; ((R_1(x,y) \wedge R_1(y,z)) \rightarrow
R_1(x,z))$ \hspace*{1cm} [K1.3]
\item $\phi_2 =$ \hspace*{0.2cm} $\phi_1$\\
\hspace*{0.85cm} $\wedge \forall \; x \; (R_1(x,x) \rightarrow \neg R_2(x,x))$
\hspace*{3.25cm} [K2.1]\\
\hspace*{0.85cm} $\wedge \forall \; x,y \; (\neg R_1(x,y) \rightarrow (R_2(x,y)
\oplus R_2(y,x)))$ \hspace*{1cm} [K2.2]\\
\hspace*{0.85cm} $\wedge \forall \; x,y,z \; ((R_2(x,y) \wedge R_2(y,z)) \rightarrow
R_2(x,z))$ \hspace*{1cm}  [K2.3]\\
\hspace*{0.85cm} $\wedge \forall \; x \; \exists \; y \, (R_2(x,y))$ \hspace*{4.8cm}
[K2.4]
\end{enumerate}

\newpage

\subsection*{L"osung zu Aufgabe 1}
\begin{enumerate}
\item \underline{Annahme:} $K(x)$ ist berechenbar\\
Unter Verwendung des Rekursionstheorems l"a"st sich folgende Turingmaschine
$\mathcal{M}$ konstruieren:\\[4pt]
1. Generiere eigene Darstellung $\langle\mathcal{M}\rangle$!\\
2. Z"ahle alle Strings $x \in \{0,1\}^*$ auf und berechne $K(x)$!\\
\hspace*{0.4cm} Falls $K(x) > |\langle\mathcal{M}\rangle|$, breche die Z"ahlschleife
ab!\\
3. Schreibe $x$ auf das Band!\\[4pt]
Damit ist $|\langle\mathcal{M}\rangle|$ kleiner als $K(x)$ und $\langle\mathcal{M}
\rangle$ ist eine Beschreibung von $x$. Damit ist also $K(x)$ nicht die L"ange\\
der kleinsten Beschreibung. Widerspruch!
\item \underline{Annahme:} $\mathcal{L}$ ist rekursiv aufz"ahlbar\\
Dann gibt es einen Aufz"ahler $T$ f"ur $\mathcal{L}$ und es l"a"st sich "uber das
Rekursionstheorem folgende Turingmaschine\\ $\mathcal{M}$ konstruieren:\\[4pt]
1. Generiere eigene Darstellung $\langle\mathcal{M}\rangle$!\\
2. Verwende $T$, um die Strings in $\mathcal{L}$ aufzuz"ahlen!\\
\hspace*{0.4cm} Sobald ein aufgez"ahlter String $x$ l"anger ist als $|\langle
\mathcal{M}\rangle|$, breche ab!\\
3. Schreibe $x$ auf das Band!\\[4pt]
Nun ist $\langle\mathcal{M}\rangle$ eine k"urzere Beschreibung von $x$, also ist $x$
damit komprimierbar. Widerspruch!
\item Eine obere Schranke ist $\log n + c$.\\
Verwende Turingmaschine $\mathcal{M}$ mit Eingabe $n$ in Bin"ardarstellung:\\
1. Ersetze Bin"ardarstellung von $n$ durch Un"ardarstellung $0^n$!\\
Die Gr"osse $c$ dieser Turingmaschine ist eine Konstante $c$ die unabh"angig von $n$ ist.

\item Eine obere Schranke ist $\log n + c$.\\
Verwende folgende Turingmaschine $\mathcal{M}$ mit Eingabe $n$ in Bin"ardarstellung:\\
1. Z"ahle alle Primzahlen auf (z.B. Sieb des Eratosthenes) und lasse einen Z"ahler
mitlaufen!\\
2. Sobald der Z"ahler den Wert $n$ erreicht, breche ab!\\
3. Gib die aktuelle Primzahl $p$ aus!
Die Gr"osse $c$ dieser Turingmaschine ist eine Konstante $c$ die unabh"angig von $n$ ist.

\item Sei $|x| = n$. Eine obere Schranke f"ur $K(x)$ ist $\frac{|x|}{2} + c$. Wir geben dazu eine Maschine $\mathcal{M}$ an die $x$ bei Eingabe der ersten Worth"alfte $x_1$ erzeugt.\\
1. Falls $x$ ein Palindrom ungerader L"ange ist, dann schreibe $x_{\lfloor \frac{n}{2} \rfloor + 1}$ auf das Band hinter $x_1$\\
2. Spiegle $x_1^R$ auf das Band hinter den Bandinhalt\\
Die Gr"osse $c$ dieser Turingmaschine ist eine Konstante $c$ die unabh"angig von $n$ ist.
\item Eine obere Schranke f"ur $K(\pi_n)$ ist auch hier $\log(n) + c$. Man kann die Bin"arentwicklung von $\pi$ zum Beispiel anhand der BPP-Folge berechnen.
\begin{align*}
\pi = \sum_{k = 0}^\infty {\frac{1}{16^k}\left( \frac{4}{8k + 1} - \frac{2}{8k + 4} - \frac{1}{8k + 5} - \frac{1}{8k + 6}\right)}
\end{align*}
\end{enumerate}
Man summiert einfach solange auf bis sich die $n$-te Stelle nicht mehr "andert und bricht dann ab. Die Gr"osse einer Turingmaschine die dies berechnet ist eine Konstante $c$ und unabh"angig von $n$.

\subsection*{L"osung zu Aufgabe 2}
\begin{enumerate}
\item Nein. W"ahle irgend ein $x \in \mathbb{N}$. Dann setze $z = x + y + 1$. Somit gilt f"ur jede m"ogliche Wahl von $z$: $x + y \neq x + y + 1 = z$ womit $\phi_1$ keine "`Wahrheit"' in $\text{Th}(\mathbb{N},+)$ ist.
\item Nein. W"ahle dazu $x \in \mathbb{N}$ beliebig und $z = y + 1$. W"are $\phi_2$ in $\text{Th}(\mathbb{N},+)$, so w"urde gelten $w = x + z = x + y + 1 = w + 1$ was ein Widerspruch ist.
\item Ja. W"ahle dazu $s = w + v$. Gilt $\neg(x + w = y)$ oder $\neg(y + v = z)$ dann ist $\phi_3$ trivialerweise wahr. Nehmen wir also an $(x + w = y) \wedge (y + v = z)$. Dann gilt $x + s = x + w + v = y + v = z$ womit $\phi_3$ auch in diesem Falle wahr ist.
\item Wir reduzieren dazu das "`Wortproblem"' von $\text{Th}(\mathbb{N},<)$ auf das Wortproblem von $\text{Th}(\mathbb{N},+)$. Sei $\phi$ eine Formel "uber $(\mathbb{N},<)$. F"ur alle Auftreten von Teilformeln $x < y$ f"ur beliebige $x,y$ in $\phi$ ersetzen wir $x < y$ durch $\exists z \exists w: (x + z = y) \wedge (1 + w = z)$ mit ungebundenen Variablen $z$ und $w$. Es ist offensichtlich dass nun $x < y$ falls $\exists z \exists w: (x + z = y) \wedge (1 + w = z)$ gilt. Wir erhalten also eine neue Formel $\phi^\prime$ die genau dann in $\text{Th}(\mathbb{N},+)$ liegt falls $\phi$ in $\text{Th}(\mathbb{N},<)$ liegt.
\end{enumerate}


\subsection*{L"osung zu Aufgabe 3}
F"ur das Universum $\mathcal{U}$ ziehen wir die klassischen Zahlenmengen $\mathbb{N},
\mathbb{N}_0, \mathbb{Z}, \mathbb{Q}, \mathbb{R}, \mathbb{C}$ in Betracht und suchen
unter\\ den Standardrelationen $=, \leq, \geq, <, >, \neq$ nach geeigneten Kandidaten
zur Interpretation der Relationszeichen $R_i$.
\begin{enumerate}
\item Die einzelnen Klauseln erlauben jeweils folgende Relationen für $R_1$:\\
Klausel [K1.1]: $=, \leq, \geq$\\
Klausel [K1.2]: $=, \neq$\\
Klausel [K1.3]: $=, \leq, \geq, <, >$\\
Da nur $=$ als einzige "Aquivalenzrelation alle drei Klauseln erf"ullt, sind die
folgenden Tupel $(\mathcal{U},R_1)$\\ m"ogliche Modelle:\\
$(\mathbb{N},=), (\mathbb{N}_0,=), (\mathbb{Z},=), (\mathbb{Q},=), (\mathbb{R},=),
(\mathbb{C},=)$
\item Da hier $R_1$ durch $\phi_1$ auf die Relation $=$ festgelegt ist, erlauben die
zus"atzlichen Klauseln jeweils folgende\\ Relationen für $R_2$:\\
Klausel [K2.1]: $<, >, \neq$\\
Klausel [K2.2]: $\leq, \geq, <, >$\\
Klausel [K2.3]: $=, \leq, \geq, <, >$\\
Klausel [K2.4]: $=, \leq, \geq, <, > (\mbox{au"ser f"ur} \; \mathbb{N}, \mathbb{N}_0),
\neq$\\
Da nur $<$ und mit Einschr"ankungen auch $>$ die vier neuen Klauseln erf"ullen, sind
die folgenden Tupel\\ $(\mathcal{U},R_1,R_2)$ m"ogliche Modelle:\\
$(\mathbb{N},=,<), (\mathbb{N}_0,=,<), (\mathbb{Z},=,<), (\mathbb{Q},=,<),
(\mathbb{R},=,<), (\mathbb{C},=,<), (\mathbb{Z},=,>), (\mathbb{Q},=,>),
(\mathbb{R},=,>), (\mathbb{C},=,>)$
\end{enumerate}

\end{document}