\documentclass[10pt,oneside,onecolumn,a4paper,german,titlepage]{article}
\usepackage[utf8]{inputenc}
\usepackage[german]{babel}
\usepackage{german}
\usepackage{amsmath,amsfonts,amssymb,latexsym,textcomp,stmaryrd}
\pagestyle{plain}
\pagenumbering{arabic}
\parskip0.5ex plus0.1ex minus0.1ex
\parindent0pt
\voffset-1.04cm
\topmargin0pt
\headheight0pt
\headsep0pt
\topskip0pt
\textheight26.7cm
\footskip20pt
\hoffset-1.04cm
\oddsidemargin0pt
\evensidemargin0pt
\textwidth18cm
\marginparsep0pt
\marginparwidth0pt
\begin{document}

\section*{Tutorien-"Ubungsblatt 9}

\subsection*{Aufgabe 1}
Gegeben sind die folgenden Probleme:
\begin{tabbing}
PARTITION:\\
\textit{Gegeben:} \= Nat"urliche Zahlen $a_1,...,a_n \in \mathbb{N} \; (n \in
\mathbb{N})$\\
\textit{Gesucht:} \> Gibt es eine Teilmenge $J \subseteq \{1,...,n\}$ mit
$\sum\limits_{1 \leq i \leq n, i \in J}a_i \; = \;
\sum\limits_{1 \leq i \leq n, i \notin J}a_i$?\\[8pt]
BIN PACKING:\\
\textit{Gegeben:} \> Eine Beh"altergr"o"se $b \in \mathbb{N}$, die Anzahl der
Beh"alter $k \in \mathbb{N}$ und Objekte $a_1,...,a_n \; (n \in \mathbb{N})$ mit\\
\> $a_i \in \mathbb{N}, a_i \leq b$ f"ur alle $i \in \{1,...,n\}$\\
\textit{Gesucht:} \> K"onnen die $n$ Objekte so auf die $k$ Beh"alter verteilt
werden, dass kein Beh"alter "uberbeladen ist?\\
\> (Das hei"st: Existiert eine Abbildung $f: \{1,...,n\} \to \{1,...,k\}$,
sodass f"ur alle $j \in \{1,...,k\}$ gilt:\\
\> $\sum\limits_{1 \leq i \leq k, f(i) = j}a_i \; \leq \; b$?)
\end{tabbing}
\begin{enumerate}
\item Zeigen Sie, dass BIN PACKING $\mathcal{NP}$-hart ist, wobei PARTITION als
$\mathcal{NP}$-vollst"andig vorausgesetzt werden\\darf!
\item Gegeben seien die Objekte der PARTITION-Probleminstanz
$(a_1,a_2,a_3,a_4,a_5,a_6) = (1,1,2,3,4,5)$. Zeigen\\
oder widerlegen Sie, ob das transformierte und das urspr"ungliche Problem eine
L"osung besitzen!
\end{enumerate}

\subsection*{Aufgabe 2}
Gehen Sie bei dieser Aufgabe durchweg von der Annahme $\mathcal{P} \neq \mathcal{NP}$
aus. Sie k"onnen Aussagen verwenden, die in\\
Vorlesung oder "Ubung gezeigt/behandelt wurden.
\begin{enumerate}
\item
\begin{tabbing}
\textbf{Problem:} 4-COLOR\\
\textit{Gegeben:} \= Ein ungerichteter Graph $G = (V,E)$\\
\textit{Gesucht:} \> Gibt es eine F"arbung der Knoten $V$, sodass je zwei durch eine
Kante aus $E$ miteinander\\
\> verbundene Knoten unterschiedlich gef"arbt sind, wenn nur vier unterschiedliche
Farben zur\\
\> Verf"ugung stehen?
\end{tabbing}
Zeigen Sie, dass 4-COLOR $\mathcal{NP}$-vollst"andig ist!\\
\underline{Hinweis:} Es kann hilfreich sein, wenn Sie die $\mathcal{NP}$-
Vollst"andigkeit des Dreif"arbbarkeitsproblems\\
3-COLOR verwenden.
\item
Es ist bekannt, dass sowohl das Erf"ullbarkeitsproblem der Aussagenlogik SAT als
auch 3-SAT, das\\
Erf"ullbarkeitsproblem mit Beschr"ankung auf Klauseln mit nur 3
Literalen, $\mathcal{NP}$-vollst"andig sind.\\
Das Problem 3-CLIQUE ist wie folgt definiert:
\begin{tabbing}
\textbf{Problem:} 3-CLIQUE\\
\textit{Gegeben:} \= Ein ungerichteter Graph $G = (V,E)$\\
\textit{Gesucht:} \> Gibt es eine Clique (vollst"andig verbundener Teilgraph) der
Gr"o"se $3$ in $G$?
\end{tabbing}
Zeigen Sie, dass 3-CLIQUE \textbf{nicht} $\mathcal{NP}$-vollst"andig ist!
\end{enumerate}

\subsection*{Aufgabe 3}
Beweisen Sie folgende Aussagen:
\begin{enumerate}
\item Die Klasse $\mathcal{NP}$ ist unter Schnittbildung abgeschlossen.
\item Die Klasse $\mathcal{NP}$ ist unter Vereinigung abgeschlossen.
\end{enumerate}
Dabei nehmen wir an, dass alle Sprachen "uber dem bin"aren Alphabet $\Sigma =
\{0,1\}$ definiert sind.

\newpage

\subsection*{L"osung zu Aufgabe 1}
\begin{enumerate}
\item Hierzu wird die Reduktion PARTITION$\leq_p$BIN PACKING benutzt:\\[4pt]
$(a_1,...,a_n) \mapsto
\left\{\begin{array}{l@{\hspace{0.2cm}}l}
\mbox{Beh"altergr"o"se:} & b := \sum\limits_{i=1}^n a_i/2\\
\mbox{Beh"alteranzahl:} & k := 2\\
\mbox{Objekte:} & (a_1,...,a_n)\\
\end{array}\right.$\\[4pt]
Falls eine L"osung des PARTITION-Problems existiert, dann gibt es nach Definition
eine Teilmenge $J \subseteq \{1,...,n\}$ mit $\sum\limits_{1 \leq i \leq n, i \in J}
a_i \; = \; \sum\limits_{1 \leq i \leq n, i \notin J} a_i$. Dann folgt:\\[4pt]
$\sum\limits_{i=1}^n a_i = \sum\limits_{1 \leq i \leq n, i \in J} a_i +
\sum\limits_{1 \leq i \leq n, i \notin J} a_i = 2 * \sum\limits_{1 \leq i \leq n,
i \in J} a_i  \Rightarrow \sum\limits_{1 \leq i \leq n, i \in J} a_i =
\sum\limits_{i=1}^n a_i/2 = b = \sum\limits_{1 \leq i \leq n, i \notin J} a_i$\\[4pt]
Man kann also die Objekte zur Indexmenge $J$ in den ersten der beiden Beh"alter tun,
der dann vollst"andig gef"ullt ist, und die restlichen Objekte in den anderen
Beh"alter. Diese Reduktion ist polynomial, ihr Aufwand ist $O(n)$.
\item F"ur die Objekte des BIN PACKING-Problems kann man eine L"osung finden, wenn
man wie oben die Abbildung anwendet: $k = 2, b = \sum\limits_{i=1}^6 a_i/2 =
(1+1+2+3+4+5)/2 = 8$. Somit ist das BIN PACKING-Problem mit der Indexmenge $J =
\{1,4,5\}$ gel"ost, denn $\sum\limits_{1 \leq i \leq n, i \in J} a_i = 1+3+4 = 8 =
b$ und $\sum\limits_{1 \leq i \leq n, i \notin J} a_i = 1+2+5 = 8 = b$. Die
urspr"ungliche Probleminstanz f"ur PARTITION besitzt damit auch eine L"osung,
n"amlich die Indexmenge $J$.
\end{enumerate}

\subsection*{L"osung zu Aufgabe 2}
\begin{enumerate}
\item
Zun"achst einmal macht man sich klar, dass 4-COLOR in $\mathcal{NP}$ liegt. Das ist
der Fall, denn r"at man eine F"arbung der Knoten, dann ist die Korrektheit der
F"arbung in $O(|E|)$ Schritten verifizierbar, indem man schlicht "uber alle Kanten
iteriert und "uberpr"uft, ob die Endknoten unterschiedlich gef"arbt sind.\\
4-COLOR ist $\mathcal{NP}$-hart, denn wir k"onnen 3-COLOR auf 4-COLOR reduzieren.
Sei $I$ eine Instanz von 3-COLOR mit $G = (V,E)$. Die Idee der Reduktion ist, einen
weiteren Knoten zu $V$ hinzuzuf"ugen, der eine der 4 Farben von 4-COLOR zwingend
``verbraucht'', dass kein weiterer Knoten dieselbe Farbe haben darf. Dazu verbindet
man ihn einfach mit allen Knoten in $V$. Die 4-COLOR Instanz $I^\prime$ hat also
$G^\prime = (V \cup \{v^\prime\}, E \cup \{(v,v^\prime) \; | \; v \in V\})$. Ist
$G$ dreif"arbbar, so behalten wir die F"arbung der Knoten von $G$ in $G^\prime$ bei
und f"arben $v^\prime$ in der vierten Farbe, $G^\prime$ ist dann also vierf"arbbar.
Ist $G^\prime$ vierf"arbbar, so hat kein anderer Knoten die selbe Farbe wie
$v^\prime$, da $v^\prime$ mit allen anderen Knoten verbunden ist. Durch Entfernen
von $v^\prime$ aus $G^\prime$ erh"alt man somit eine korrekte Dreif"arbung von $G$,
$G$ ist also dreif"arbbar.
\item
Wir geben einen Algorithmus an, der 3-CLIQUE in Polynomialzeit entscheidet. Da wir
in dieser Aufgabe von der Annahme $\mathcal{P} \neq \mathcal{NP}$ ausgehen, kann
3-CLIQUE nicht $\mathcal{NP}$-vollst"andig sein, denn sonst w"are $\mathcal{P} =
\mathcal{NP}$. Das CLIQUE-Problem geh"ort zu den ``fixed parameter tractable''-
Problemen. Das hei"st, ist die Gr"o"se einer Clique fest vorgegeben, in diesem
Fall 3, so ist das zugeh"orige Entscheidungsproblem in Polynomialzeit l"osbar.
Der Algorithmus verfolgt dabei schlicht den Brute-Force-Ansatz: Man betrachtet
alle Tripel von Knoten und testet, ob diese eine Dreierclique bilden. Ist nach
durchlaufen aller Tripel eine Dreierclique gefunden, so gibt man als Ausgabe ``JA''
aus, ansonsten ``NEIN''. Es gibt insgesamt ${|V| \choose 3}$ Tripel von Knoten,
daher ist der Aufwand des Algorithmus im wesentlichen $O({|V| \choose 3}) =
O(|V|^3)$, was polynomial in der Eingabe ist.
\end{enumerate}

\subsection*{L"osung zu Aufgabe 3}
\begin{enumerate}
\item
Seien $R_{L_1}, R_{L_2}$ die zu den $\mathcal{NP}$-Sprachen $L_1, L_2$ geh"orenden,
polynomiell entscheidbaren Zeugenrelationen. Dann ist
\begin{eqnarray*}
R_{L_1 \cap L_2} := \{(x,(w_1,w_2)) \; | \; (x,w_1) \in R_{L_1} \wedge (x,w_2) \in
R_{L_2}\}
\end{eqnarray*}
eine zu $L_1 \cap L_2$ geh"orende, polynomiell entscheidbare Zeugenrelation, denn es
gilt:
\begin{eqnarray*}
&x \in L_1 \cap L_2 &\Leftrightarrow x \in L_1 \wedge x \in L_2\\
&&\Leftrightarrow \exists \; w_1,w_2: (x,w_1) \in R_{L_1} \wedge (x,w_2) \in R_{L_2}\\
&&\Leftrightarrow \exists \; (w_1,w_2): (x,(w_1,w_2)) \in R_{L_1 \cap L_2}
\end{eqnarray*}

\newpage

\item
Analog ist
\begin{eqnarray*}
R_{L_1 \cup L_2} := \{(x,(w_1,w_2)) \; | \; (x,w_1) \in R_{L_1} \vee (x,w_2) \in
R_{L_2}\}
\end{eqnarray*}
eine zu $L_1 \cup L_2$ geh"orende, polynomiell entscheidbare Zeugenrelation, denn es
gilt:
\begin{eqnarray*}
&x \in L_1 \cup L_2 &\Leftrightarrow x \in L_1 \vee x \in L_2\\
&&\Leftrightarrow \exists \; w_1,w_2: (x,w_1) \in R_{L_1} \vee (x,w_2) \in R_{L_2}\\
&&\Leftrightarrow \exists \; (w_1,w_2): (x,(w_1,w_2)) \in R_{L_1 \cup L_2}
\end{eqnarray*}
\end{enumerate}

\end{document}