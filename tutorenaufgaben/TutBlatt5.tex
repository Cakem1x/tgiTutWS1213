\documentclass[10pt,oneside,onecolumn,a4paper,german,titlepage]{article}
\usepackage[utf8]{inputenc}
\usepackage[german]{babel}
\usepackage{german}
\usepackage{amsmath,amsfonts,amssymb,latexsym,textcomp,stmaryrd}
\usepackage{tikz}
\usetikzlibrary{automata,positioning}
\pagestyle{plain}
\pagenumbering{arabic}
\parskip0.5ex plus0.1ex minus0.1ex
\parindent0pt
\voffset-1.04cm
\topmargin0pt
\headheight0pt
\headsep0pt
\topskip0pt
\textheight26.7cm
\footskip20pt
\hoffset-1.04cm
\oddsidemargin0pt
\evensidemargin0pt
\textwidth18cm
\marginparsep0pt
\marginparwidth0pt
\begin{document}

\newcommand{\start}[3]
{
  \draw (#1*2,#2*2) node{$#3$};
  \draw (#1*2,#2*2) circle(0.4cm);
  \draw [->] (#1*2-0.9,#2*2) -- (#1*2-0.4,#2*2);
}
\newcommand{\startfinal}[3]
{
  \draw (#1*2,#2*2) node{$#3$};
  \draw (#1*2,#2*2) circle(0.4cm);
  \draw (#1*2,#2*2) circle(0.32cm);
  \draw [->] (#1*2-0.9,#2*2) -- (#1*2-0.4,#2*2);
}
\newcommand{\state}[3]
{
  \draw (#1*2,#2*2) node{$#3$};
  \draw (#1*2,#2*2) circle(0.4cm);
}
\newcommand{\final}[3]
{
  \draw (#1*2,#2*2) node{$#3$};
  \draw (#1*2,#2*2) circle(0.4cm);
  \draw (#1*2,#2*2) circle(0.32cm);
}
\newcommand{\tol}[4]
{
  \draw (#1+#3,#2*2) node[above]{$#4$};
  \draw [->] (#1*2-0.4,#2*2) -- (#3*2+0.4,#2*2);
}
\newcommand{\tor}[4]
{
  \draw (#1+#3,#2*2) node[above]{$#4$};
  \draw [->] (#1*2+0.4,#2*2) -- (#3*2-0.4,#2*2);
}
\newcommand{\tot}[4]
{
  \draw (#1*2,#2+#3) node[right]{$#4$};
  \draw [->] (#1*2,#2*2+0.4) -- (#1*2,#3*2-0.4);
}
\newcommand{\tob}[4]
{
  \draw (#1*2,#2+#3) node[right]{$#4$};
  \draw [->] (#1*2,#2*2-0.4) -- (#1*2,#3*2+0.4);
}
\newcommand{\totl}[5]
{
  \draw (#1+#3,#2+#4) node[above right]{$#5$};
  \draw [->] (#1*2-0.283,#2*2+0.283) -- (#3*2+0.283,#4*2-0.283);
}
\newcommand{\totr}[5]
{
  \draw (#1+#3,#2+#4) node[above left]{$#5$};
  \draw [->] (#1*2+0.283,#2*2+0.283) -- (#3*2-0.283,#4*2-0.283);
}
\newcommand{\tobl}[5]
{
  \draw (#1+#3,#2+#4) node[below right]{$#5$};
  \draw [->] (#1*2-0.283,#2*2-0.283) -- (#3*2+0.283,#4*2+0.283);
}
\newcommand{\tobr}[5]
{
  \draw (#1+#3,#2+#4) node[below left]{$#5$};
  \draw [->] (#1*2+0.283,#2*2-0.283) -- (#3*2-0.283,#4*2+0.283);
}
\newcommand{\rloopl}[3]
{
  \draw (#1*2-1,#2*2) node[left]{$#3$};
  \draw [->] (#1*2-0.35,#2*2-0.2) arc (-30:-320:0.32cm);
}
\newcommand{\rloopr}[3]
{
  \draw (#1*2+1,#2*2) node[right]{$#3$};
  \draw [->] (#1*2+0.35,#2*2+0.2) arc (150:-140:0.32cm);
}
\newcommand{\rloopt}[3]
{
  \draw (#1*2,#2*2+1) node[above]{$#3$};
  \draw [->] (#1*2-0.2,#2*2+0.35) arc (240:-50:0.32cm);
}
\newcommand{\rloopb}[3]
{
  \draw (#1*2,#2*2-1) node[below]{$#3$};
  \draw [->] (#1*2+0.2,#2*2-0.35) arc (60:-230:0.32cm);
}
\newcommand{\lloopl}[3]
{
  \draw (#1*2-1,#2*2) node[left]{$#3$};
  \draw [->] (#1*2-0.35,#2*2+0.2) arc (30:320:0.32cm);
}
\newcommand{\lloopr}[3]
{
  \draw (#1*2+1,#2*2) node[right]{$#3$};
  \draw [->] (#1*2+0.35,#2*2-0.2) arc (-150:140:0.32cm);
}
\newcommand{\lloopt}[3]
{
  \draw (#1*2,#2*2+1) node[above]{$#3$};
  \draw [->] (#1*2+0.2,#2*2+0.35) arc (-60:230:0.32cm);
}
\newcommand{\lloopb}[3]
{
  \draw (#1*2,#2*2-1) node[below]{$#3$};
  \draw [->] (#1*2-0.2,#2*2-0.35) arc (-240:50:0.32cm);
}
\newcommand{\cyksymbol}[3]
{
  \draw (#2*#1-#1*0.5,-0.25) node {#3};
}
\newcommand{\cykfield}[4]
{
  \draw (#2*#1-#1+#3*#1*0.5-#1*0.5,#3*0.5-0.5) rectangle
        (#2*#1   +#3*#1*0.5-#1*0.5,#3*0.5    );
  \draw (#2*#1-#1+#3*#1*0.5       ,#3*0.5-0.5+0.25) node{$#4$};
}

\section*{Tutorien-"Ubungsblatt 5}

\subsection*{Aufgabe 1}
Gegeben sei folgende Grammatik $\mathcal{G} = (\{a,b,c\},\{S,A,B,C\},S,\mathcal{P})$
mit\\
$\mathcal{P} = \{S \rightarrow aABcb \; | \; aBC \; | \; CBABc,$\\
$aA \rightarrow BCaAa \; | \; bb,$\\
$BC \rightarrow CB,$\\
$B \rightarrow aCaa \; | \; babAcc,$\\
$aCa \rightarrow aAc \; | \; aca\}$.\\[4pt]
L"osen Sie das Wortproblem anhand des in der Vorlesung angegebenen Algorithmus f"ur
das Wort $w = acbbca$!

\subsection*{Aufgabe 2}
Geben Sie, sofern m"oglich, je eine L"osung f"ur die folgenden Post-Systeme an!
Begr"unden Sie gegebenenfalls,\\
warum es keine L"osung geben kann!
\begin{enumerate}
\item \begin{displaymath} \{ {aa \choose a}, {b \choose aa}, {a \choose aab} \}
\end{displaymath}
\item \begin{displaymath} \{ {01 \choose 0}, {0 \choose 101}, {101 \choose 0} \}
\end{displaymath}
\end{enumerate}

\subsection*{Aufgabe 3}
Geben Sie f"ur die nachfolgenden Sprachen "uber dem Alphabet $\Sigma = \{0,1\}$
jeweils eine Turingmaschine an,\\
die die entsprechende Sprache akzeptiert!
\begin{enumerate}
\item $\mathcal{L} = \{ \omega \in \{0,1\}^* \; | \; \omega \;
\mbox{enth"alt das Zeichen $0$ gleich oft wie das Zeichen $1$} \}$
\item $\mathcal{L}' = \{ \omega \in \{0,1\}^* \; | \; \omega \;
\mbox{enth"alt das Zeichen $0$ doppelt so oft wie das Zeichen $1$} \}$
\item $\mathcal{L}'' = \{ \omega \in \{0,1\}^* \; | \; \omega \;
\mbox{enth"alt das Zeichen $0$ nicht doppelt so oft wie das Zeichen $1$} \}$
\end{enumerate}

\subsection*{Aufgabe 4}
Zeigen Sie, dass die Sprache $\mathcal{L} = \{ \langle \mathcal{M} \rangle \; | \;
\mbox{Turingmaschine $\mathcal{M}$ akzeptiert jede Eingabe} \}$ nicht entscheidbar
ist!

\newpage

\subsection*{L"osung zu Aufgabe 1}
Mit $|\omega| = 6$ ergibt die Anwendung des Algorithmus für kontextfreie Grammatiken
folgende Tabelle:\\[4pt]
\begin{tabular}{c|l}
Durchlauf & $\mathcal{M}$ \\
\hline
0 & $\mathcal{M}_0 := \{S\}$ \\
1 & $\mathcal{M}_1 := \mathcal{M}_0 \cup \{aABcb,aBC,CBABc\}$ \\
2 & $\mathcal{M}_2 := \mathcal{M}_1 \cup \{bbBcb,aCB,aaCaaC\}$ \\
3 & $\mathcal{M}_3 := \mathcal{M}_2 \cup \{aCaCaa,aaAcaC,aacaaC\}$ \\
4 & $\mathcal{M}_4 := \mathcal{M}_3 \cup \{abbcaC,aAcCaa,acaCaa,aCaAca,aCacaa\}$ \\
5 & $\mathcal{M}_5 := \mathcal{M}_4 \cup \{bbcCaa,aCbbca,acaAca,acacaa,aAcAca,
aAccaa\}$ \\
6 & $\mathcal{M}_6 := \mathcal{M}_5 \cup \{acbbca,...\}$ bzw. 
$\mathcal{M}_6 := \mathcal{M}_5 \cup \{acbbca,bbcAca,bbccaa\}$ \\
7 & $\mathcal{M}_7 := \mathcal{M}_6 \cup \varnothing$
\end{tabular}\\[4pt]
$acbbca \in \mathcal{M}_6 \Rightarrow acbbca \in \mathcal{L}(\mathcal{G})$

\subsection*{L"osung zu Aufgabe 2}
\begin{enumerate}
\item Die k"urzeste L"osung ist $(1,3,2,1,1)$, welche das L"osungswort $aaabaaaa$
erzeugt.
\item Jede L"osung mu"s mit einem Wortpaar enden, bei dem das obere und das untere
Wort auf dem gleichen\\
Zeichen enden oder bei dem eines von den beiden W"ortern das leere Wort ist. Beides
ist hier nicht der\\
Fall, weswegen keine L"osung existieren kann.
\end{enumerate}

\subsection*{L"osung zu Aufgabe 3}
\begin{enumerate}
\item Turingmaschine zu $\mathcal{L}$:
\begin{center}
\begin{tikzpicture}[>=latex, shorten >=1pt, auto]
  \draw (175.0pt, -298.0pt)node[state](0){$q_0$};
  \draw (365.0pt, -298.0pt)node[state](1){$r$};
  \draw (547.0pt, -298.0pt)node[state](2){$q_1$};
  \draw (273.0pt, -115.0pt)node[state, initial, initial text =](3){$s$};
  \draw (475.0pt, -115.0pt)node[state, accepting](4){$f$};
  \draw (362.0pt, -417.0pt)node[state](5){$e$};
  \path[->] (1) edge[loop above] node[text width=2cm,text centered]{1 : 1, L\\ 0 : 0, L\\ E : E, L}(1);
  \path[->] (5) edge[loop above] node{$\square$ : $\square$, N}(5);
  \path[->] (3) edge node{$\square$ : $\square$, N}(4);
  \path[->] (2) edge node{1 : E, L}(1);
  \path[->] (0) edge node{$\square$ : $\square$, N}(5);
  \path[->] (2) edge[loop above] node[text width=2cm,text centered]{0 : 0, R\\ E : E, R}(2);
  \path[->] (3) edge[loop above] node{E : E, R}(3);
  \path[->] (0) edge[loop above] node[text width=2cm,text centered]{E : E, R\\ 1 : 1, R}(0);
  \path[->] (0) edge node{0 : E, L}(1);
  \path[->] (1) edge node{$\square$ : $\square$, R}(3);
  \path[->] (2) edge node{$\square$ : $\square$, N}(5);
  \path[->] (3) edge node{1 : E, R}(0);
  \path[->] (3) edge node{0 : E, R}(2);
\end{tikzpicture}
\end{center}

\newpage

\item Turingmaschine zu $\mathcal{L}'$:
\begin{center}
\begin{tikzpicture}[>=latex, shorten >=1pt, auto, scale=0.85]
  \draw (244.0pt, -344.0pt)node[state](0){$q_0$};
  \draw (341.0pt, -233.0pt)node[state](1){$r$};
  \draw (438.0pt, -285.0pt)node[state](2){$q_1$};
  \draw (268.0pt, -50.0pt)node[state, initial, initial text =](3){$s$};
  \draw (399.0pt, -50.0pt)node[state, accepting](4){$f$};
  \draw (350.0pt, -478.0pt)node[state](5){$e$};
  \draw (135.0pt, -344.0pt)node[state](6){$q_{00}$};
  \draw (631.0pt, -344.0pt)node[state](7){$q_{01}$};
  \path[->] (1) edge node{$\square$ : $\square$, R}(3);
  \path[->] (6) edge node{0 : E, R}(0);
  \path[->] (1) edge[loop above] node[text width=2cm,text centered]{E : E, L\\ 0 : 0, L\\ 1 : 1, L}(1);
  \path[->] (6) edge[loop above] node[text width=2cm,text centered]{1 : 1, R\\ E : E, R}(6);
  \path[->] (2) edge[loop above] node[text width=2cm,text centered]{E : E, R\\ 0 : 0, R}(2);
  \path[->] (0) edge[loop above] node[text width=2cm,text centered]{1 : 1, R\\ E : E, R}(0);
  \path[->] (3) edge[loop above] node{E : E, R}(3);
  \path[->] (7) edge[loop above] node{E : E, R}(7);
  \path[->] (0) edge node{$\square$ : $\square$, N}(5);
  \path[->] (7) edge node{0 : E, R}(2);
  \path[->] (7) edge node{1 : E, R}(0);
  \path[->] (2) edge node{1 : E, L}(1);
  \path[->] (5) edge[loop above] node{$\square$ : $\square$, N}(5);
  \path[->] (3) edge node{$\square$ : $\square$, N}(4);
  \path[->] (6) edge node{$\square$ : $\square$, N}(5);
  \path[->] (0) edge node{0 : E, L}(1);
  \path[->] (3) edge node{1 : E, R}(6);
  \path[->] (3) edge node{0 : E, R}(7);
  \path[->] (2) edge node{$\square$ : $\square$, N}(5);
  \path[->] (7) edge node{$\square$ : $\square$, N}(5);
\end{tikzpicture}
\end{center}

\newpage

\item Turingmaschine zu $\mathcal{L}''$:
\begin{center}
\begin{tikzpicture}[>=latex, shorten >=1pt, auto, scale=0.85]
  \draw (244.0pt, -344.0pt)node[state](0){$q_0$};
  \draw (341.0pt, -233.0pt)node[state](1){$r$};
  \draw (438.0pt, -285.0pt)node[state](2){$q_1$};
  \draw (268.0pt, -50.0pt)node[state, initial, initial text =](3){$s$};
  \draw (399.0pt, -50.0pt)node[state](4){$e$};
  \draw (350.0pt, -478.0pt)node[state, accepting](5){$f$};
  \draw (135.0pt, -344.0pt)node[state](6){$q_{00}$};
  \draw (631.0pt, -344.0pt)node[state](7){$q_{01}$};
  \path[->] (1) edge node{$\square$ : $\square$, R}(3);
  \path[->] (6) edge node{0 : E, R}(0);
  \path[->] (1) edge[loop above] node[text width=2cm,text centered]{E : E, L\\ 0 : 0, L\\ 1 : 1, L}(1);
  \path[->] (6) edge[loop above] node[text width=2cm,text centered]{1 : 1, R\\ E : E, R}(6);
  \path[->] (2) edge[loop above] node[text width=2cm,text centered]{E : E, R\\ 0 : 0, R}(2);
  \path[->] (0) edge[loop above] node[text width=2cm,text centered]{1 : 1, R\\ E : E, R}(0);
  \path[->] (3) edge[loop above] node{E : E, R}(3);
  \path[->] (7) edge[loop above] node{E : E, R}(7);
  \path[->] (0) edge node{$\square$ : $\square$, N}(5);
  \path[->] (7) edge node{0 : E, R}(2);
  \path[->] (7) edge node{1 : E, R}(0);
  \path[->] (2) edge node{1 : E, L}(1);
  \path[->] (4) edge[loop above] node{$\square$ : $\square$, N}(4);
  \path[->] (3) edge node{$\square$ : $\square$, N}(4);
  \path[->] (6) edge node{$\square$ : $\square$, N}(5);
  \path[->] (0) edge node{0 : E, L}(1);
  \path[->] (3) edge node{1 : E, R}(6);
  \path[->] (3) edge node{0 : E, R}(7);
  \path[->] (2) edge node{$\square$ : $\square$, N}(5);
  \path[->] (7) edge node{$\square$ : $\square$, N}(5);
\end{tikzpicture}
\end{center}
Es wurden im Vergleich zur vorherigen Turingmaschine nur die Zust"ande $f$ und $e$
vertauscht!
\end{enumerate}

\subsection*{L"osung zu Aufgabe 4}
\underline{Beweis:} Zeige die Reduktion $\mbox{HALT} \leq \mathcal{L}$!\\
Konstruiere dazu aus einer Instanz $(\langle \mathcal{M} \rangle,w) \in \mbox{HALT}$,
also aus der Turingmaschine $\mathcal{M}$ und deren Eingabe $w$,\\
eine neue Turingmaschine $\mathcal{M}'$:\\
- Leere das Band\\
- Schreibe $w$ auf das Band\\
- Simuliere $\mathcal{M}$\\
- Gehe in einen akzeptierenden Zustand\\[4pt]
Sei nun $f: \mbox{HALT} \to \mathcal{L}, (\langle \mathcal{M} \rangle,w) \mapsto \langle \mathcal{M}' \rangle$ die totale und berechenbare Funktion, die die Reduktion nach\\
der obigen Beschreibung liefert. Dann gilt:\\[4pt]
$(\langle \mathcal{M} \rangle,w) \in \mbox{HALT} \Leftrightarrow \mathcal{M} \;
\mbox{h"alt bei der Eingabe von} \; w \Leftrightarrow \mathcal{M}' \;
\mbox{akzeptiert jede Eingabe} \Leftrightarrow$\\
$\langle\mathcal{M}' \rangle = f((\langle \mathcal{M} \rangle,w)) \in \mathcal{L}$\\[4pt]
Damit ergibt sich aus der Annahme, dass $\mathcal{L}$ berechenbar ist, direkt,
dass auch $\mbox{HALT}$ berechenbar ist. Dies ist\\
aber ein Widerspruch, da $\mbox{HALT}$ als nicht berechenbar bekannt ist. Damit
kann also $\mathcal{L}$ nicht berechenbar sein.

\end{document}