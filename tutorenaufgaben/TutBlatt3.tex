\documentclass[10pt,oneside,onecolumn,a4paper,german,titlepage]{article}
\usepackage[utf8]{inputenc}
\usepackage[german]{babel}
\usepackage{german}
\usepackage{amsmath,amsfonts,amssymb,latexsym,textcomp}
\usepackage{tikz}
\pagestyle{plain}
\pagenumbering{arabic}
\parskip0.5ex plus0.1ex minus0.1ex
\parindent0pt
\voffset-1.04cm
\topmargin0pt
\headheight0pt
\headsep0pt
\topskip0pt
\textheight26.7cm
\footskip20pt
\hoffset-1.04cm
\oddsidemargin0pt
\evensidemargin0pt
\textwidth18cm
\marginparsep0pt
\marginparwidth0pt
\begin{document}

\newcommand{\start}[3]
{
  \draw (#1*2,#2*2) node{$#3$};
  \draw (#1*2,#2*2) circle(0.4cm);
  \draw [->] (#1*2-0.9,#2) -- (#1*2-0.4,#2);
}
\newcommand{\startfinal}[3]
{
  \draw (#1*2,#2*2) node{$#3$};
  \draw (#1*2,#2*2) circle(0.4cm);
  \draw (#1*2,#2*2) circle(0.32cm);
  \draw [->] (#1*2-0.9,#2) -- (#1*2-0.4,#2);
}
\newcommand{\state}[3]
{
  \draw (#1*2,#2*2) node{$#3$};
  \draw (#1*2,#2*2) circle(0.4cm);
}
\newcommand{\final}[3]
{
  \draw (#1*2,#2*2) node{$#3$};
  \draw (#1*2,#2*2) circle(0.4cm);
  \draw (#1*2,#2*2) circle(0.32cm);
}
\newcommand{\tol}[4]
{
  \draw (#1+#3,#2*2) node[above]{$#4$};
  \draw [->] (#1*2-0.4,#2*2) -- (#3*2+0.4,#2*2);
}
\newcommand{\tor}[4]
{
  \draw (#1+#3,#2*2) node[above]{$#4$};
  \draw [->] (#1*2+0.4,#2*2) -- (#3*2-0.4,#2*2);
}
\newcommand{\tot}[4]
{
  \draw (#1*2,#2+#3) node[right]{$#4$};
  \draw [->] (#1*2,#2*2+0.4) -- (#1*2,#3*2-0.4);
}
\newcommand{\tob}[4]
{
  \draw (#1*2,#2+#3) node[right]{$#4$};
  \draw [->] (#1*2,#2*2-0.4) -- (#1*2,#3*2+0.4);
}
\newcommand{\totl}[5]
{
  \draw (#1+#3,#2+#4) node[above right]{$#5$};
  \draw [->] (#1*2-0.283,#2*2+0.283) -- (#3*2+0.283,#4*2-0.283);
}
\newcommand{\totr}[5]
{
  \draw (#1+#3,#2+#4) node[above left]{$#5$};
  \draw [->] (#1*2+0.283,#2*2+0.283) -- (#3*2-0.283,#4*2-0.283);
}
\newcommand{\tobl}[5]
{
  \draw (#1+#3,#2+#4) node[below right]{$#5$};
  \draw [->] (#1*2-0.283,#2*2-0.283) -- (#3*2+0.283,#4*2+0.283);
}
\newcommand{\tobr}[5]
{
  \draw (#1+#3,#2+#4) node[below left]{$#5$};
  \draw [->] (#1*2+0.283,#2*2-0.283) -- (#3*2-0.283,#4*2+0.283);
}
\newcommand{\rloopl}[3]
{
  \draw (#1*2-1,#2*2) node[left]{$#3$};
  \draw [->] (#1*2-0.35,#2*2-0.2) arc (-30:-320:0.32cm);
}
\newcommand{\rloopr}[3]
{
  \draw (#1*2+1,#2*2) node[right]{$#3$};
  \draw [->] (#1*2+0.35,#2*2+0.2) arc (150:-140:0.32cm);
}
\newcommand{\rloopt}[3]
{
  \draw (#1*2,#2*2+1) node[above]{$#3$};
  \draw [->] (#1*2-0.2,#2*2+0.35) arc (240:-50:0.32cm);
}
\newcommand{\rloopb}[3]
{
  \draw (#1*2,#2*2-1) node[below]{$#3$};
  \draw [->] (#1*2+0.2,#2*2-0.35) arc (60:-230:0.32cm);
}
\newcommand{\lloopl}[3]
{
  \draw (#1*2-1,#2*2) node[left]{$#3$};
  \draw [->] (#1*2-0.35,#2*2+0.2) arc (30:320:0.32cm);
}
\newcommand{\lloopr}[3]
{
  \draw (#1*2+1,#2*2) node[right]{$#3$};
  \draw [->] (#1*2+0.35,#2*2-0.2) arc (-150:140:0.32cm);
}
\newcommand{\lloopt}[3]
{
  \draw (#1*2,#2*2+1) node[above]{$#3$};
  \draw [->] (#1*2+0.2,#2*2+0.35) arc (-60:230:0.32cm);
}
\newcommand{\lloopb}[3]
{
  \draw (#1*2,#2*2-1) node[below]{$#3$};
  \draw [->] (#1*2-0.2,#2*2-0.35) arc (-240:50:0.32cm);
}
\newcommand{\cyksymbol}[3]
{
  \draw (#2*#1-#1*0.5,-0.25) node {#3};
}
\newcommand{\cykfield}[4]
{
  \draw (#2*#1-#1+#3*#1*0.5-#1*0.5,#3*0.5-0.5) rectangle
        (#2*#1   +#3*#1*0.5-#1*0.5,#3*0.5    );
  \draw (#2*#1-#1+#3*#1*0.5       ,#3*0.5-0.5+0.25) node{$#4$};
}

\section*{Tutorien-"Ubungsblatt 3}

\subsection*{Aufgabe 1}
Gegeben sei die Sprache $\mathcal{L} = \{w \in \{a,b\}^* \; | \; w \;
\mbox{enth"alt gleich viele $a$ wie $b$}\}$.
\begin{enumerate}
\item Wie lautet das Pumping Lemma? Was genau muss man zeigen, falls man die
Kontraposition des\\
Pumping Lemmas verwenden will?
\item Zeigen Sie mit Hilfe des Pumping Lemmas, dass L nicht regul"ar ist!
\item Zeigen Sie mit Hilfe des Pumping Lemmas, dass die Sprache $\mathcal{L}' =
\{a^p\; | \; p \; \mbox{Primzahl}\}$ nicht regul"ar ist.
\item Betrachten Sie nun die Sprache $\mathcal{L}'' = \{a,aab,aaab\}$! Ist diese
regul"ar? Falls ja, geben Sie einen endlichen\\
Automaten an, der diese Sprache akzeptiert! Kann man mit dem Pumping Lemma zeigen,
dass die\\
Sprache regul"ar ist?
\end{enumerate}

\subsection*{Aufgabe 2}
Gegeben sei die folgende Grammatik: $\mathcal{G} = (\mathcal{T},\mathcal{V},S,
\mathcal{P})$ mit\\
$\mathcal{T} := \{a,b,c,d\}$, $\mathcal{V} := \{S,A,D,M\}$, $\mathcal{P} := \{
S \rightarrow AMD \; | \; M, A \rightarrow AA \; | \; a, D \rightarrow DD \; | \; d,
M \rightarrow bMc \; | \; \lambda\}$
\begin{enumerate}
\item Geben Sie die erzeugte Sprache an!
\item Wandeln Sie die gegebene kontextfreie Grammatik $\mathcal{G}$ in eine
"aquivalente kontextfreie Grammatik $\mathcal{G}'$ in\\
Chomsky-Normalform um, indem sie jeden Schritt durch eine neue Grammatik beschreiben!
\item Zeigen oder widerlegen Sie mit Hilfe des CYK-Algorithmus, ob die folgenden
W"orter in der Sprache $\mathcal{L}$\\
liegen, die durch die Grammatik $\mathcal{G}$ erzeugt wird!
\begin{enumerate}
\item $aabbccdd$
\item $abbcc$
\item $abcdd$
\end{enumerate}
\end{enumerate}

\subsection*{Aufgabe 3}
Gegeben sei folgende Sprache f"ur das Alphabet $\Sigma = \{a,b,c\}$:
\begin{eqnarray*}
\mathcal{L} = \{w_1w_2 \in \Sigma^* \; | \; w_1 \in \{a,b\}^*,w_2 \in \{b,c\}^*,
\#_a w_1 + \#_b w_1 = \#_b w_2 + \#_c w_2\}
\end{eqnarray*}
Hier gibt $\#_x w$ die H"aufigkeit des Vorkommens eines Zeichens $x \in \Sigma$ in
einem Wort $w \in \Sigma^*$ an.
\begin{enumerate}
\item Zeigen Sie, dass $\mathcal{L}$ nicht regul"ar ist!
\item Geben Sie eine Chomsky-2-Grammatik an, die genau die Sprache $\mathcal{L}$
erzeugt!
\item Geben Sie einen Kellerautomaten $\mathcal{M}$ an, der genau die Sprache
$\mathcal{L}$ erkennt! Zeichnen Sie den\\
Zustands"ubergangsgraphen f"ur $\mathcal{M}$!
\end{enumerate}

\newpage

\subsection*{L"osung zu Aufgabe 1}
\begin{enumerate}
\item Das Pumping Lemma: Sei $L$ eine Sprache "uber einem Alphabet $\Sigma$.\\
$L \; \mbox{regul"ar} \Rightarrow \exists \; n \in \mathbb{N}:
\forall \; \omega \in L, |\omega| \geq n: \exists \; \alpha,\beta,\gamma \in
\Sigma^{*}, \omega = \alpha\beta\gamma:$\\
$\beta \not= \lambda \wedge |\alpha\beta| \leq n \wedge
\forall \; i \in \mathbb{N}_0: \alpha\beta^i\gamma \in L$\\[4pt]
Kontraposition des Pumping Lemmas: Sei $L$ eine Sprache "uber einem Alphabet
$\Sigma$.\\
$\forall \; n \in \mathbb{N}: \exists \; \omega \in L, |\omega| \geq n:
\forall \; \alpha,\beta,\gamma \in \Sigma^{*}, \omega = \alpha\beta\gamma:$\\
$\beta = \lambda \vee |\alpha\beta| > n \vee \exists \; i \in \mathbb{N}_0:
\alpha\beta^i\gamma \notin L \Rightarrow L \; \mbox{nicht regul"ar}$\\[4pt]
Falls man also die Kontraposition des Pumping Lemmas verwenden will, muss man
zu \underline{jeder nat"urlichen Zahl} als Mindestwortl"ange \underline{ein
beliebiges, frei w"ahlbares} Wort $\omega$ finden, dass \underline{jede m"ogliche}
Zerlegung von $\omega$ eine der drei Eigenschaften des Pumping Lemmas verletzt.
Dazu betrachtet man der Einfachheit wegen nur alle Zerlegungen von $\omega$, die
die ersten beiden Eigenschaften des Pumping Lemmas erf"ullen, und zeigt dann, dass
diese die dritte Eigenschaft verletzen.
\item \underline{Behauptung:} $\mathcal{L}$ nicht regul"ar\\
\underline{Beweis:} Verwendung der Kontraposition des Pumping Lemmas\\
Sei $n_0 \in \mathbb{N}$ beliebig.\\
W"ahle $\omega \in \mathcal{L}$ geeignet mit $|\omega| \geq n_0$, w"ahle also
beispielsweise $\omega = a^{n_0}b^{n_0}$. $\Rightarrow$\\
$\forall \; \alpha,\beta,\gamma \in \{a,b\}^*, \omega = \alpha\beta\gamma,
\beta \not= \lambda, |\alpha\beta| \leq n_0: \alpha\beta = a^k$ mit
$k \in \mathbb{N}_0, 1 \leq k \leq n_0$\\
Seien also $\beta = a^i$ mit $i \in \mathbb{N}_0, 1 \leq i \leq n_0$,
$\alpha = a^j$ mit $j \in \mathbb{N}_0, 0 \leq j \leq n_0-i$ und
$\gamma = a^{n_0-i-j}b^{n_0}$. $\Rightarrow$\\
Es gelten $\omega=\alpha\beta\gamma,\beta \not= \lambda,|\alpha\beta| \leq n_0$.\\
Aber $\alpha\beta^2\gamma = a^{n_0+i}b^{n_0} \notin L$ wegen $i \geq 1$.
$\Rightarrow L \; \mbox{nicht regul"ar}$ 
\item \underline{Behauptung:} $\mathcal{L}'$ nicht regul"ar\\
\underline{Beweis:} Verwendung der Kontraposition des Pumping Lemmas\\
Sei $n_0 \in \mathbb{N}$ beliebig.\\
W"ahle $\omega \in \mathcal{L}$ geeignet mit $|\omega| \geq n_0$, w"ahle also
beispielsweise $\omega = a^p$ mit $p \; \mbox{Primzahl}, p \geq n_0 + 2$.
$\Rightarrow$\\
$\forall \; \alpha,\beta,\gamma \in \{a\}^*, \omega = \alpha\beta\gamma,
\beta \not= \lambda, |\alpha\beta| \leq n_0: \alpha\beta = a^k$ mit
$k \in \mathbb{N}_0, 1 \leq k \leq n_0$\\
Seien also $\beta = a^i$ mit $i \in \mathbb{N}_0, 1 \leq i \leq n_0$,
$\alpha = a^j$ mit $j \in \mathbb{N}_0, 0 \leq j \leq n_0-i$ und
$\gamma = a^{p-i-j}$. $\Rightarrow$\\
Es gelten $\omega=\alpha\beta\gamma,\beta \not= \lambda,|\alpha\beta| \leq n_0$.\\
Aber $\alpha\beta^{p-i}\gamma = a^{j}a^{i\cdot(p-i)}a^{p-i-j} = a^{(i+1)\cdot(p-i)}
\notin L$, denn wegen $i \geq 1$ folgt $i+1 \geq 2$ und wegen $p \geq n_0 + 2$ und
$i+j \leq n_0$ folgt $p-i \geq p-i-j = p-(i+j) \geq p-n_0 \geq 2$.
$\Rightarrow L \; \mbox{nicht regul"ar}$
\item Ja, die Sprache ist regul"ar, denn \underline{jede endliche} Sprache ist
regul"ar. Folgender endliche Automat akzeptiert $\mathcal{L}''$:
\begin{center}
\begin{tikzpicture}[line width=1pt]
\start{0}{0}{q_0}
\tor{0}{0}{1}{a}
\final{1}{0}{q_1}
\tor{1}{0}{2}{a}
\state{2}{0}{q_2}
\tor{2}{0}{3}{a}
\tob{2}{0}{-1}{b}
\state{3}{0}{q_3}
\tobl{3}{0}{2}{-1}{b}
\final{2}{-1}{q_4}
\end{tikzpicture}
\end{center}
Man kann mit dem Pumping Lemma niemals zeigen, dass eine Sprache regul"ar ist.
\end{enumerate}

\subsection*{L"osung zu Aufgabe 2}
\begin{enumerate}
\item Die gesuchte Sprache ist $\mathcal{L}(\mathcal{G}) = \{a^mb^nc^nd^k \; | \;
m,k \in \mathbb{N}, n \in \mathbb{N}_0\} \cup \{b^nc^n \; | \; n \in \mathbb{N}_0\}$.
\item Wir gehen in 3 Schritten vor und konstruieren zu $\mathcal{G}$ "aquivalente
Grammatiken $\mathcal{G}_i = (\mathcal{T},\mathcal{V}_i,S,\mathcal{P}_i)$ f"ur
$i \in \{1,2,3\}$, die der Chomsky-Normalform immer n"aher kommen.
\begin{itemize}
\item F"ur jedes Terminalzeichen $a \in \mathcal{T}$ und f"ur jede Produktion, die
dieses Terminalzeichen $a$ auf der rechten Seite enth"alt, ausgenommen nat"urlich
Produktionen der Form $V \rightarrow a$ mit $V \in \mathcal{V}$, wird jedes Vorkommen
von $a$ in der Produktion durch ein \underline{neues} Nichtterminalzeichen $A$,
welches in die Variablenmenge aufgenommen wird, ersetzt und zus"atzlich wird noch
die Produktion $A \rightarrow a$ hinzugef"ugt.\\
Dies ergibt: $\mathcal{V}_1 := \{S,A,B,C,D,M\}$ und\\
$\mathcal{P}_1 := \{
S \rightarrow AMD \; | \; M, A \rightarrow AA \; | \; a, D \rightarrow DD \; | \; d,
M \rightarrow BMC \; | \; \lambda, B \rightarrow b, C \rightarrow c\}$
\item In diesem Schritt werden alle Produktionen der Form $V \rightarrow \lambda$
f"ur $V \in \mathcal{V}, V \not= S$ entfernt. Dazu m"ussen diese Produktion aber
vorher \underline{rekursiv} durch ihre ``Vorwegnahme'' mit den anderen Produktionen
``verschmolzen'' werden, es wird also f"ur jede Produktion mit einem der obigen $V$
auf der rechten Seite eine neue Produktion ohne dieses $V$ hinzugef"ugt.\\
Dies ergibt: $\mathcal{V}_2 := \{S,A,B,C,D,M\}$ und\\
$\mathcal{P}_2 := \{
S \rightarrow AMD \; | \; AD \; | \; M \; | \; \lambda, A \rightarrow AA \; | \; a,
D \rightarrow DD \; | \; d, M \rightarrow BMC \; | \; BC, B \rightarrow b,
C \rightarrow c\}$

\newpage

\item Im letzten Schritt werden die Produktionen, deren rechte Seite nur aus einer
einzigen oder mehr als zwei Variablen bestehen, durch einen Trick ersetzt. Bei den
einzelnen Variablen werden diese Produktionen wieder durch ihre ``Vorwegnahme'' mit
den anderen Produktionen ``verschmolzen''. F\"ur die Produktionen mit mehr als zwei
Variablen auf der rechten Seite werden \underline{neue} Nichterminale eingef\"uhrt
und dazu passende Produktionen hinzugef"ugt.\\
Dies ergibt: $\mathcal{V}_3 := \{S,A,B,C,D,M,X,Y\}$ und\\
$\mathcal{P}_3 := \{
S \rightarrow AX \; | \; AD \; | \; BY \; | \; BC \; | \; \lambda, A \rightarrow AA
\; | \; a, D \rightarrow DD \; | \; d, M \rightarrow BY \; | \; BC, X \rightarrow MD,
Y \rightarrow MC,$\\
$B \rightarrow b, C \rightarrow c\}$
\end{itemize}
\item
\begin{enumerate}
\item $aabbccdd \in \mathcal{L}$?
\begin{center}
\begin{tikzpicture}[line width=1pt]
\cyksymbol{1}{1}{a}
\cyksymbol{1}{2}{a}
\cyksymbol{1}{3}{b}
\cyksymbol{1}{4}{b}
\cyksymbol{1}{5}{c}
\cyksymbol{1}{6}{c}
\cyksymbol{1}{7}{d}
\cyksymbol{1}{8}{d}
\cykfield{1}{1}{1}{A}
\cykfield{1}{2}{1}{A}
\cykfield{1}{3}{1}{B}
\cykfield{1}{4}{1}{B}
\cykfield{1}{5}{1}{C}
\cykfield{1}{6}{1}{C}
\cykfield{1}{7}{1}{D}
\cykfield{1}{8}{1}{D}
\cykfield{1}{1}{2}{A}
\cykfield{1}{2}{2}{}
\cykfield{1}{3}{2}{}
\cykfield{1}{4}{2}{S,M}
\cykfield{1}{5}{2}{}
\cykfield{1}{6}{2}{}
\cykfield{1}{7}{2}{D}
\cykfield{1}{1}{3}{}
\cykfield{1}{2}{3}{}
\cykfield{1}{3}{3}{}
\cykfield{1}{4}{3}{Y}
\cykfield{1}{5}{3}{}
\cykfield{1}{6}{3}{}
\cykfield{1}{1}{4}{}
\cykfield{1}{2}{4}{}
\cykfield{1}{3}{4}{S,M}
\cykfield{1}{4}{4}{}
\cykfield{1}{5}{4}{}
\cykfield{1}{1}{5}{}
\cykfield{1}{2}{5}{*}
\cykfield{1}{3}{5}{X}
\cykfield{1}{4}{5}{}
\cykfield{1}{1}{6}{}
\cykfield{1}{2}{6}{S}
\cykfield{1}{3}{6}{X}
\cykfield{1}{1}{7}{S}
\cykfield{1}{2}{7}{S}
\cykfield{1}{1}{8}{S}
\end{tikzpicture}
\end{center}
Markierung $*$ f"ur n"achste Teilaufgabe!
\item $abbcc \in \mathcal{L}$?\\
Es gilt $abbcc \notin \mathcal{L}$, da dann das mit $*$ markierte Feld in der
vorherigen Teilaufgabe die Startvariable $S$ enthalten m"usste! Man kann also
die L"osung des CYK-Algorithmus zu einem bestimmten Wort f"ur Teilworte
weiterverwenden!
\item $abcdd \in \mathcal{L}$?
\begin{center}
\begin{tikzpicture}[line width=1pt]
\cyksymbol{1}{1}{a}
\cyksymbol{1}{2}{b}
\cyksymbol{1}{3}{c}
\cyksymbol{1}{4}{d}
\cyksymbol{1}{5}{d}
\cykfield{1}{1}{1}{A}
\cykfield{1}{2}{1}{B}
\cykfield{1}{3}{1}{C}
\cykfield{1}{4}{1}{D}
\cykfield{1}{5}{1}{D}
\cykfield{1}{1}{2}{}
\cykfield{1}{2}{2}{S,M}
\cykfield{1}{3}{2}{}
\cykfield{1}{4}{2}{D}
\cykfield{1}{1}{3}{}
\cykfield{1}{2}{3}{X}
\cykfield{1}{3}{3}{}
\cykfield{1}{1}{4}{S}
\cykfield{1}{2}{4}{X}
\cykfield{1}{1}{5}{S}
\end{tikzpicture}
\end{center}
\end{enumerate}
\end{enumerate}

\subsection*{L"osung zu Aufgabe 3}
\begin{enumerate}
\item Wir verwenden das Pumping-Lemma. Man w"ahlt das Wort $w = a^nc^n \in
\mathcal{L}$, wobei $n \in \mathbb{N}$ die Konstante aus dem Pumping-Lemma ist. Dann
gilt mit $w = \alpha\beta\gamma$, wobei $|\alpha\beta| \leq n$ und $\beta \not=
\lambda$, dass $\alpha$, $\beta$ und $\gamma$ von der Form $\alpha = a^{n-r_1-r_2}$,
$\beta = a^{r_1}$ ($r_1 > 0$), $\gamma = a^{r_2}c^n$ ($r_2 \geq 0$) sind. Damit ist
$w^\prime = \alpha\beta^2\gamma = a^{n+r_1}c^n \notin \mathcal{L}$, da die Zerlegung
von $w^\prime = w_1w_2$ eindeutig in $w_1 = a^{n+r_1}$ und $w_2 = c^{n}$ vorgegeben
ist und $\#_a w_1 \neq \#_c w_2$ gilt. Damit ist $\mathcal{L}$ nicht regul"ar.
\item Die Grammatik $\mathcal{G} = (\mathcal{T},\mathcal{V},S,\mathcal{P})$ mit den
Terminalen $\mathcal{T} = \{a,b,c\}$, den Nichtterminalen $\mathcal{V} = \{S,X\}$
und den Produktionen $\mathcal{P}$
\begin{eqnarray*}
S \rightarrow aXb \; | \; aXc \; | \; bXb \; | \; bXc \; | \; ab \; | \; ac \; | \;
bb \; | \; bc \; | \; \lambda \\
X \rightarrow aXb \; | \; aXc \; | \; bXb \; | \; bXc \; | \; ab \; | \; ac \; | \;
bb \; | \; bc
\end{eqnarray*}
leistet das Gew"unschte. Der Induktionsbeweis hierzu besteht darin zu zeigen, dass
f"ur alle Teilableitugen $w = w_1Xw_2$ gilt $w_1 \in \{a,b\}^*$, $w_2 \in \{b,c\}^*$
und $\#_a w_1 + \#_b w_1 = \#_b w_2 + \#_c w_2$. Man sieht recht einfach, dass der
erste Ableitungsschritt hierf"ur die Induktionsverankerung und der zweite den
Induktionsschritt liefert.
\item Ein Kellerautomat $\mathcal{M} = (\mathcal{Q},\Sigma,\Gamma,\delta,q_0,
\mathcal{F})$ mit $\Gamma = \{\#,T\}$ und der Akzeptionsbedingung ``leerer Keller'',
der die Sprache $\mathcal{L}$ akzeptiert, kann beispielsweise so aussehen:
\begin{center}
\begin{tikzpicture}[line width=1pt]
\state{0}{1}{q_1}
\rloopt{0}{1}{(a,\lambda,T),(b,\lambda,T)}
\tor{0}{1}{3}{(b,T,\lambda),(c,T,\lambda)}
\state{3}{1}{q_2}
\rloopt{3}{1}{(b,T,\lambda),(c,T,\lambda)}
\tob{3}{1}{0}{(\lambda,\#,\lambda)}
\startfinal{0}{0}{q_0}
\tot{0}{0}{1}{(\lambda,\lambda,\#)}
\final{3}{0}{q_3}
\end{tikzpicture}
\end{center}
\end{enumerate}

\end{document}