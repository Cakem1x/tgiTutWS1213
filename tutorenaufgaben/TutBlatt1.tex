\documentclass[10pt,oneside,onecolumn,a4paper,german,titlepage]{article}
\usepackage[utf8]{inputenc}
\usepackage[german]{babel}
\usepackage{german}
\usepackage{amsmath,amsfonts,amssymb,latexsym,textcomp}
\usepackage{tikz}
\pagestyle{plain}
\pagenumbering{arabic}
\parskip0.5ex plus0.1ex minus0.1ex
\parindent0pt
\voffset-1.04cm
\topmargin0pt
\headheight0pt
\headsep0pt
\topskip0pt
\textheight26.7cm
\footskip20pt
\hoffset-1.04cm
\oddsidemargin0pt
\evensidemargin0pt
\textwidth18cm
\marginparsep0pt
\marginparwidth0pt
\begin{document}

\newcommand{\start}[3]
{
  \draw (#1*2,#2*2) node{$#3$};
  \draw (#1*2,#2*2) circle(0.4cm);
  \draw [->] (#1*2-0.9,#2) -- (#1*2-0.4,#2);
}
\newcommand{\state}[3]
{
  \draw (#1*2,#2*2) node{$#3$};
  \draw (#1*2,#2*2) circle(0.4cm);
}
\newcommand{\final}[3]
{
  \draw (#1*2,#2*2) node{$#3$};
  \draw (#1*2,#2*2) circle(0.4cm);
  \draw (#1*2,#2*2) circle(0.32cm);
}
\newcommand{\tol}[4]
{
  \draw (#1+#3,#2*2) node[above]{$#4$};
  \draw [->] (#1*2-0.4,#2*2) -- (#3*2+0.4,#2*2);
}
\newcommand{\tor}[4]
{
  \draw (#1+#3,#2*2) node[above]{$#4$};
  \draw [->] (#1*2+0.4,#2*2) -- (#3*2-0.4,#2*2);
}
\newcommand{\tot}[4]
{
  \draw (#1*2,#2+#3) node[right]{$#4$};
  \draw [->] (#1*2,#2*2+0.4) -- (#1*2,#3*2-0.4);
}
\newcommand{\tob}[4]
{
  \draw (#1*2,#2+#3) node[right]{$#4$};
  \draw [->] (#1*2,#2*2-0.4) -- (#1*2,#3*2+0.4);
}
\newcommand{\totl}[5]
{
  \draw (#1+#3,#2+#4) node[above right]{$#5$};
  \draw [->] (#1*2-0.283,#2*2+0.283) -- (#3*2+0.283,#4*2-0.283);
}
\newcommand{\totr}[5]
{
  \draw (#1+#3,#2+#4) node[above left]{$#5$};
  \draw [->] (#1*2+0.283,#2*2+0.283) -- (#3*2-0.283,#4*2-0.283);
}
\newcommand{\tobl}[5]
{
  \draw (#1+#3,#2+#4) node[below right]{$#5$};
  \draw [->] (#1*2-0.283,#2*2-0.283) -- (#3*2+0.283,#4*2+0.283);
}
\newcommand{\tobr}[5]
{
  \draw (#1+#3,#2+#4) node[below left]{$#5$};
  \draw [->] (#1*2+0.283,#2*2-0.283) -- (#3*2-0.283,#4*2+0.283);
}
\newcommand{\rloopl}[3]
{
  \draw (#1*2-1,#2*2) node[left]{$#3$};
  \draw [->] (#1*2-0.35,#2*2-0.2) arc (-30:-320:0.32cm);
}
\newcommand{\rloopr}[3]
{
  \draw (#1*2+1,#2*2) node[right]{$#3$};
  \draw [->] (#1*2+0.35,#2*2+0.2) arc (150:-140:0.32cm);
}
\newcommand{\rloopt}[3]
{
  \draw (#1*2,#2*2+1) node[above]{$#3$};
  \draw [->] (#1*2-0.2,#2*2+0.35) arc (240:-50:0.32cm);
}
\newcommand{\rloopb}[3]
{
  \draw (#1*2,#2*2-1) node[below]{$#3$};
  \draw [->] (#1*2+0.2,#2*2-0.35) arc (60:-230:0.32cm);
}
\newcommand{\lloopl}[3]
{
  \draw (#1*2-1,#2*2) node[left]{$#3$};
  \draw [->] (#1*2-0.35,#2*2+0.2) arc (30:320:0.32cm);
}
\newcommand{\lloopr}[3]
{
  \draw (#1*2+1,#2*2) node[right]{$#3$};
  \draw [->] (#1*2+0.35,#2*2-0.2) arc (-150:140:0.32cm);
}
\newcommand{\lloopt}[3]
{
  \draw (#1*2,#2*2+1) node[above]{$#3$};
  \draw [->] (#1*2+0.2,#2*2+0.35) arc (-60:230:0.32cm);
}
\newcommand{\lloopb}[3]
{
  \draw (#1*2,#2*2-1) node[below]{$#3$};
  \draw [->] (#1*2-0.2,#2*2-0.35) arc (-240:50:0.32cm);
}

\section*{Tutorien-Übungsblatt 1}

\subsection*{Aufgabe 1}
Gegeben sei der folgende endliche Automat:\\
$\mathcal{M} = (\mathcal{Q},\Sigma,\delta,S,\mathcal{F})$ mit
$\Sigma = \{a,b\}$, $\mathcal{Q} = \{S,B,C,D\}$, $\mathcal{F} = \{B,C\}$
und $\delta$ gegeben durch:
\begin{center}
\begin{tikzpicture}[line width=1pt]
\start{0}{0}{S}
\totr{0}{0}{1}{1}{a}
\tobr{0}{0}{1}{-1}{b}
\final{1}{1}{B}
\rloopt{1}{1}{a}
\draw (3,1) node[above right]{$b$};
\draw [->] (2.283,1.717) -- (3.717,0.283);
\final{1}{-1}{C}
\draw (3,-1) node[below right]{$a$};
\draw [->] (2.283,-1.717) -- (3.717,-0.283);
\rloopb{1}{-1}{b}
\state{2}{0}{D}
\rloopr{2}{0}{a,b}
\end{tikzpicture}
\end{center}
\begin{enumerate}
\item Geben Sie die von diesem Automaten akzeptierte Sprache in einem regulären
Ausdruck an!
\item Um was für einen Automaten handelt es sich?
\item Konstruieren Sie einen äquivalenten endlichen Automaten, der nur einen
einzigen Endzustand besitzt!
\item Geben Sie eine linkslineare Grammatik für die Sprache dieses Automaten an,
die keine überflüssigen\\
Nichtterminale und Regeln enthält!
\end{enumerate}

\subsection*{Aufgabe 2}
\begin{enumerate}
\item Formulieren Sie einen regulären Ausdruck über dem Alphabet $\Sigma =
\{0,1\}$, der jedes beliebige Wort erfasst,\\
wobei die vorletzte Ziffer $0$ sein soll!
\item Geben Sie für diese Sprache den möglichst größten Chomsky-Typ und eine
zugehörige Grammatik an!
\item Geben Sie einen dazugehörigen Automaten an, der diese Sprache akzeptiert!
\end{enumerate}

\subsection*{Aufgabe 3}
Gegeben sei der folgende endliche Akzeptor $\mathcal{M}$ mit dem Eingabealphabet
$\Sigma = \{a,b,c,d\}$:
\begin{center}
\begin{tikzpicture}[line width=1pt]
\start{0}{0}{q_0}
\tor{0}{0}{1}{a,b}
\state{1}{0}{q_1}
\rloopt{1}{0}{a,b}
\totr{1}{0}{3}{1}{c}
\state{3}{1}{q_2}
\rloopt{3}{1}{c}
\draw (8,1) node[above right]{$d$};
\draw [->] (6.283,1.717) -- (9.717,0.283);
\final{5}{0}{q_3}
\tol{5}{0}{1}{d}
\end{tikzpicture}
\end{center}
\begin{enumerate}
\item Welche Sprache $\mathcal{L}(\mathcal{M})$ wird von dem Akzeptor $\mathcal{M}$
akzeptiert?
\item Konstruieren Sie aus $\mathcal{M}$ eine rechtslineare Grammatik, die
$\mathcal{L}(\mathcal{M})$ erzeugt!
\end{enumerate}

\subsection*{Aufgabe 4}
Die Sprache $\mathcal{L}$ sei durch den regulären Ausdruck $(aa^*b^*)^*cc^*$
definiert.
\begin{enumerate}
\item Geben Sie eine rechtslineare Grammatik $\mathcal{G}$ an, die $\mathcal{L}$
erzeugt!
\item Konstruieren Sie aus $\mathcal{G}$ einen endlichen Akzeptor, der $\mathcal{L}$
akzeptiert!
\end{enumerate}

\newpage

\subsection*{Lösung zu Aufgabe 1}
\begin{enumerate}
\item $(a \cdot a^*) + (b \cdot b^*)  = (a^+) + (b^+)$
\item Es handelt sich um einen (endlichen) Akzeptor.
\item So könnte ein gesuchter nichtdeterministischer endlicher Automat aussehen:
\begin{center}
\begin{tikzpicture}[line width=1pt]
\start{0}{0}{q_0}
\totr{0}{0}{1}{1}{\varepsilon}
\tobr{0}{0}{1}{-1}{\varepsilon}
\state{1}{1}{q_1}
\rloopt{1}{1}{a}
\draw (3,1) node[above right]{$a$};
\draw [->] (2.283,1.717) -- (3.717,0.283);
\state{1}{-1}{q_2}
\draw (3,-1) node[below right]{$b$};
\draw [->] (2.283,-1.717) -- (3.717,-0.283);
\rloopb{1}{-1}{b}
\final{2}{0}{q_3}
\end{tikzpicture}
\end{center}
\item Grammatik: $\mathcal{G} = (\mathcal{T},\mathcal{V},S,\mathcal{P})$ mit\\
$\mathcal{V} := \{S,B,C\}$, $\mathcal{T} := \{a,b\}$,
$\mathcal{P} := \{S \rightarrow Ca \; | \; Bb \; | \; a \; | \; b,
B \rightarrow Bb \; | \; b, C \rightarrow Ca \; | \; a\}$
\end{enumerate}

\subsection*{Lösung zu Aufgabe 2}
\begin{enumerate}
\item $(0+1)^* \cdot 0 \cdot (0+1)$
\item Die Sprache ist vom Chomsky-Typ 3!\\
Grammatik: $\mathcal{G}=(\mathcal{T},\mathcal{V},S,\mathcal{P})$ mit\\
$\mathcal{V} := \{S,B\}$, $\mathcal{T} := \{0,1\}$,
$\mathcal{P} := \{S \rightarrow 0S \; | \; 1S \; | \; 0B, B \rightarrow 0 \; | \; 1\}$
\item So könnte der gesuchte endliche Automat aussehen:\\
$\mathcal{M} = (\mathcal{Q},\Sigma,\delta,S,\mathcal{F})$ mit
$\Sigma = \{0,1\}$, $\mathcal{Q} = \{S,B,C\}$, $\mathcal{F} = \{C\}$
und $\delta$ gegeben durch:
\begin{center}
\begin{tikzpicture}[line width=1pt]
\start{0}{0}{S}
\rloopt{0}{0}{0,1}
\tor{0}{0}{1}{0}
\state{1}{0}{B}
\tor{1}{0}{2}{0,1}
\final{2}{0}{C}
\draw [->] (4,0.4) arc (0:180:1cm);
\draw (3,1.4) node[above]{$0$};
\draw [->] (4,-0.4) arc (0:-180:2cm);
\draw (2,-2.4) node[above]{$1$};
\end{tikzpicture}
\end{center}
\end{enumerate}
\newpage

\subsection*{Lösung zu Aufgabe 3}
\begin{enumerate}
\item Der Akzeptor $\mathcal{M}$ akzeptiert die Sprache $\mathcal{L}(\mathcal{M}) =
\{a,b\}^+\{c\}^+\{d\}(\{d\}\{a,b\}^*\{c\}^+\{d\})^*$.\\[10pt]
Da sich die Sprache $\mathcal{L}(\mathcal{M})$ leichter und leserlicher mit einem
regulären Ausdruck beschreiben lässt, folgt hier noch zusätzlich der reguläre
Ausdruck $R$, der $\mathcal{L}(\mathcal{M})$ beschreibt:\\
$R = (a+b)(a+b)^*cc^*d(d(a+b)^*cc^*d)^*$

\item Aus dem endlichen Akzeptor $\mathcal{M} = (\mathcal{Q},\Sigma,\delta,q_0,
\mathcal{F})$ kann direkt eine rechtslineare Grammatik konstruiert werden, die
$\mathcal{L}(\mathcal{M})$ erzeugt:\\
\underline{Schritt 1:} Definiere  das Terminalalphabet der Grammatik als das
Eingabealphabet des Automaten, also $\mathcal{T} := \Sigma = \{a,b,c,d\}$.\\
\underline{Schritt 2:} Füge für jeden Zustand $q$ des Akzeptors dem Variablenalphabet
der Grammatik eine Variable hinzu: $\mathcal{V} := \{S,A,B,C\}$, wobei das
Startzeichen $S$ dem Startzustand $q_0$, $A$ dem Zustand $q_1$, $B$ dem Zustand $q_2$
und $C$ dem Zustand $q_3$ entspricht.\\
\underline{Schritt 3:} Übersetze die Transitionen des Akzeptors in Produktionen der
Grammatik. Füge dazu für jede Transition $\delta(q_l,x)=q_m$ des Akzeptors mit 
$q_l,q_m \in Q, x \in \Sigma$ eine Produktion $V_1 \rightarrow xV_2$ hinzu,
wobei $V_1$ und $V_2$ die Variablen sind, die den Zuständen $q_l$ und $q_m$
entsprechen. Füge außerdem für jeden Endzustand des Akzeptors eine Produktion
$V \rightarrow \lambda$ für die dem Endzustand entsprechenden Variable $V \in
\mathcal{V}$ hinzu.
Die Sprache $\mathcal{L}(\mathcal{M})$ wird damit also erzeugt von der Grammatik
$\mathcal{G} = (\mathcal{T},\mathcal{V},S,\mathcal{P})$ mit dem Terminalalphabet
$\mathcal{T} = \{a,b,c,d\}$, dem Variablenalphabet $\mathcal{V} = \{S,A,B,C\}$,
dem Startzeichen $S$ und der Produktionenmenge\\
$\mathcal{P} = \{S \rightarrow aA \; | \; bA\\
A \rightarrow aA \; | \; bA \; | \; cB\\
B \rightarrow cB \; | \; dC\\
C \rightarrow dA \; | \; \lambda\}.$
\end{enumerate}

\subsection*{Lösung zu Aufgabe 4}
\begin{enumerate}
\item $\mathcal{L}$ wird von folgender Grammatik $\mathcal{G} = (\mathcal{T},\mathcal{V},
S,\mathcal{P})$ mit Terminalalphabet $\mathcal{T} = \{a,b,c\}$ und
Variablenalphabet $\mathcal{V} = \{S,A,B\}$ und folgenden Produktionen erzeugt:\\
$S \rightarrow aA \; | \; cB$\\
$A \rightarrow aA \; | \; bA \; | \; cB$\\
$B \rightarrow cB \; | \; \lambda$
\item Aus der rechtslinearen Grammatik $\mathcal{G}$ lässt sich folgendermaßen ein
endlicher Akzeptor $\mathcal{M}$ konstruieren, der $\mathcal{L}$ akzeptiert, wobei
das Eingabealphabet des Akzeptors das Terminalalphabet der Grammatik ist:\\
\underline{Schritt 1}: Für jede Variable $V \in \mathcal{V}$ bekommt der Akzeptor
einen Zustand $q_V$, dabei entspricht der Startzustand $q_S$ dem Startzeichen $S$.\\
\underline{Schritt 2}: Für jede Produktion der Form $V_1 \rightarrow xV_2$ mit
$V_1,V_2 \in \mathcal{V}$ und $x \in \mathcal{T}$ bekommt der Akzeptor eine
Transition von $q_{V_1}$ nach $q_{V_2}$ mit Eingabezeichen x.\\
\underline{Schritt 3}: Enthält die Grammatik für eine Variable $V \in \mathcal{V}$
eine Produktion $V \rightarrow \lambda$, so wird der entsprechende Zustand $q_V$
zum Endzustand.\\
Damit erhalten wir folgenden Automaten:
\begin{center}
\begin{tikzpicture}[line width=1pt]
\start{0}{0}{q_S}
\totr{0}{0}{1}{1}{a}
\tor{0}{0}{2}{c}
\state{1}{1}{q_A}
\rloopt{1}{1}{a,b}
\draw (3,1) node[above right]{$c$};
\draw [->] (2.283,1.717) -- (3.717,0.283);
\final{2}{0}{q_B}
\rloopr{2}{0}{c}
\end{tikzpicture}
\end{center}
\end{enumerate}

\end{document}