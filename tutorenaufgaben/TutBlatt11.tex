\documentclass[10pt,oneside,onecolumn,a4paper,german,titlepage]{article}
\usepackage[utf8]{inputenc}
\usepackage[german]{babel}
\usepackage{german}
\usepackage{amsmath,amsfonts,amssymb,latexsym,textcomp,stmaryrd}
\usepackage{tikz}
\pagestyle{plain}
\pagenumbering{arabic}
\parskip0.5ex plus0.1ex minus0.1ex
\parindent0pt
\voffset-1.04cm
\topmargin0pt
\headheight0pt
\headsep0pt
\topskip0pt
\textheight26.7cm
\footskip20pt
\hoffset-1.04cm
\oddsidemargin0pt
\evensidemargin0pt
\textwidth18cm
\marginparsep0pt
\marginparwidth0pt
\begin{document}

\section*{Tutorien-"Ubungsblatt 11}

\subsection*{Aufgabe 1}
\begin{enumerate}
\item Wie gro"s sind der Informationsgehalt und die Entropie, wenn eine Quelle mit
dem Alphabet $\{0,1\}$ nur aus dem Zeichen $0$ bestehende Folgen sendet?
\item An einer Quelle mit $n$ Zeichen tritt jedes Zeichen gleichverteilt auf. Wie
gro"s sind der Informationsgehalt und die Entropie eines einzelnen Zeichens?
\item Berechnen Sie die Entropie des Wurfes eines idealen W"urfels mit 8 Seiten,
dessen Wahrscheinlichkeit f"ur jede Seite $p = \frac{1}{8}$ ist!
\item Was ist der Unterschied zwischen den beiden Folgen, die aus verschiedenen
ged"achtnislosen Quellen mit der gleichen Wahrscheinlichkeit f"ur $0$ und $1$
gesendet werden, wenn man sie unter dem Aspekt Entropie und Ordnung betrachtet?
\begin{enumerate}
\item ...10101010101010101010...
\item ...01101100110111000010...
\end{enumerate}
\end{enumerate}

\subsection*{Aufgabe 2}
Studieren Sie den Fall eines asymmetrischen bin"aren Kanals mit Quelle $X$ und
Empf"anger $Y$. Die "Ubertragungswahrscheinlichkeiten $P(Y|X)$ seien durch das
folgende Diagramm gegeben:
\begin{center}
\begin{tikzpicture}
\draw (0,0) circle (10pt);
\draw (0,0) node {$1$};
\draw (0,2) node {$X$};
\draw (0,4) circle (10pt);
\draw (0,4) node {$0$};
\draw (4,0) circle (10pt);
\draw (4,0) node {$1$};
\draw (4,2) node {$Y$};
\draw (4,4) circle (10pt);
\draw (4,4) node {$0$};
\draw [->] (0.35,0) -- (3.65,0);
\draw (2,0.3) node {$0.9$};
\draw [->] (0.35,4) -- (3.65,4);
\draw (2,3.7) node {$0.8$};
\draw [->] (0.25,0.25) -- (3.75,3.75);
\draw (1.5,1) node {$0.1$};
\draw [->] (0.25,3.75) -- (3.75,0.25);
\draw (1.5,3) node {$0.2$};
\end{tikzpicture}
\end{center}
\begin{enumerate}
\item Wie gro"s ist die Wahrscheinlichkeit daf"ur, dass die Bitkette ``$1100$'' als
``$1001$'' "ubertragen wird?
\item Wenn die Entropie der Quelle $H(X) = 1 \; \mbox{bit}$ ist, wie gro"s ist dann
$H(Y)$?
\item Wie gro"s muss $H(X)$ sein, damit $H(Y) = 1 \; \mbox{bit}$ gilt?
\item Wie gro"s ist die Verbundentropie $H(X,Y)$ des "Ubertragungssystems? Gehen Sie
ab dieser Teilaufgabe von der Situation der 2. Teilaufgabe aus!
\item Wie gro"s ist die sog. Irrelevanz $H(Y|X)$? Und wie gro"s ist die sog.
"Aquivokation $H(X|Y)$?
\item Wie gro"s ist schlie"slich die Transinformation $I(X;Y)$?
\end{enumerate}

\newpage

\subsection*{Aufgabe 3}
Gegeben sei eine Quelle mit Alphabet $\{A,B,C,D\}$ und mit den folgenden
Wahrscheinlichkeiten:\\
$P(A)=\frac{1}{2}, P(B)=\frac{1}{4}, P(C)=\frac{1}{8}, P(D)=\frac{1}{8}$
\begin{enumerate}
\item Berechnen Sie die Entropie der Quelle!
\item Erstellen Sie eine entsprechende Huffman-Codierung!
\item Was ist die mittlere Codewortl"ange? Gibt es einen Zusammenhang zur Entropie?
\item Gegeben sei der folgende Huffman-Baum:
\begin{center}
\begin{tikzpicture}
\node[circle,draw]{}
child{
  node[circle,draw]{}
  child{
    node[circle,draw]{$A$}
    edge from parent
    node[left]{\textbf{$0$}}
  }
  child{
    node[circle,draw]{$B$}
    edge from parent
    node[right]{\textbf{$1$}}
  }
  edge from parent
  node[left]{\textbf{$0$}}
}
child{
  node[circle,draw]{$C$}
  edge from parent
  node[right]{\textbf{$1$}}
};
\end{tikzpicture}
\end{center}
Dekodieren Sie $011011101100101011$! Ist der Huffman-Code geeignet?
\end{enumerate}

\subsection*{Aufgabe 4}
Gegeben sei eine ged"achtnislose Quelle $Q$, die mit Wahrscheinlichkeit $p_0 =
\frac{1}{4}$ eine $0$ und mit Wahrscheinlichkeit $p_1 = \frac{3}{4}$ eine $1$ sendet.
Gegeben sei zudem ein Empf"anger $R$, der die Zeichen von $Q$ zu empfangen versucht.
Dieser Empf"anger empf"angt eine $0$ immer richtig. Sendet die Quelle $Q$ jedoch
eine $1$, so empf"angt $R$ mit Wahrscheinlichkeit $\frac{1}{2}$ eine $1$ und mit
Wahrscheinlichkeit $\frac{1}{2}$ eine $0$.
\begin{enumerate}
\item Berechnen Sie die Information $I(0)$ und $I(1)$ bez"uglich der Quelle $Q$!
\item Berechnen Sie die Entropie der Quelle $Q$!
\item Die Quelle $Q$ sendet die Zeichenfolge $0110$. Wie hoch ist der
Informationsgehalt dieser Zeichenfolge?
\item Berechnen Sie die Totalinformation $H(Q,R)$, die Fehlinformation $H(R|Q)$,
die "Aquivokation $H(Q|R)$ und die Transinformation $I(Q;R)$!
\end{enumerate}

\newpage

\subsection*{L"osung zu Aufgabe 1}
\begin{enumerate}
\item Sei $p$ die Wahrscheinlichkeitsfunktion, f"ur die nach der Aufgabenstellung
$p(X=0) = 1$ und $p(X=1) = 0$ gilt. Der Informationsgehalt $I(x)$ eines von der
Quelle gesendeten Zeichens $x$ ist $I(x) = -\log_2{p(X=x)}$.
Somit ist $I(0) = -\log_2{1} = 0$.\\
Die Entropie ist $H(X) = \sum\limits_{x \in \{0,1\}}(p(X=x) \cdot I(x)) =
1 \cdot 0 + 0 = 0$.
\item Für alle $n$ Zeichen der Quelle gilt $p(X=x) = \frac{1}{n}$. Somit gilt $I(x)=
-\log_2{p(X=x)} = -\log_2{\frac{1}{n}} = \log_2{n}$ f"ur jedes Zeichen $x$.\\
Die Entropie eines Zeichens ist dann $H(X) = \sum\limits_{i=1}^n(\frac{1}{n} \cdot
\log_2{n}) = \log_2{n}$.
\item F"ur jede der 8 W"urfelseiten gilt $p(X=x) = \frac{1}{8}$ und somit ist die
Entropie eines einzelnen W"urfelwurfs $H(X) = \log_2{8} = 3$.
\item Der Informationsgehalt \underline{eines} Zeichens ist f"ur die gegebenen Folgen
gleich, da die Zeichen $0$ und $1$ von beiden Quellen mit der gleichen
Wahrscheinlichkeit gesendet werden. Zu erkennen ist, dass die Zeichen der ersten
Quelle durch eine sich wiederholende Struktur geordnet sind, weshalb die erste
Zeichenkette im Mittel weit weniger Information als in der zweiten Zeichenkette
enth"alt und damit eine niedrigere Entropie hat.\\
Wenn die Folge aus voneinander abh"angigen Folgengliedern besteht, dann sind die
n"achsten Folgenglieder leichter vorherzusagen und damit ist ihr mittlerer
Informationsgehalt geringer. Man sieht dies an der ersten Zeichenkette besonders
einfach, wenn man immer Paare der Zeichen $0$ und $1$ jeweils als ein Zeichen eines
neuen, vierelementigen Alphabets betrachtet. Dann sendet die erste Quelle immer das
gleiche Zeichen $10$ und der Informationsgehalt dieses Zeichens und die Entropie
sind $0$.
\end{enumerate}

\subsection*{L"osung zu Aufgabe 2}
\begin{enumerate}
\item Da der Informationskanal kein Ged"achtnis hat, geschieht die "Ubertragung der
einzelnen Zeichen unabh"angig voneinander. Die Wahrscheinlichkeiten multiplizieren
sich also:\\
$p(1001|1100) = p(1|1) \cdot p(0|1) \cdot p(0|0) \cdot p(1|0) = 0.9 \cdot 0.1 \cdot
0.8 \cdot 0.2 = 0.0144$
\item Die Entropie der Quelle erreicht wegen des zweielementigen Alphabets genau dann
ihr Maximum von $H(X) = 1 \; \mbox{bit}$, wenn die beiden Zeichen $0$ und $1$ gleich
oft auftreten, also wenn $P(X=0) = P(X=1) = 0.5$ gilt.\\
Dann gilt f"ur den Empf"anger $Y$ $P(Y=y) = \sum\limits_{x \in \{0,1\}}(P(X=x) \cdot
P(Y=y|X=x))$ und wir erhalten wir $P(Y=0) = 0.5 \cdot 0.8 + 0,5 \cdot 0.1 = 0.45$ und
$P(Y=1) = 0.5 \cdot 0.9 + 0.5 \cdot 0.9 = 0.55$.\\
Die Entropie auf der Empf"angerseite ist also $H(Y) = -(0.45 \cdot \log_2{0.45} +
0.55 \cdot \log_2{0.55}) \; \mbox{bit} \approx 0,9928 \; \mbox{bit}$
\item Es gilt wieder $H(Y) = 1 \; \mbox{bit} \Leftrightarrow P(Y = 0) = P(Y = 1) =
0.5$.\\
Daraus ergeben sich nun die beiden folgenden linearen Gleichungen:\\
$P(Y=0) = 0.5 = 0.8 \cdot P(X=0) + 0.1 \cdot P(X=1)$\\
$P(Y=1) = 0.5 = 0.2 \cdot P(X=0) + 0.9 \cdot P(X=1)$\\
Man erh"alt dann $P(X=0) = \frac{4}{7} \approx 0.571$ und $P(X=1) = \frac{3}{7}
\approx 0.429$.\\
Damit ist die Entropie der Quelle $H(X) \approx 0,9852 \; \mbox{bit}$.
\item Die Verbundentropie des Kanals berechnet sich durch\\
$H(X,Y) = -\sum\limits_{x \in \{0,1\}}\sum\limits_{y \in \{0,1\}}(P(X=x,Y=y) \cdot
\log_2{P(X=x,Y=y)}) \approx 1,5955 \; \mbox{bit}$.
\item Aus der Verbundentropie lassen sich leicht die bedingten Entropien, n"amlich
die Irrelevanz $H(Y|X)$ und die "Aquivokation $H(X|Y)$, berechnen:\\
$H(Y|X) = H(X,Y) - H(X) \approx 1,5955 \; \mbox{bit} \; -1 \; \mbox{bit}
\approx 0,5955 \; \mbox{bit}$\\
$H(X|Y) = H(X,Y) - H(Y) \approx 1,5955 \; \mbox{bit} \; -0,9928 \; \mbox{bit}
\approx 0,6027 \; \mbox{bit}$
\item Die Transinformation $I(X;Y)$ bezeichnet denjenigen Anteil der Quellenentropie
$H(X)$, der bei der "Ubertragung beim Empf"anger $Y$ ankommt, also genau denjenigen
Anteil der empfangenen Information $H(Y)$, der "uberhaupt von der Quelle $X$ stammt.
Sie l"asst sich daher berechnen durch:\\
$I(X;Y) = H(X) - H(X|Y) = H(Y) - H(Y|X) \approx 0,3973 \; \mbox{bit}$
\end{enumerate}

\subsection*{L"osung zu Aufgabe 3}
\begin{enumerate}
\item $H(X) = -(0.5 \cdot \log_2{0.5} + 0.25 \cdot \log_2{0.25} + 2 \cdot 0.125 \cdot
\log_2{0.125}) = 0.5 + 0.5 + 0.75 = 1.75$

\newpage

\item \underline{Hinweis:} Der Code ist nicht eindeutig!
\begin{center}
\begin{tikzpicture}
\node[circle,draw]{}
child{
  node[circle,draw]{$A$}
  edge from parent
  node[left]{\textbf{$0$}}
}
child{
  node[circle,draw]{}
  child{
    node[circle,draw]{$B$}
    edge from parent
    node[left]{\textbf{$0$}}
  }
  child{
    node[circle,draw]{}
    child{
      node[circle,draw]{$C$}
      edge from parent
      node[left]{\textbf{$0$}}
    }
    child{
      node[circle,draw]{$D$}
      edge from parent
      node[right]{\textbf{$1$}}
    }
    edge from parent
    node[right]{\textbf{$1$}}
  }
  edge from parent
  node[right]{\textbf{$1$}}
};
\end{tikzpicture}
\end{center}
\item Die mittlere Codewortl"ange ergibt sich aus der Aufsummierung der Produkte aus
Auftrittswahrscheinlichkeit und Codewortl"ange "uber allen Codew"ortern. Also ist die
mittlere Codewortl"ange $0.5 \cdot 1 + 0.25 \cdot 2 + 2 * 0.125 \cdot 3 = 1.75$. Die
mittlere Codewortl"ange ist immer gr"o"ser oder gleich der Entropie. Hier erreicht
sie sogar den minimalen Wert, weil die Auftrittswahrscheinlichkeiten Potenzen von $2$
sind.
\item Dekodiert ergibt sich das Wort BCBCCBCACBBC. Der Huffman-Code eignet sich gut
wegen der Beachtung der Auftrittswahrscheinlichkeiten.
\end{enumerate}

\subsection*{L"osung zu Aufgabe 4}
\begin{enumerate}
\item Information des Zeichens $0$: $I(0) = -\log_2{p_0} = -\log_2{0.25} =
2 \; \mbox{bit}$\\
Information des Zeichens $1$: $I(1) = -\log_2{p_1} = -\log_2{0.75} \approx
0.415 \; \mbox{bit}$
\item Entropie der Quelle $Q$: $H(Q) = -(p_0 \cdot \log_2{p_0} + p_1 \cdot
\log_2{p_1}) = 0.25 \cdot 2 + 0.75 \cdot \log_2{\frac{4}{3}} \approx
0.81 \; \mbox{bit}$
\item Informationsgehalt der Zeichenfolge $0110$: $I(0110) = 2 \cdot I(0) + 2 \cdot
I(1) \approx 4.83 \; \mbox{bit}$
\item Wir fassen die bedingten Wahrscheinlichkeiten in folgender Skizze zusammen:
\begin{center}
\begin{tikzpicture}
\draw (0,0) circle (10pt);
\draw (0,0) node {$1$};
\draw (0,2) node {$Q$};
\draw (0,4) circle (10pt);
\draw (0,4) node {$0$};
\draw (4,0) circle (10pt);
\draw (4,0) node {$1$};
\draw (4,2) node {$R$};
\draw (4,4) circle (10pt);
\draw (4,4) node {$0$};
\draw [->] (0.35,0) -- (3.65,0);
\draw (2,0.3) node {$P(1|1)=0.5$};
\draw [->] (0.35,4) -- (3.65,4);
\draw (2,3.7) node {$P(0|0)=1$};
\draw [->] (0.25,0.25) -- (3.75,3.75);
\draw (1.3,1) node[above left] {$P(0|1)=0.5$};
\draw [->] (0.25,3.75) -- (3.75,0.25);
\draw (1.1,3) node[below left] {$P(1|0)=0$};
\end{tikzpicture}
\end{center}
Weiterhin ben"otigen wir folgende Wahrscheinlichkeiten:\\
$P(0,0) = p_0 \cdot P(0|0) = 0.25$\\
$P(0,1) = p_0 \cdot P(1|0) = 0$\\
$P(1,0) = p_1 \cdot P(0|1) = 0.375$\\
$P(1,1) = p_1 \cdot P(1|1) = 0.375$\\[4pt]
Totalinformation: $H(Q,R) = -\sum\limits_{x,y \in \{0,1\}}(P(x,y) \cdot
\log_2{P(x,y)}) \approx 1,56 \; \mbox{bit}$\\
Fehlinformation: $H(R|Q) = -\sum\limits_{x,y \in \{0,1\}}(P(x,y) \cdot
\log_2{P(y|x)}) = -(2 \cdot 0.375 \cdot \log_2{0.5}) = 0,75 \; \mbox{bit}$\\
Alternativ: $H(R|Q) = H(Q,R) - H(Q) \approx 0,75 \; \mbox{bit}$
Zur Berechnung der "Aquivokation berechnen wir erst die Empfangswahrscheinlichkeiten
der Zeichen $0$ und $1$ und die Entropie des Empf"angers $R$.\\
$p_0' = p_0 \cdot P(0|0) + p_1 \cdot P(0|1) = 0.25 \cdot 1 + 0.75 \cdot 0.5 =
0.625$\\
$p_1' = p_0 \cdot P(1|0) + p_1 \cdot P(1|1) = 0.25 \cdot 0 + 0.75 \cdot 0.5 =
0.375$\\
$H(R) = -(p_0' \cdot \log_2{p_0'} + p_1' \cdot \log_2{p_1'}) \approx
0.95 \; \mbox{bit}$\\
"Aquivokation: $H(Q|R) = H(Q,R) - H(R) \approx 0.61 \; \mbox{bit}$\\
Transinformation: $I(Q;R) = H(Q) - H(Q|R) \approx 0.20 \; \mbox{bit}$\\
Alternativ: $I(Q;R) = H(R) - H(R|Q) \approx 0.20 \; \mbox{bit}$
\end{enumerate}

\end{document}