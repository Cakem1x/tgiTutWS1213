\documentclass[10pt,oneside,onecolumn,a4paper,german,titlepage]{article}
\usepackage[utf8]{inputenc}
\usepackage[german]{babel}
\usepackage{german}
\usepackage{amsmath,amsfonts,amssymb,latexsym,textcomp,stmaryrd}
\usepackage{tikz}
\pagestyle{plain}
\pagenumbering{arabic}
\parskip0.5ex plus0.1ex minus0.1ex
\parindent0pt
\voffset-1.04cm
\topmargin0pt
\headheight0pt
\headsep0pt
\topskip0pt
\textheight26.7cm
\footskip20pt
\hoffset-1.04cm
\oddsidemargin0pt
\evensidemargin0pt
\textwidth18cm
\marginparsep0pt
\marginparwidth0pt
\begin{document}

\section*{Tutorien-"Ubungsblatt 12}

\subsection*{Aufgabe 1}
Gegeben sei ein bin"arer Kanal mit Sender $X$ und Empf"anger $Y$, genannt Z-Kanal,
durch die folgende Matrix:\\[4pt]
$Q = \left( \begin{array}{cc}
P(Y=0|X=0) & P(Y=0|X=1) \\ P(Y=1|X=0) & P(Y=1|X=1)
\end{array} \right) = \left( \begin{array}{cc}
1 & 0.5 \\ 0 & 0.5
\end{array} \right)$\\[4pt]
Bestimmen Sie die Kanalkapazit"at!\\[4pt]
\underline{Hinweis:} Sie k"onnen dabei folgendes verwenden:
\[ \frac{\log_b{x}}{dx} = \frac{1}{x \cdot \ln{b}} \]

\subsection*{Aufgabe 2}
Sei $\mathcal{C}$ ein bin"arer Code, der durch die folgende Generatormatrix gegeben
ist:\\[4pt]
$G = \left( \begin{array}{cccc}
1 & 0 & 0 & 0 \\
0 & 1 & 0 & 0 \\
0 & 0 & 1 & 0 \\
0 & 0 & 0 & 1 \\
1 & 1 & 0 & 0 \\
0 & 0 & 1 & 1 \\
1 & 1 & 1 & 1
\end{array} \right)$\\[4pt]
Dekodieren Sie die folgenden empfangenen W"orter!
\begin{enumerate}
\item $w_1 = (\begin{array}{ccccccc}1 & 1 & 0 & 1 & 0 & 1 & 1\end{array})$
\item $w_2 = (\begin{array}{ccccccc}0 & 1 & 1 & 0 & 1 & 1 & 1\end{array})$
\item $w_3 = (\begin{array}{ccccccc}0 & 1 & 1 & 1 & 0 & 0 & 0\end{array})$
\end{enumerate}

\subsection*{Aufgabe 3}
Gegeben sei der $[7,4]$-Hamming-Code $\mathcal{C}_H$ mit der Erzeugermatrix\\[4pt]
$G = \left( \begin{array}{cccc}
1 & 1 & 0 & 1 \\
1 & 0 & 1 & 1 \\
1 & 0 & 0 & 0 \\
0 & 1 & 1 & 1 \\
0 & 1 & 0 & 0 \\
0 & 0 & 1 & 0 \\
0 & 0 & 0 & 1
\end{array} \right)$\\[4pt]
und der Pr"ufmatrix\\[4pt]
$H = \left( \begin{array}{ccccccc}
0 & 0 & 0 & 1 & 1 & 1 & 1 \\
0 & 1 & 1 & 0 & 0 & 1 & 1 \\
1 & 0 & 1 & 0 & 1 & 0 & 1
\end{array} \right)$.\\[4pt]
Dekodieren Sie die folgenden empfangenen W"orter!
\begin{enumerate}
\item $w_1 = (\begin{array}{ccccccc}0 & 0 & 0 & 1 & 1 & 1 & 1\end{array})$
\item $w_2 = (\begin{array}{ccccccc}1 & 1 & 0 & 0 & 1 & 1 & 1\end{array})$
\item $w_3 = (\begin{array}{ccccccc}1 & 1 & 1 & 0 & 0 & 0 & 0\end{array})$
\item $w_4 = (\begin{array}{ccccccc}0 & 1 & 1 & 1 & 1 & 1 & 1\end{array})$
\end{enumerate}

\newpage

\subsection*{L"osung zu Aufgabe 1}
Die Kanalkapazit"at wird berechnet durch: $C = \max\limits_{X} I(X;Y) = 
\max\limits_{p(x)}I(X;Y)$\\[4pt]
Weil es sich um einen bin"aren Kanal handelt, gilt $P(X=0) = p$ und $P(X=1) = 1 - p$
f"ur eine feste Wahrscheinlichkeit $p \in \mathbb{R}, 0 \leq p \leq 1$ bez"uglich
einer entsprechenden Quelle $X$.\\[4pt]
Dann gilt:\\
1. $P(Y=0) = P(X=0) \cdot P(Y=0|X=0) + P(X=1) \cdot P(Y=0|X=1) = p \cdot 1 +
(1 - p) \cdot 0.5 = 0.5 \cdot (1 + p)$\\[4pt]
2. $P(Y=1) = P(X=0) \cdot P(Y=1|X=0) + P(X=1) \cdot P(Y=1|X=1) = p \cdot 0 +
(1 - p) \cdot 0.5 = 0.5 \cdot (1 - p)$\\[4pt]
3. $H(Y) = -(P(Y=0) \cdot \log_2{P(Y=0)} + P(Y=1) \cdot \log_2{P(Y=1)}) =$\\
$(-0.5) \cdot (1 + p) \cdot \log_2{(0.5 \cdot (1 + p))} +
 (-0.5) \cdot (1 - p) \cdot \log_2{(0.5 \cdot (1 - p))}$\\[4pt]
4. $H(Y|X=0) = -(P(Y=0|X=0) \cdot \log_2{P(Y=0|X=0)} + P(Y=1|X=0) \cdot
\log_2{P(Y=1|X=0)}) =$\\
$-(1 \cdot \log_2{1} + 0 \cdot \log_2{0}) = 0$\\[4pt]
5. $H(Y|X=1) = -(P(Y=0|X=1) \cdot \log_2{P(Y=0|X=1)} + P(Y=1|X=1) \cdot
\log_2{P(Y=1|X=1)}) =$\\
$-(0.5 \cdot \log_2{0.5} + 0.5 \cdot \log_2{0.5}) = 1$\\[4pt]
6. $H(Y|X) = P(X=0) \cdot H(Y|X=0) + P(X=1) \cdot H(Y|X=1) = p \cdot 0 +
(1 - p) \cdot 1 = 1 - p$\\[4pt]
7. $I(X;Y) = H(Y) - H(Y|X) = (-0.5) \cdot (1 + p) \cdot \log_2{(0.5 \cdot (1 + p))} +
(-0.5) \cdot (1 - p) \cdot \log_2{(0.5 \cdot (1 - p))} - 1 + p$\\[4pt]
Damit gilt nun $C = \max\limits_{p} I(X;Y)$ und wir k"onnen "uber die Ableitung von
$I(X;Y)$ nach $p$ den Wert f"ur $p$ finden, f"ur den $I(X;Y)$ maximal wird.\\[4pt]
$I'(X;Y) = \frac{dI(X;Y)}{dp} =$\\[4pt]
$(-0.5) \cdot \log_2{(0.5 \cdot (1 + p))} +
(-0.5) \cdot (1 + p) \cdot \frac{1}{0.5 \cdot (1 + p) \cdot \ln{2}} \cdot 0.5 +$
\\[4pt]
$0.5 \cdot \log_2{(0.5 \cdot (1 - p))} +
(-0.5) \cdot (1 - p) \cdot \frac{1}{0.5 \cdot (1 - p) \cdot \ln{2}} \cdot (-0.5) + 1
=$\\[4pt]
$0.5 \cdot (\log_2{(0.5 \cdot (1 - p))} - \log_2{(0.5 \cdot (1 + p))}) +
(-0.5) \cdot (\frac{1}{\ln{2}} - \frac{1}{\ln{2}}) + 1 =$\\[4pt]
$0.5 \cdot \log_2{(\frac{1 - p}{1 + p})} + 1$\\[4pt]
$I'(X;Y) = 0 \Rightarrow 0.5 \cdot \log_2{(\frac{1 - p}{1 + p})} + 1 = 0
\Rightarrow \log_2{(\frac{1 - p}{1 + p})} = -2
\Rightarrow \frac{1 - p}{1 + p} = 0.25
\Rightarrow 1 - p = 0.25 + 0.25 \cdot p \Rightarrow p = 0.6$\\[4pt]
Durch Einsetzen von $p = 0.6$ in $I(X;Y)$ ergibt sich dann:\\
$C = (-0.8) \cdot \log_2{0.8} + (-0.2) \cdot \log_2{0.2} - 0.4 \approx 0.322 \;
\mbox{bit}$

\subsection*{L"osung zu Aufgabe 2}
Aufgrund der Tatsache, dass es sich um einen systematischen Code handelt, wie man
an der Teilmatrix $I_4$ der Generatormatrix $G$ erkennen kann, kann man die drei
Pr"ufgleichungen direkt aus $G$ ablesen. Sie lauten f"ur ein Codewort
$c \in \mathcal{C} \subseteq \mathbb{F}_2^7$:\\[4pt]
1. $c_1 + c_2 + c_5 = s_1$\\
2. $c_3 + c_4 + c_6 = s_2$\\
3. $c_1 + c_2 + c_3 + c_4 + c_7 = s_3$\\[4pt]
Daraus kann nun die Kontrollmatrix bilden:\\[4pt]
$H = \left( \begin{array}{ccccccc}
1 & 1 & 0 & 0 & 1 & 0 & 0 \\
0 & 0 & 1 & 1 & 0 & 1 & 0 \\
1 & 1 & 1 & 1 & 0 & 0 & 1
\end{array} \right)$\\[4pt]
Nun kann man f"ur ein empfangenes Wort $w$ "uber die Gleichung $Hw^T = s^T$ das
Syndrom $s =$ ($s_1$ $s_2$ $s_3$) bestimmen. Gilt $s =$ (0 0 0), so geht man von
einer korrekten "Ubertragung aus. Gilt $s \not=$ (0 0 0), so geht man zun"achst von
einem 1-Bit-Fehler aus und vergleicht $s^T$ mit den Spalten der Kontrollmatrix. Bei
einer eindeutigen "Ubereinstimmung kann das fehlerhafte Bit korrigiert werden. Bei
mehreren "Ubereinstimmungen ist dann eine Korrektur nicht mehr m"oglich. Bei keiner
"Ubereinstimmung liegt dann ein Mehr-Bit-Fehler vor und man muss dann Spaltensummen
f"ur den Vergleich verwenden.
\begin{enumerate}
\item Das Syndrom lautet (0 0 0), das Codewort wurde also korrekt "ubertragen und
die Originalnachricht lautet\\
(1 1 0 1).
\item Das Syndrom lautet (0 0 1), das Codewort hat damit einen Fehler in Bit 7 und
die Originalnachricht lautet\\
(0 1 1 0).
\item Das Syndrom lautet (1 0 1), das Codewort hat damit einen Fehler in Bit 1 oder
in Bit 2. Bei einem Fehler\\
in Bit 1 lautet die Originalnachricht (1 1 1 1), bei einem Fehler in Bit 2 lautet
sie (0 0 1 1).
\end{enumerate}

\newpage

\subsection*{L"osung zu Aufgabe 3}
\begin{enumerate}
\item Das Syndrom lautet (0 0 0), das Codewort wurde also korrekt "ubertragen und
die Originalnachricht lautet\\
(0 1 1 1).
\item Das Syndrom lautet (1 1 1), das Codewort hat damit einen Fehler in Bit 7 und
die Originalnachricht lautet\\
(0 1 1 0).
\item Das Syndrom lautet (0 0 0), das Codewort wurde also korrekt "ubertragen und
die Originalnachricht lautet\\
(1 0 0 0).
\item Das Syndrom lautet (0 0 1), das Codewort hat damit einen Fehler in Bit 1 und
die Originalnachricht lautet\\
(1 1 1 1).
\end{enumerate}

\end{document}