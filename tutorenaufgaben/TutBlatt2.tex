\documentclass[10pt,oneside,onecolumn,a4paper,german,titlepage]{article}
\usepackage[utf8]{inputenc}
\usepackage[german]{babel}
\usepackage{german}
\usepackage{amsmath,amsfonts,amssymb,latexsym,textcomp}
\usepackage{tikz}
\pagestyle{plain}
\pagenumbering{arabic}
\parskip0.5ex plus0.1ex minus0.1ex
\parindent0pt
\voffset-1.04cm
\topmargin0pt
\headheight0pt
\headsep0pt
\topskip0pt
\textheight26.7cm
\footskip20pt
\hoffset-1.04cm
\oddsidemargin0pt
\evensidemargin0pt
\textwidth18cm
\marginparsep0pt
\marginparwidth0pt
\begin{document}

\newcommand{\start}[3]
{
  \draw (#1*2,#2*2) node{$#3$};
  \draw (#1*2,#2*2) circle(0.4cm);
  \draw [->] (#1*2-0.9,#2) -- (#1*2-0.4,#2);
}
\newcommand{\startfinal}[3]
{
  \draw (#1*2,#2*2) node{$#3$};
  \draw (#1*2,#2*2) circle(0.4cm);
  \draw (#1*2,#2*2) circle(0.32cm);
  \draw [->] (#1*2-0.9,#2) -- (#1*2-0.4,#2);
}
\newcommand{\state}[3]
{
  \draw (#1*2,#2*2) node{$#3$};
  \draw (#1*2,#2*2) circle(0.4cm);
}
\newcommand{\final}[3]
{
  \draw (#1*2,#2*2) node{$#3$};
  \draw (#1*2,#2*2) circle(0.4cm);
  \draw (#1*2,#2*2) circle(0.32cm);
}
\newcommand{\tol}[4]
{
  \draw (#1+#3,#2*2) node[above]{$#4$};
  \draw [->] (#1*2-0.4,#2*2) -- (#3*2+0.4,#2*2);
}
\newcommand{\tor}[4]
{
  \draw (#1+#3,#2*2) node[above]{$#4$};
  \draw [->] (#1*2+0.4,#2*2) -- (#3*2-0.4,#2*2);
}
\newcommand{\tot}[4]
{
  \draw (#1*2,#2+#3) node[right]{$#4$};
  \draw [->] (#1*2,#2*2+0.4) -- (#1*2,#3*2-0.4);
}
\newcommand{\tob}[4]
{
  \draw (#1*2,#2+#3) node[right]{$#4$};
  \draw [->] (#1*2,#2*2-0.4) -- (#1*2,#3*2+0.4);
}
\newcommand{\totl}[5]
{
  \draw (#1+#3,#2+#4) node[above right]{$#5$};
  \draw [->] (#1*2-0.283,#2*2+0.283) -- (#3*2+0.283,#4*2-0.283);
}
\newcommand{\totr}[5]
{
  \draw (#1+#3,#2+#4) node[above left]{$#5$};
  \draw [->] (#1*2+0.283,#2*2+0.283) -- (#3*2-0.283,#4*2-0.283);
}
\newcommand{\tobl}[5]
{
  \draw (#1+#3,#2+#4) node[below right]{$#5$};
  \draw [->] (#1*2-0.283,#2*2-0.283) -- (#3*2+0.283,#4*2+0.283);
}
\newcommand{\tobr}[5]
{
  \draw (#1+#3,#2+#4) node[below left]{$#5$};
  \draw [->] (#1*2+0.283,#2*2-0.283) -- (#3*2-0.283,#4*2+0.283);
}
\newcommand{\rloopl}[3]
{
  \draw (#1*2-1,#2*2) node[left]{$#3$};
  \draw [->] (#1*2-0.35,#2*2-0.2) arc (-30:-320:0.32cm);
}
\newcommand{\rloopr}[3]
{
  \draw (#1*2+1,#2*2) node[right]{$#3$};
  \draw [->] (#1*2+0.35,#2*2+0.2) arc (150:-140:0.32cm);
}
\newcommand{\rloopt}[3]
{
  \draw (#1*2,#2*2+1) node[above]{$#3$};
  \draw [->] (#1*2-0.2,#2*2+0.35) arc (240:-50:0.32cm);
}
\newcommand{\rloopb}[3]
{
  \draw (#1*2,#2*2-1) node[below]{$#3$};
  \draw [->] (#1*2+0.2,#2*2-0.35) arc (60:-230:0.32cm);
}
\newcommand{\lloopl}[3]
{
  \draw (#1*2-1,#2*2) node[left]{$#3$};
  \draw [->] (#1*2-0.35,#2*2+0.2) arc (30:320:0.32cm);
}
\newcommand{\lloopr}[3]
{
  \draw (#1*2+1,#2*2) node[right]{$#3$};
  \draw [->] (#1*2+0.35,#2*2-0.2) arc (-150:140:0.32cm);
}
\newcommand{\lloopt}[3]
{
  \draw (#1*2,#2*2+1) node[above]{$#3$};
  \draw [->] (#1*2+0.2,#2*2+0.35) arc (-60:230:0.32cm);
}
\newcommand{\lloopb}[3]
{
  \draw (#1*2,#2*2-1) node[below]{$#3$};
  \draw [->] (#1*2-0.2,#2*2-0.35) arc (-240:50:0.32cm);
}

\section*{Tutorien-Übungsblatt 2}

\subsection*{Aufgabe 1}
Gegeben sei der folgende endliche Automat:\\
$\mathcal{M} = (\mathcal{Q},\Sigma,\delta,s_0,\mathcal{F})$ mit
$\mathcal{Q} = \{s_0,s_1,s_2,s_3,s_4\}, \Sigma = \{0,1\}, \mathcal{F} = \{s_4\}$
und $\delta$ gegeben durch:
\begin{center}
\begin{tikzpicture}[line width=1pt]
\start{0}{0}{s_0}
\totr{0}{0}{1}{1}{0}
\tor{0}{0}{1}{1}
\state{1}{1}{s_1}
\tor{1}{1}{2}{0}
\tob{1}{1}{0}{1}
\state{1}{0}{s_2}
\tor{1}{0}{2}{0}
\rloopb{1}{0}{1}
\state{2}{0}{s_3}
\tot{2}{0}{1}{0}
\draw [->] (4,-0.4) arc (0:-180:2cm);
\draw (2,-2.4) node[below]{$1$};
\final{2}{1}{s_4}
\rloopr{2}{1}{0,1}
\end{tikzpicture}
\end{center}
\begin{enumerate}
\item Ist der gegebene endliche Automat deterministisch?
\item Zeichnen Sie den Äquivalenzklassenautomaten!
\item Geben Sie die Äquivalenzklassen der Zustände vom entstandenen Automaten an!
\end{enumerate}

\subsection*{Aufgabe 2}
Gegeben sei ein nichtdeterministischer endlicher Automat (NEA):\\
$\mathcal{M} = (\mathcal{Q},\Sigma,\delta,q_0,\mathcal{F})$ mit
$\mathcal{Q} = \{q_0,q_1,q_2\}, \Sigma = \{a,b\}, \mathcal{F} = \{q_2\}$
und $\delta$ gegeben durch:
\begin{center}
\begin{tikzpicture}[line width=1pt]
\start{0}{0}{q_0}
\tor{0}{0}{1}{a}
\state{1}{0}{q_1}
\rloopt{1}{0}{a,b}
\tor{1}{0}{2}{b}
\final{2}{0}{q_2}
\end{tikzpicture}
\end{center}
\begin{enumerate}
\item Geben Sie einen entsprechenden deterministischen endlichen Automaten
(DEA) an, der die gleiche\\
Sprache akzeptiert! Benutzen Sie hierbei das Potenzmengenkonstruktionsverfahren!
\item Ist der entstandene Automat vollständig? Wenn nicht, wie kann man den
Automaten vervollständigen?\\
Welche Mengen stellen bei dem Potenzmengenkonstruktionsverfahren einen
Fehlerzustand dar?
\end{enumerate}

\subsection*{Aufgabe 3}
Gegeben sei der folgende deterministische endliche Automat:\\
$\mathcal{M} = (\mathcal{Q},\Sigma,\delta,q_0,\mathcal{F})$ mit
$\mathcal{Q} = \{q_0,q_1,q_2,q_3,q_4,q_5\}, \Sigma = \{0,1\}, \mathcal{F} =
\{q_3,q_5\}$ und $\delta$ gegeben durch:
\begin{center}
\begin{tikzpicture}[line width=1pt]
\start{0}{0}{q_0}
\tobr{0}{0}{1}{-1}{0}
\tor{0}{0}{1}{1}
\state{1}{0}{q_1}
\tor{1}{0}{2}{0}
\rloopt{1}{0}{1}
\state{2}{0}{q_2}
\tor{2}{0}{3}{0}
\final{3}{0}{q_3}
\rloopt{3}{0}{1}
\state{1}{-1}{q_4}
\tor{1}{-1}{2}{0}
\final{2}{-1}{q_5}
\draw (5,-1) node[below right]{$1$};
\draw [->] (4.283,-1.717) -- (5.717,-0.283);
\end{tikzpicture}
\end{center}
\begin{enumerate}
\item Vervollständigen Sie den Automaten, d.h. führen Sie einen Fehlerzustand ein!
\item Minimieren Sie den vervollständigten Automaten!
\end{enumerate}

\newpage

\subsection*{Aufgabe 4}
Gegeben seien die folgenden beiden nichtdeterministischen endlichen Automaten:
\begin{center}
\begin{tikzpicture}[line width=1pt]
\startfinal{0}{0}{q_1}
\rloopt{0}{0}{a}
\draw [->] (0.4,0) arc (90:-90:1cm);
\draw (1.4,-1) node[right]{$a,b$};
\state{0}{-1}{q_2}
\tot{0}{-1}{0}{b}
\start{2}{0}{s_1}
\tor{2}{0}{3}{\varepsilon}
\tob{2}{0}{-1}{a}
\final{3}{0}{s_2}
\draw [->] (6,0.4) arc (0:180:1cm);
\draw (5,1.4) node[above]{$a$};
\state{2}{-1}{s_3}
\draw (5,-1) node[below right]{$a,b$};
\draw [->] (4.283,-1.717) -- (5.717,-0.283);
\rloopb{2}{-1}{b}
\end{tikzpicture}
\end{center}
Wandeln Sie diese mittels des Potenzmengenkonstruktionsverfahrens in
deterministische endliche\\
Automaten um!

\newpage

\subsection*{Lösung zu Aufgabe 1}
\begin{enumerate}
\item Ja!
\item Der minimale Automat:
\begin{center}
\begin{tikzpicture}[line width=1pt]
\state{0}{0}{s_0s_2}
\draw [->] (-0.8,0.8) -- (-0.23,0.33);
\tor{0}{0}{1}{0}
\lloopl{0}{0}{1}
\state{1}{0}{s_1s_3}
\tor{1}{0}{2}{0}
\draw [->] (2,0.4) arc (0:180:1cm);
\draw (1,1.4) node[above]{$1$};
\final{2}{0}{s_4}
\lloopr{2}{0}{0,1}
\draw (0,-2) node{$[\varepsilon]$};
\draw [->] (0,-1) -- (0,-0.5);
\draw [->] (0,-1) -- (0,-1.7);
\draw (2,-2) node{$[0]$};
\draw [->] (2,-1) -- (2,-0.5);
\draw [->] (2,-1) -- (2,-1.7);
\draw (4,-2) node{$[00]$};
\draw [->] (4,-1) -- (4,-0.5);
\draw [->] (4,-1) -- (4,-1.7);
\end{tikzpicture}
\end{center}
\item Die Äquivalenzklassen der Zustände vom minimalen Automaten:\\
$[\varepsilon] \; = \{w \in \{0,1\}^* \; | \; w \; \mbox{enthält nicht 00 und
endet nicht mit 0}\}$\\
$[0] \; = \{w \in \{0,1\}^* \; | \; w \; \mbox{enthält nicht 00 und endet mit 0}\}$\\
$[00] = \{w \in \{0,1\}^* \; | \; w \; \mbox{enthält 00}\}$
\end{enumerate}

\subsection*{Lösung zu Aufgabe 2}
\begin{enumerate}
\item
$\begin{array}[t]{c|c|c}
& a & b \\
\hline
\{q_0\} & \{q_1\} & \varnothing \\
\hline
\{q_1\} & \{q_1\} & \{q_1,q_2\} \\
\hline
\varnothing & \varnothing & \varnothing \\
\hline
\{q_1,q_2\} & \{q_1\} & \{q_1,q_2\}
\end{array}$\\[4pt]
Der DEA sieht folgendermaßen aus:
\begin{center}
\begin{tikzpicture}[line width=1pt]
\start{0}{0}{s_0}
\tor{0}{0}{1}{a}
\tob{0}{0}{-1}{b}
\state{1}{0}{s_1}
\rloopt{1}{0}{a}
\tor{1}{0}{2}{b}
\state{0}{-1}{s_2}
\rloopl{0}{-1}{a,b}
\final{2}{0}{s_3}
\draw [->] (4,-0.4) arc (0:-180:1cm);
\draw (3,-1.4) node[below]{$a$};
\rloopt{2}{0}{b}
\end{tikzpicture}
\end{center}
Hierbei sind die neuen Zustände durch die Potenzmengenkonstruktion (PMK):\\
$s_0 = \{q_0\}, s_1 = \{q_1\}, s_2 = \varnothing, s_3 = \{q_1,q_2\}$
\item Beim Potenzmengenkonstruktionsverfahren stellt zunächst die leere Menge,
aber auch jede Menge, die nur\\
Fehlerzustände des ursprünglichen Automaten enthält, im neuen Automaten einen
Fehlerzustand dar.
\end{enumerate}

\newpage

\subsection*{Lösung zu Aufgabe 3}
\begin{enumerate}
\item Der vervollständigte Automat:\\
$\mathcal{M} = (\mathcal{Q},\Sigma,\delta,q_0,\mathcal{F})$ mit
$\mathcal{Q} = \{q_0,q_1,q_2,q_3,q_4,q_5,q_6\}, \Sigma = \{0,1\}, \mathcal{F} =
\{q_3,q_5\}$ und $\delta$ gegeben durch:
\begin{center}
\begin{tikzpicture}[line width=1pt]
\start{0}{0}{q_0}
\tobr{0}{0}{1}{-2}{0}
\tor{0}{0}{1}{1}
\state{1}{0}{q_1}
\tor{1}{0}{2}{0}
\rloopt{1}{0}{1}
\state{2}{0}{q_2}
\tor{2}{0}{3}{0}
\tob{2}{0}{-1}{1}
\final{3}{0}{q_3}
\tobl{3}{0}{2}{-1}{0}
\rloopt{3}{0}{1}
\state{1}{-2}{q_4}
\tor{1}{-2}{3}{0}
\totr{1}{-2}{2}{-1}{1}
\final{3}{-2}{q_5}
\totl{3}{-2}{2}{-1}{0}
\tot{3}{-2}{0}{1}
\state{2}{-1}{q_6}
\rloopl{2}{-1}{0,1}
\end{tikzpicture}
\end{center}
\item Die Minimierung:\\
$\begin{array}[t]{|c|c|c|c|c|c|c|c|}
\hline
q_0 &  X  &  X  &  X  &  X  &  X  &  X  &  X \\
\hline
q_1 &     &  X  &  X  &  X  &  X  &  X  &  X \\
\hline
q_2 &  1  &  1  &  X  &  X  &  X  &  X  &  X \\
\hline
q_3 &  0  &  0  &  0  &  X  &  X  &  X  &  X \\
\hline
q_4 &  1  &  1  &     &  0  &  X  &  X  &  X \\
\hline
q_5 &  0  &  0  &  0  &     &  0  &  X  &  X \\
\hline
q_6 &  2  &  2  &  1  &  0  &  1  &  0  &  X \\
\hline
    & q_0 & q_1 & q_2 & q_3 & q_4 & q_5 & q_6 \\
\hline
\end{array}$\\[4pt]
Der minimierte Automat:\\
$\mathcal{M} = (\mathcal{Q},\Sigma,\delta,s_0,\mathcal{F})$ mit
$\mathcal{Q} = \{s_0,s_1,s_2,s_3\}, \Sigma = \{0,1\}, \mathcal{F} = \{s_2\}$
und $\delta$ gegeben durch:
\begin{center}
\begin{tikzpicture}[line width=1pt]
\start{0}{0}{s_0}
\tor{0}{0}{1}{0}
\rloopt{0}{0}{1}
\state{1}{0}{s_1}
\tor{1}{0}{2}{0}
\tob{1}{0}{-1}{1}
\final{2}{0}{s_2}
\tobl{2}{0}{1}{-1}{0}
\rloopt{2}{0}{1}
\state{1}{-1}{s_3}
\rloopl{1}{-1}{0,1}
\end{tikzpicture}
\end{center}
Hierbei ersetzen die neuen Zustände durch die Minimierung die alten Zustände
wie folgt:\\
$s_0 \; \mbox{ersetzt} \; q_0,q_1$\\
$s_1 \; \mbox{ersetzt} \; q_2,q_4$\\
$s_2 \; \mbox{ersetzt} \; q_3,q_5$\\
$s_3 \; \mbox{ersetzt} \; q_6$
\end{enumerate}

\newpage

\subsection*{Lösung zu Aufgabe 4}
$\begin{array}[t]{l|c|c}
& a & b \\
\hline
r_1 := \{q_1\} & \{q_1,q_2\} & \{q_2\} \\
\hline
r_2 := \{q_1,q_2\} & \{q_1,q_2\} & \{q_1,q_2\} \\
\hline
r_3 := \{q_2\} & \varnothing & \{q_1\} \\
\hline
r_4 := \varnothing & \varnothing & \varnothing
\end{array}$\\[4pt]
Der 1. DEA sieht folgendermaßen aus:
\begin{center}
\begin{tikzpicture}[line width=1pt]
\startfinal{0}{0}{r_1}
\tob{0}{0}{-1}{a}
\tor{0}{0}{1}{b}
\final{0}{-1}{r_2}
\rloopr{0}{-1}{a,b}
\state{1}{0}{r_3}
\tor{1}{0}{2}{a}
\draw [->] (2,0.4) arc (0:180:1cm);
\draw (1,1.4) node[above]{$b$};
\state{2}{0}{r_4}
\rloopr{2}{0}{a,b}
\end{tikzpicture}
\end{center}

$\begin{array}[t]{l|c|c}
& a & b \\
\hline
z_1 := \{s_1,s_2\} & \{s_1,s_2,s_3\} & \varnothing \\
\hline
z_2 := \{s_1,s_2,s_3\} & \{s_1,s_2,s_3\} & \{s_2,s_3\} \\
\hline
z_3 := \varnothing & \varnothing & \varnothing \\
\hline
z_4 := \{s_2,s_3\} & \{s_1,s_2\} & \{s_2,s_3\}
\end{array}$\\[4pt]
Der 2. DEA sieht folgendermaßen aus:
\begin{center}
\begin{tikzpicture}[line width=1pt]
\startfinal{0}{0}{z_1}
\tor{0}{0}{1}{a}
\tob{0}{0}{-1}{b}
\final{1}{0}{z_2}
\rloopr{1}{0}{a}
\tob{1}{0}{-1}{b}
\state{0}{-1}{z_3}
\rloopl{0}{-1}{a,b}
\final{1}{-1}{z_4}
\totl{1}{-1}{0}{0}{a}
\rloopr{1}{-1}{b}
\end{tikzpicture}
\end{center}

\end{document}