\documentclass[10pt,oneside,onecolumn,a4paper,german,titlepage]{article}
\usepackage[utf8]{inputenc}
\usepackage[german]{babel}
\usepackage{german}
\usepackage{amsmath,amsfonts,amssymb,latexsym,textcomp,stmaryrd}
\usepackage{tikz}
\usetikzlibrary{automata,positioning}
\pagestyle{plain}
\pagenumbering{arabic}
\parskip0.5ex plus0.1ex minus0.1ex
\parindent0pt
\voffset-1.04cm
\topmargin0pt
\headheight0pt
\headsep0pt
\topskip0pt
\textheight26.7cm
\footskip20pt
\hoffset-1.04cm
\oddsidemargin0pt
\evensidemargin0pt
\textwidth18cm
\marginparsep0pt
\marginparwidth0pt
\begin{document}

\section*{Tutorien-"Ubungsblatt 8}

\subsection*{Aufgabe 1}
Ein Algorithmus, der eine Zahl $n \in \mathbb{N}$ als Eingabe erh"alt und pr"uft,
ob $n$ prim ist, sei gegeben durch die\\
nachfolgende Beschreibung einer Turingmaschine $\mathcal{M}$, die die Funktion
$f: \mathbb{N} \to \{0,1\}, n \mapsto
\left\{\begin{array}{c@{\hspace{0.2cm},\hspace{0.2cm}}l}
1 & n \; \mbox{prim}\\0 & \mbox{sonst}
\end{array}\right.$\\
realisiert:\\[4pt]
1. Initialisiere Z"ahler $z$ mit $2$ und Ausgabe $a$ mit $1$\\
2. Berechne $n \; \% \; z$\\
3. Pr"ufe, ob $n \; \% \; z = 0$ gilt:\\
- Falls ja: Setze $a$ auf $0$ und gehe zu Schritt 4.\\
- Falls nein: Gehe zu Schritt 4.\\
4. Pr"ufe, ob $z < n$ gilt:\\
- Falls ja: Erh"ohe $z$ um $1$ und gehe zu Schritt 2.\\
- Falls nein: L"osche das Band, schreibe $a$ auf das Band und stoppe\\[4pt]
Geben Sie die Komplexit"at des oben angegebenen Algortihmus bez"uglich der Eingabe
$n$ und auch bez"uglich der\\
L"ange der Bin"ardarstellung von $n$ an!

\subsection*{Aufgabe 2}
Die Komplexit"atsklasse \textbf{co-P} sei definiert als die Menge der Sprachen
$\mathcal{L}$, deren Komplementsprache $\mathcal{L}^C$ in der\\
Komplexit"atsklasse \textbf{P} liegt.\\
\underline{Erinnerung:} Zu einer Sprache $\mathcal{L}$ "uber einem Alphabet $\Sigma$
ist die Komplementsprache $\mathcal{L}^C = \Sigma^* \setminus \mathcal{L}$.\\[4pt]
Beweisen Sie: \textbf{co-P} = \textbf{P}

\subsection*{Aufgabe 3 - Graph-2-F"arbbarkeit}
Gegeben seien ein ungerichteter Graph $G = (V,E)$ mit Knoten $v \in V$ und
Kanten $e = (v_1,v_2) \in E$ mit $v_1, v_2 \in V$\\ und zwei Farben $A$ und $B$.\\
Beweisen Sie: Die Sprache $\mbox{2-COLOR} \; =$\\
$\{G \; | \; G = (V,E) \; \mbox{ungerichteter Graph mit} \; \exists \; \mbox{totale
Funktion} \; g: V \to \{A,B\}: \forall \; (v_1, v_2) \in E: g(v_1) \neq g(v_2)\}$\\
liegt in der Komplexit"atsklasse \textbf{P}.

\subsection*{Aufgabe 4}
Das Problem 2-SAT ist folgenderma"sen definiert:\\
\begin{figure}[ht]
\setbox0\vbox{\small
\textbf{2-SAT}\\
Gegeben eine in ihrer Gr"o"se polynomiell beschr"ankte aussagenlogische Formel $F$
in konjuktiver Normalform, wobei jede Klausel genau 2 Literale enth"alt. $F$ hat
also die Form
\begin{eqnarray*}F = \Lambda_{i = 1}^n (L_i \vee N_i),\end{eqnarray*}
wobei $L_i$ und $N_i$ Literale sind, also von der Form $X$ oder $\neg X$ f"ur eine
Variable $X$ sind.\\
Gibt es eine erfüllende Belegung für $F$?
}
\centerline{\fbox{\box0}}
\end{figure}\\
Geben Sie eine polynomielle Reduktion von 2-COLOR auf 2-SAT an!

\newpage

\subsection*{L"osung zu Aufgabe 1}
Die Komplexit"at bez"uglich der Eingabe $n$ ist $O(n)$. Sei die L"ange der
Bin"ardarstellung von $n$ bezeichnet mit $|n|$, dann gilt: $|n| = \log_2(n)$.
Daraus folgt, dass $n = 2^{|n|}$ ist. Somit ist die Komplexit"at in $O(2^{|n|})$.

\subsection*{L"osung zu Aufgabe 2}
\textbf{P} $\subseteq$ \textbf{co-P}:\\
Sei $\mathcal{L} \in$ \textbf{P}. Dann gibt es eine polynomiale deterministische
Turingmaschine $T$ mit der Zustandsmenge $Q$ und der Endzustandsmenge $F \subseteq
Q$, die $\mathcal{L}$ akzeptiert, die also f"ur alle Eingaben $w \in \Sigma^*$ nach
polynomialer Zeit h"alt und nur genau dann in einem Endzustand $q \in F$ h"alt,
wenn $w \in \mathcal{L}$ gilt. Konstruiere daraus die polynomiale deterministische
Turingmaschine $T'$, die "ahnlich aussieht wie $T$ und nur folgende "Anderungen
aufweist:
\begin{enumerate}
\item Alle bisherigen Endzust"ande $q \in F$ sind in $T'$ \underline{keine}
Endzust"ande mehr.
\item Die Zustandsmenge $Q$ wird um zwei \underline{neue} Zust"ande $q^+,q^-$
erweitert, wobei $T'$ in diesen beiden Zust"anden f"ur jedes Zeichen $a \in \Gamma$
immer sofort h"alt und $q^+$ der einzige Endzustand von $T'$ ist.
\item Statt in einem ehemaligen Endzustand $q \in F$ zu halten, geht $T'$ von diesem
Zustand f"ur jedes Zeichen $a \in \Gamma$ in den neuen Zustand $q^-$ "uber.
\item Statt f"ur ein Zeichen $a \in \Gamma$ in einem Zustand $q \in Q \setminus F$
zu halten, geht $T'$ von diesem Zustand f"ur das Zeichen $a$ in den neuen Zustand
$q^+$ "uber.
\end{enumerate}
$T'$ akzeptiert genau die Komplement"arsprache $\mathcal{L}^C$ und es gilt daher
$\mathcal{L}^C \in$ \textbf{P}. Damit gilt also auch $\mathcal{L} \in$ \textbf{co-P}.
\\[4pt]
\textbf{co-P} $\subseteq$ \textbf{P}:\\
Sei $\mathcal{L} \in$ \textbf{co-P}. Dann ist die Komplementsprache $\mathcal{L}^C
\in$ \textbf{P}. Damit gibt es eine polynomiale deterministische Turingmaschine
$T^C$, die $\mathcal{L}^C$ in polynomialer Zeit entscheidet. Analog zu oben kann
man daraus eine polynomiale deterministische Turingmaschine $T$ konstruieren, die
$\mathcal{L}$ akzeptiert. Damit ist $\mathcal{L} \in$ \textbf{P}.
\\[4pt]
Wir haben nun gezeigt, dass \textbf{P} $\subseteq$ \textbf{co-P} und \textbf{co-P}
$\subseteq$ \textbf{P}, damit erhalten wir insgesamt \textbf{P} = \textbf{co-P}.

\subsection*{L"osung zu Aufgabe 3}
\textbf{Vorgehen:}\\
Wir zeigen zuerst, dass ein Graph genau dann 2-f"arbbar ist, wenn er bipartit
(siehe unten) ist. Es reicht dann zum Beweisen von 2-COLOR $\in$ \textbf{P}, einen
polynomialen Algorithmus anzugeben, der pr"uft, ob ein Graph bipartit ist.\\[4pt]
\textbf{Definition:}\\
Ein Graph $G = (V,E)$ hei"st \textit{bipartit}, wenn sich seine Knoten in zwei
disjunkte Mengen $M$ und $N$ einteilen lassen, so dass keine Knoten innerhalb der
Teilmengen miteinander durch eine Kante verbunden sind. D.h. f"ur jede Kante
$(v_1, v_2) \in E$ gilt $(v_1 \in M) \wedge (v_2 \in N)$ oder $(v_1 \in N) \wedge
(v_2 \in M)$.\\[4pt]
Jeder bipartite Graph ist 2-f"arbbar:\\
Sei $G = (V,E)$ ein bipartiter Graph. Damit gibt es zwei disjunkte Teilmengen $M$ und
$N$ wie in der obigen Definition von bipartit beschrieben. F"arbe nun alle Knoten in
der Menge $M$ mit der Farbe $A$ und alle Knoten in der Menge $N$ mit der Farbe $B$.
Zwei Knoten haben genau dann die gleiche Farbe, wenn sie in der gleichen Menge ($M$
oder $N$) liegen. Da Knoten innerhalb der jeweiligen Mengen nicht durch Kanten
miteinander verbunden sind, und damit nur Kanten zwischen einem Knoten aus $M$ und
einem Knoten aus $N$ existieren, haben zwei durch Kanten verbundene Knoten nie die
gleiche Farbe. Damit ist $G$ 2-f"arbbar, d.h. $G \in$ 2-COLOR.\\[4pt]
Jeder 2-f"arbbare Graph ist bipartit:\\
Sei $G \in$ 2-COLOR ein 2-f"arbbarer Graph. Dann gibt es eine 2-F"arbung $g: V \to
\{A,B\}$ f"ur $G$. Definiere die Mengen $M$ und $N$ so, dass in $M$ alle Knoten $v$
mit der Farbe $g(v) = A$ und in $N$ alle Knoten $v$ mit der Farbe $g(v) = B$ liegen.
Knoten innerhalb von $M$ oder innerhalb von $N$ k"onnen nicht durch Kanten
miteinander verbunden sein, denn sonst w"are ja $g$ keine korrekte 2-F"arbung. Damit
haben wir die Knoten von $G$ in zwei disjunkte Mengen unterteilt, so dass keine
Kanten zwischen Knoten innerhalb einer Menge existieren. Damit ist $G$ bipartit.
\\[4pt]
Insgesamt haben wir nun gezeigt, dass ein Graph $G$ genau dann in 2-COLOR ist, wenn
er bipartit ist. Um zu testen, ob ein Graph $G$ in 2-COLOR ist, k"onnen wir also
einen Algorithmus verwenden, der testet, ob $G$ bipartit ist.

\newpage

Ein Algorithmus, der in polynomialer Zeit pr"uft, ob ein Graph $G = (V, E)$ bipartit
ist, ist z.B. der folgende:\\[4pt]
1. Setze zu Beginn die zwei Mengen $M$ und $N$ auf die leere Menge: $M = N =
\varnothing$.\\
2. Falls der Graph nicht zusammenh"angend ist, f"uhre den Rest des Algorithmus f"ur
jeden der nicht miteinander verbundenen Teilgraphen aus.\\
3. Beginne mit einem beliebigen Knoten $v_0 \in V$. F"uge $v_0$ zu $M$ hinzu und
f"uge au"serdem alle Knoten, die mit $v_0$ durch eine Kante verbunden sind, der
Menge $N$ hinzu.\\
4. F"ur alle Knoten $v \in M$: F"uge alle Knoten, die mit $v$ durch eine Kante
verbunden und noch nicht in der Menge $M$ oder $N$ sind, der Menge $N$ hinzu.\\
5. F"ur alle Knoten $v \in N$: F"uge alle Knoten, die mit $v$ durch eine Kante
verbunden und noch nicht in der Menge $M$ oder $N$ sind, der Menge $M$ hinzu.\\
6. Wiederhole 4. und 5., bis alle Knoten von $G$ in einer der beiden Mengen $M$
oder $N$ sind.\\
7. Gib als Ergebnis aus: Falls Knoten innerhalb von $M$ oder innerhalb von $N$
durch eine Kante miteinander verbunden sind, ist der Graph nicht bipartit.
Ansonsten ist der Graph bipartit.\\[4pt]
\textbf{Aufwand:}\\
Dieser Algorithmus hat polynomialen Aufwand in der Anzahl $n$ der Knoten: Das
Sortieren in die Mengen $M$ und $N$ hat einen Aufwand von $O(n^2)$ (F"ur jeden
Knoten alle Kanten durchgehen: $O(n^2)$, f"ur jeden verbundenen Knoten testen,
ob er schon in $M$ oder $N$ liegt: $O(n)$). Das Testen, ob eine Kante zwischen
zwei Knoten aus $M$ oder zwei Knoten aus $N$ existiert, hat auch einen Aufwand
von $O(n^2)$. Damit hat der Algorithmus insgesamt polynomiale Laufzeit in der
L"ange der Darstellung des Graphen (die etwa der Anzahl Knoten entspricht).\\[4pt]
\textbf{Korrektheit:}\\
Der Algorithmus funktioniert deshalb, weil in einem bipartiten Graphen $G = (V,E)$
und der Aufteilung in die zwei Mengen $M$ und $N$ (nach der Definition) f"ur jeden
Knoten $v \in M$ gilt, dass alle mit $v$ verbundenen Knoten in $N$ liegen, und
umgekehrt. Der Algorithmus geht also erstmal davon aus, der Graph sei bipartit
und stellt dann eine Partition in die Mengen $M$ und $N$ her. Falls $G$ wirklich
bipartit ist, ist das Ergebnis eine korrekte Aufteilung in zwei disjunkte Mengen,
innerhalb derer keine Kanten existieren. Falls nicht, so kann keine solche korrekte
Aufteilung herauskommen, und damit m"ussen Knoten innerhalb von $M$ oder innerhalb
von $N$ miteinander verbunden sein.

\subsection*{L"osung zu Aufgabe 4}
Sei $G = (V,E)$ ein ungerichteter Graph, f"ur den wir feststellen wollen, ob $G \in$
2-COLOR gilt. Dazu seien wieder die beiden Farben $A$ und $B$ gegeben. Wir bilden
f"ur die Reduktion zun"achst die Menge der zur Verf"ugung stehenden Variablen
$\chi := \{X_v \; | \; v \in V\}$. Falls der gegebene Graph $G$ 2-f"arbbar ist, so
gibt es eine 2-F"arbung $g: V \to \{A,B\}$, andernfalls gibt es zumindest eine
Abbildung $g: V \to \{A,B\}$. Diese verwenden wir zur Definition einer
Variablenbelegung:\\[4pt]
$f: \chi \to \{\mbox{TRUE},\mbox{FALSE}\}, X_v \mapsto
\left\{\begin{array}{c@{\hspace{0.5cm},\hspace{0.5cm}}l}
\mbox{TRUE} & \mbox{falls} \; g(v) = A\\
\mbox{FALSE} & \mbox{falls} \; g(v) = B
\end{array}\right.$\\[4pt]
Nun bilden wir jede Kante $(u,v) \in E$ auf die passende aussagenlogische Formel
$(X_u \oplus X_v)$ ab. Die obige Variablenbelegung $f$ ergibt genau dann f"ur alle
diese aussagenlogischen Formeln eine gemeinsame erf"ullende Variablenbelegung, wenn
die zugrunde liegende Abbildung $g$ bereits eine korrekte 2-F"arbung und der Graph
$G$ damit 2-f"arbbar ist. Nun m"ussen die Formeln noch in die richtige Form gebracht
werden:\\[4pt]
$(X_u \oplus X_v) \cong ((X_u \wedge \neg X_v) \vee (\neg X_u \wedge X_v)) \cong
((X_u \vee X_v) \wedge (\neg X_u \vee \neg X_v))$\\[4pt]
Wir erhalten somit f"ur jede Kante $(u,v) \in E$ die beiden Klauseln $X_u \vee X_v,
\neg X_u \vee \neg X_v$ und f"ur den gesamten Graphen $G$ die 2-SAT-Instanz
$K := \bigcup\limits_{(u,v) \in E} \{X_u \vee X_v,\neg X_u \vee \neg X_v\}$, die
also genau dann eine erf"ullende Belegung besitzt, wenn $G$ 2-f"arbbar ist. Wir haben
damit also eine Reduktion gefunden und m"ussen nur noch zeigen, dass diese mit
polynomialem Aufwand m"oglich ist. Dies ist aber recht einfach einzusehen, da zur
Bildung der Klauselmenge $K$ nach deren Definition nur einmal alle Kanten durchlaufen
werden müssen, der Aufwand liegt also in $O(|E|) = O(|V|^2)$.
\end{document}