\documentclass[10pt,oneside,onecolumn,a4paper,german,titlepage]{article}
\usepackage[utf8]{inputenc}
\usepackage[german]{babel}
\usepackage{german}
\usepackage{amsmath,amsfonts,amssymb,latexsym,textcomp,stmaryrd}

\usepackage{graphics}
\usepackage{graphicx}
\usepackage{tikz}
\usepackage{color}
\usepackage{enumerate}

\pagestyle{plain}
\pagenumbering{arabic}
\parskip0.5ex plus0.1ex minus0.1ex
\parindent0pt
\voffset-1.04cm
\topmargin0pt
\headheight0pt
\headsep0pt
\topskip0pt
\textheight26.7cm
\footskip20pt
\hoffset-1.04cm
\oddsidemargin0pt
\evensidemargin0pt
\textwidth18cm
\marginparsep0pt
\marginparwidth0pt
\begin{document}

\section*{Tutorien-"Ubungsblatt 10}

\subsection*{Aufgabe 1}
Gegeben ist das folgende Problem:
\begin{tabbing}
HALF-CLIQUE:\\
\textit{Gegeben:} \= Ein ungerichteter Graph $G = (V,E)$\\
\textit{Gesucht:} \> Gibt es eine Teilmenge $V' \subseteq V$ mit
$\forall \; v,w \in V', v \not= w: (v,w) \in E$ und $|V'| \geq |V|/2$
\end{tabbing}
Beweisen Sie, dass HALF-CLIQUE \textbf{NP}-vollst"andig ist!\\[4pt]
\underline{Zur Erinnerung:}\\
Das als \textbf{NP}-vollst"andig bekannte Problem CLIQUE ist definiert durch:
\begin{tabbing}
CLIQUE:\\
\textit{Gegeben:} \= Ein ungerichteter Graph $G = (V,E)$ und $k \in \mathbb{N}$\\
\textit{Gesucht:} \> Gibt es eine Teilmenge $V' \subseteq V$ mit
$\forall \; v,w \in V', v \not= w: (v,w) \in E$ und $|V'| \geq k$
\end{tabbing}

\subsection*{Aufgabe 2}
Finden Sie den Fehler im folgenden ``Beweis'' f"ur \textbf{P} $\not=$ \textbf{NP}!\\
Betrachten Sie folgenden Algorithmus f"ur SAT:\\[4pt]
- Durchlaufe f"ur die gegebene Formel $\phi$ alle m"oglichen Belegungen der
Variablen mit den Wahrheitswerten\\
- Akzeptiere $\phi$, wenn eine der durchlaufenen Belegungen $\phi$ erf"ullt\\[4pt]
Dieser Algorithmus hat eine mit der Anzahl der Variablen exponentiell wachsende
Laufzeit. Daher hat das Problem SAT einen exponentiellen Aufwand und kann nicht in
\textbf{P} liegen. Weil aber SAT in \textbf{NP} liegt, mu"s also \textbf{P} $\not=$
\textbf{NP} gelten.

\subsection*{Aufgabe 3}
\begin{enumerate}
\item Zeigen Sie, dass es unter der Voraussetzung \textbf{P} $=$ \textbf{NP} m"oglich
ist, f"ur eine aussagenlogische Formel $\phi$ in\\
polynomieller Zeit eine erf"ullende Belegung der Variablen zu finden, falls eine
solche Belegung existiert!
\end{enumerate}

\subsection*{Aufgabe 4}
\begin{enumerate}
 \item Gegeben sind folgende Probleme: 
 \begin{tabbing}
 \textbf{Hamiltonkreisproblem:} \\
 \hspace{10pt} \= \textit{Gegeben:} \= Ein ungerichteter Baum $G=(V,E)$.\\
 \> \textit{Gesucht:} \> Besitzt $G$ einen Hamiltonkreis? (Dies ist eine Permutation $\pi$ der Knotenindizes\\ 
 \> \> ($v_{\pi(1)}$,$v_{\pi(2)}$,...,$v_{\pi(n)}$), sodass für $i=1,...,n-1$ gilt: $\{v_{\pi(i)},v_{\pi(i+1)}\}\in E$) und \\ 
 \> \> außerdem $\{v_{\pi(n)},v_{\pi(1)}\} \in E)$.\\ \\
 \textbf{Travelling Salesman(TSP):}\\
 \> \textit{Gegeben:} Ein Graph $G=(V,V \times V)$\\
 \> \textit{Gesucht:} \> Ein einfacher Kreis $C=(v_1,v_2,...,v_n,v_1)$, sodass $n=|V|$ und $\sum_{(u,v)\in C} d(u,v)$\\
 \> \> minimiert wird, wobei $d(u,v)$ die Entfernung zwischen den Knoten $u$ und $v$ ist.\\
\end{tabbing}
Zeigen Sie, dass TSP NP-Vollständig ist, wobei das Hamiltonkreisproblem auch NP-Vollständig ist. Benutzen Sie für den Beweis die Reduktion Hamiltonkreisproblem$\leq_p$TSP. 
\item Gegeben sei folgender Graph:\newline
\begin{center}
\begin{tikzpicture}
 \draw (0,0) circle (8pt);
 \draw (0,0) node {$v_0$};
 \draw (2,0) circle (8pt);
 \draw (2,0) node {$v_1$};
 \draw (2,2) circle (8pt);
 \draw (2,2) node {$v_2$};
 \draw (1,3) circle (8pt);
 \draw (1,3) node {$v_3$};
 \draw (0,2) circle (8pt);
 \draw (0,2) node {$v_4$};
 \draw (0.2,0.2) -- (1.8,1.8);
 \draw (0.2,1.8) -- (1.8,0.2);
 \draw (0.2,2.2) -- (0.8,2.8);
 \draw (1.2,2.8) -- (1.8,2.2);
 \draw (0.3,0) -- (1.7,0);
 \draw (0.3,2) -- (1.7,2);
 \draw (0,0.3) -- (0,1.7);
 \draw (2,0.3) -- (2,1.7);
\end{tikzpicture}
\end{center}
Gibt es einen Hamiltonkreis? Wandeln Sie hierzu das Problem in ein TSP um und finden Sie eine optimale Rundtour.
\end{enumerate}

\newpage

\subsection*{L"osung zu Aufgabe 1}
HALF-CLIQUE ist in \textbf{NP}, da f"ur ein Teilgraph der Gr"o"se $|V|/2$ eines
ungerichteten Graphen $G = (V,E)$ in $O(|V|^2 \cdot |E|) = O(|V|^4)$ gepr"uft werden
kann, ob dieser Teilgraph vollst"andig und damit eine Clique ist.\\[4pt]
Zum Nachweis der \textbf{NP}-H"arte von HALF-CLIQUE zeigen wir die polynomielle
Reduktion CLIQUE $\leq_p$ HALF-CLIQUE. Dazu sei eine Instanz $(G,k)$ mit $G = (V,E)$
des Problems CLIQUE gegeben.\\
\underline{Fall 1:} $k = |V|/2$\\
Dann ist $G$ bereits die gesuchte Instanz von HALF-CLIQUE und es gilt $(G,k) \in$
CLIQUE $\Leftrightarrow G \in$ HALF-CLIQUE.\\
\underline{Fall 2:} $k > |V|/2$\\
Bilde den Graphen $G' = (V',E')$ aus $G$ durch das Hinzuf"ugen $2k - |V| > 0$ Knoten,
die mit keinem anderen Knoten in $V'$ verbunden sind, also den Grad 0 haben. Es gilt
$|V'|/2 = (|V| + 2k - |V|)/2 = k$. Damit gilt $(G,k) \in \mbox{CLIQUE}
\Leftrightarrow G \; \mbox{hat eine Clique mit Umfang} \; k
\Leftrightarrow G' \; \mbox{hat eine Clique mit Umfang} \; |V'|/2
\Leftrightarrow G' \in \mbox{HALF-CLIQUE}$.
Die Bildung von $G'$ aus $(G,k)$ ist in $O(k+|V|)$ m"oglich.\\
\underline{Fall 3:} $k < |V|/2$\\
Bilde den Graphen $G' = (V',E')$ aus $G$ durch das Hinzuf"ugen $|V| - 2k > 0$ Knoten,
die mit jedem anderen Knoten in $V'$ verbunden sind, also den Grad $|V'| - 1$ haben.
Es gilt $|V'|/2 = (|V| + |V| - 2k)/2 = |V| - k$. Damit gilt $(G,k) \in \mbox{CLIQUE}
\Leftrightarrow G \; \mbox{hat eine Clique mit Umfang} \; k
\Leftrightarrow G' \; \mbox{hat eine Clique mit Umfang} \; k + |V| - 2k = |V'|/2
\Leftrightarrow G' \in \mbox{HALF-CLIQUE}$.
Die Bildung von $G'$ aus $(G,k)$ ist in $O(|V|^2 \cdot |E|) = O(|V|^4)$ m"oglich.

\subsection*{L"osung zu Aufgabe 2}
Die Vorgabe eines Algorithmus mit exponentiellem Aufwand zur L"osung eines gegebenen
Problems bedeutet nicht, dass das Problem eine exponentielle Komplexit"at besitzt und
damit in \textbf{NP} liegt. Der Denkfehler besteht darin, dass aus der Tatsache, dass
man nicht auf triviale Weise einen effizienten (polynomiellen) Algorithmus findet,
geschlossen wird, dass ein solcher effizienter (polynomieller) Algorithmus auch nicht
existieren kann.

\subsection*{L"osung zu Aufgabe 3}
\begin{enumerate}
\item Es ist bekannt, dass SAT $\in$ \textbf{NP} gilt. Wegen der Voraussetzung
\textbf{P} $=$ \textbf{NP} gilt damit auch SAT $\in$ \textbf{P}. F"ur eine gegebene
aussagenlogische Formel $\phi$ seien die Variablen mit $X_1,...,X_n$ f"ur ein
geeignetes $n \in \mathbb{N}$ bezeichnet.\\
Es wird nun folgender Algorithmus betrachtet:\\[4pt]
1. Initialisiere $i$ mit $1$\\
2. Ersetze $X_i$ in $\phi$ mit TRUE und pr"ufe, ob $\phi \in$ SAT gilt\\
- Falls ja: Gehe zu 4.\\
- Falls nein: Gehe zu 3.\\
3. Ersetze $X_i$ in $\phi$ mit FALSE und pr"ufe, ob $\phi \in$ SAT gilt\\
- Falls ja: Gehe zu 4.\\
- Falls nein: Gib aus ``$\phi$ ist nicht erf"ullbar''\\
4. Pr"ufe, ob $i < n$ gilt\\
- Falls ja: Erh"ohe $i$ um $1$ und gehe zu 2.\\
- Falls nein: Gib die aktuelle Belegung f"ur $X_1,...,X_n$ als L"osung aus\\[4pt]
Der Aufwand des Algorithmus ist im wesentlichen das $2n$-fache des Aufwandes von SAT.
Da SAT nach der Voraussetzung dieser Aufgabe einen polynomiellen Aufwand in $n$ hat,
so hat auch der obige Algorithmus einen polynomiellen Aufwand in $n$.
\end{enumerate}

\subsection*{L"osung zu Aufgabe 4}
\begin{enumerate}
 \item TSP ist in NP, da man durch einen nichtdeterminitischen Turingmaschine in polynomielle Zeit feststellen kann (da man n Kanten raten kann), ob die Kantenmenge das TSP löst. Somit ist TSP in NP.\newline Es ist noch zu zeigen, dass das TSP in NP-Hart ist: \newline Sei $G=(V,E)$ beliebiger ungerichteter Graph.\newline Definiere
$d(u,v):=$  
  $\left\{
     \begin{array}{l l}
      &  1, \quad \quad \quad \quad \quad \quad \quad $falls$ (u,v) \in E\\
      &  1+\alpha , \quad \quad \quad \quad \quad  $sonst$ \\
     \end{array}\right.$ \newline \newline
Dann nur dann, wenn $G$ einen Hamiltonkreis hat, gilt $\exists$ TSP Tour mit Kosten $n$. Ansonsten sind die optimale Kosten mindestens $n+\alpha$. Also es existiert einen Rundtour $C$ mit Gewicht $n$, dann gibt es keine Kante, deren Gewicht echt größer als 1 ist. Somit hat jede Kante in $C$ das Gewicht 1 und alle Kanten in $C$ sind in $G$ und $C$ bildet einen Hamiltonkreis in $G$. Umgekehrt hat $G$ einen Hamiltonkreis $C$ und $C$ ist Rundtour mit Gewicht $n$.
\item Eine optimale Rundtour ist z.B.: $(v_0,v_1,v_2,v_3,v_4,v_0)$ mit Kosten 5, also ist diese optimale Rundtour (da ihre Kosten nicht größer als 5 sind) auch ein Hamiltonkreis.
\end{enumerate}


\end{document}