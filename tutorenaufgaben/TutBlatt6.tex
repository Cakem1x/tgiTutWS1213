\documentclass[10pt,oneside,onecolumn,a4paper,german,titlepage]{article}
\usepackage[utf8]{inputenc}
\usepackage[german]{babel}
\usepackage{german}
\usepackage{amsmath,amsfonts,amssymb,latexsym,textcomp,stmaryrd}
\usepackage{tikz}
\usetikzlibrary{automata,positioning}
\pagestyle{plain}
\pagenumbering{arabic}
\parskip0.5ex plus0.1ex minus0.1ex
\parindent0pt
\voffset-1.04cm
\topmargin0pt
\headheight0pt
\headsep0pt
\topskip0pt
\textheight26.7cm
\footskip20pt
\hoffset-1.04cm
\oddsidemargin0pt
\evensidemargin0pt
\textwidth18cm
\marginparsep0pt
\marginparwidth0pt
\begin{document}

\section*{Tutorien-"Ubungsblatt 6}

\subsection*{Aufgabe 1}
Zeigen Sie, dass die Sprache $\mathcal{L} = \{\langle\mathcal{M}\rangle \; | \;
\mbox{Turingmaschine $\mathcal{M}$ hat mindestens einen unerreichbaren Zustand}\}$\\
nicht entscheidbar ist!

\subsection*{Aufgabe 2}
Beweisen Sie, dass es eine G"odelnummer $n = \langle\mathcal{M}\rangle \in
\mathbb{N}_0$ zu einer Turingmaschine $\mathcal{M}$ gibt, die die Funktion\\
$f_n(x) = (n+x)^2$ f"ur alle $x \in \mathbb{N}_0$ berechnet!

\subsection*{Aufgabe 3}
Welche der folgenden Mengen sind rekursiv aufz"ahlbar? Beweisen Sie Ihre Aussage!
\begin{enumerate}
\item $M_1 := \{q \in \mathbb{Q} \; | \; 0<q<1\}$
\item $M_2 := \{r \in \mathbb{R} \; | \; 0<r<1\}$
\end{enumerate}

\subsection*{Aufgabe 4}
Sei $A \subseteq \mathbb{N}_0$ eine entscheidbare Menge. Zeigen Sie, dass
$ B := \{x+2y^2+17+11^x \; | \; x,y \in A\}$ entscheidbar ist!

\subsection*{L"osung zu Aufgabe 1}
\underline{Beweis:} Es gilt: $\mathcal{L} \; \mbox{entscheidbar} \Leftrightarrow
\bar{\mathcal{L}} \; \mbox{entscheidbar}$\\
Zeige also die Reduktion $\mbox{HALT} \leq_m \bar{\mathcal{L}} = \{\langle\mathcal{M}
\rangle \; | \; \mbox{Turingmaschine $\mathcal{M}$ hat keinen unerreichbaren
Zustand}\}$!\\
Konstruiere dazu aus einer Instanz $(\langle\mathcal{M}\rangle,w) \in \mbox{HALT}$,
also aus der Turingmaschine $\mathcal{M}$ und deren Eingabe $w$,\\
eine neue Turingmaschine $\mathcal{M}'$:\\
1. Leere das Band\\
2. Schreibe $w$ auf das Band\\
3. Simuliere $\mathcal{M}$\\
4. Gehe in einen speziellen Zustand $q_S$\\
Dabei hat $\mathcal{M}'$ bez"uglich der Schritte 1. bis 3. keine unerreichbaren
Zust"ande, der einzige potentiell\\
unerreichbare Zustand ist also $q_S$.\\[4pt]
Sei nun $f: \mbox{HALT} \to \bar{\mathcal{L}}, (\langle\mathcal{M}\rangle,w) \mapsto
\langle\mathcal{M}'\rangle$ die totale und berechenbare Funktion, die die Reduktion
nach der\\
obigen Beschreibung liefert. Dann gilt:\\[4pt]
$(\langle\mathcal{M}\rangle,w) \in \mbox{HALT} \Leftrightarrow \mathcal{M} \;
\mbox{h"alt bei der Eingabe von} \; w \Leftrightarrow \mathcal{M}' \;
\mbox{erreicht den Zustand} \; q_S \Leftrightarrow$\\
$\mathcal{M}' \; \mbox{hat keinen unerreichbaren Zustand} \Leftrightarrow \langle
\mathcal{M}'\rangle = f((\langle\mathcal{M}\rangle,w)) \in \bar{\mathcal{L}}$\\[4pt]
Damit ergibt sich aus der Annahme, dass $\bar{\mathcal{L}}$ berechenbar ist, direkt,
dass auch $\mbox{HALT}$ berechenbar ist. Dies ist\\
aber ein Widerspruch, da $\mbox{HALT}$ als nicht berechenbar bekannt ist. Damit
kann also $\bar{\mathcal{L}}$ und damit auch $\mathcal{L}$\\
nicht berechenbar sein.

\newpage

\subsection*{L"osung zu Aufgabe 2}
Wir wenden das Rekursionstheorem auf folgende Turingmaschine $\mathcal{M}$ an, welche eine Eingabe $x$ erh"alt:\\[4pt]
1. Hole eigene Beschreibung $n = \langle \mathcal{M} \rangle$\\
2. Berechne $y = n + x$\\
3. Berechne $z = y^2$\\
4. Gib $z$ aus\\[4pt]
Sei $n = \langle\mathcal{M}'\rangle$ die G"odelnummer von $\mathcal{M}$. $\mathcal{M}$ berechnet also die Funktion $f_n(x) = (n+x)^2$.


\subsection*{L"osung zu Aufgabe 3}
\begin{enumerate}
\item Es wird als bekannt vorausgesetzt, dass es berechenbare Bijektionen von
$\mathbb{N}_0$ nach $\mathbb{N}_0^2$ gibt. Sei $f$ eine beliebige, aber feste
derartige Bijektion. Seien $\kappa_1: \mathbb{N}_0^2 \to \mathbb{N}_0, (n_1,n_2)
\mapsto n_1$ und $\kappa_2: \mathbb{N}_0^2 \to \mathbb{N}_0, (n_1,n_2) \mapsto n_2$
die ebenfalls berechenbaren Projektionen.\\[4pt]
Betrachte nun die Funktion $g: \mathbb{N}_0 \to \mathbb{Q}$ mit\\
$g(x) := \left\{
\begin{array}{c@{\hspace{0.5cm},\hspace{0.5cm}}l}
\frac{\kappa_1(f(x))}{\kappa_2(f(x))} & \mbox{falls} \; 
0 < \kappa_1(f(x)) < \kappa_2(f(x)) \\
\frac{1}{2} & \mbox{sonst}
\end{array}
\right.$\\[4pt]
$g$ ist berechenbar und es gilt $\mbox{Bild}(g) = M_1$, also ist $M_1$
rekursiv aufz"ahlbar.
\item \underline{Annahme:} $M_2$ ist rekursiv aufz"ahlbar\\
Da $M_2 \not= \varnothing$ gilt, gibt es eine totale und berechenbare Funktion
$f: \mathbb{N}_0 \to \mathbb{R}$ mit $\mbox{Bild}(f) = M_2$.\\
Seien dann $f(0) = 0,a_{00}a_{01}... \; , \; f(1) = 0,a_{10}a_{11}... \; ,$ usw.,
wobei also $a_{ij}$ die Ziffer der Nachkommastelle mit dem Positionsindex $j$ zu
$f(i)$ bezeichnet.\\[4pt]
Konstruiere $b := 0,b_1b_2...$ mit\\
$b_i := \left\{
\begin{array}{c@{\hspace{0.5cm},\hspace{0.5cm}}l}
1 & \mbox{falls} \; a_{ii} \neq 1 \\
0 & \mbox{sonst}
\end{array}
\right.$\\[4pt]
Dann gilt $b \in M_2$, es muss also eine Zahl $n \in \mathbb{N}_0$ geben mit
$f(n) = b$. Daraus folgt:\\
$b_n = 1 \Leftrightarrow a_{nn} = b_n \neq 1$\\
Dies ist ein Widerpruch, damit ist $M_2$ nicht rekursiv aufz"ahlbar.
\end{enumerate}

\subsection*{L"osung zu Aufgabe 4}
Da die Funktion $x+2y^2+17+11^x$ streng monoton wachsend ist und zudem $A \subseteq
\mathbb{N}_0$ gilt, folgt somit:\\
$z \in B \Leftrightarrow \exists \; x,y \in A: 0 \leq x \leq z \wedge 0 \leq y \leq z
\wedge z = x+2y^2+17+11^x$\\
Da $A$ entscheidbar ist, ist damit ihre charakteristische Funktion $\chi_A$
berechenbar.\\[4pt]
Betrachte nun folgende Mehrband-Turingmschine, die auf dem ersten Band die Eingabe
$z$ enth"alt:\\
1. Initialisiere $x$ auf dem zweiten Band und $y$ auf dem dritten Band mit $0$\\
2. Berechne auf weiteren B"andern $\chi_A(x)$, $\chi_A(y)$ und $x+2y^2+17+11^x$\\
3. Pr"ufe, ob $\chi_A(x) = 1 \wedge \chi_A(y) = 1 \wedge x+2y^2+17+11^x = z$ gilt:\\
- Falls ja: Ersetze $z$ auf dem ersten Band durch eine $1$ und stoppe\\
- Falls nein: Gehe zu Schritt 4.\\
4. Erh"ohe $y$ um $1$ und pr"ufe, ob $y \leq z$ gilt:\\
- Falls ja: Gehe zu Schritt 2.\\
- Falls nein: Setze $y$ auf $0$ zur"uck und gehe zu Schritt 5.\\
5. Erh"ohe $x$ um $1$ und pr"ufe, ob $x \leq z$ gilt:\\
- Falls ja: Gehe zu Schritt 2.\\
- Falls nein: Ersetze $z$ auf dem ersten Band durch eine $0$ und stoppe\\[4pt]
Die Turingmaschine berechnet die charakteristische Funktion $\chi_B$ und ist wie
jede Mehrband-Turingmschine durch eine Einband-Turingmschine simulierbar. Wichtig
ist dabei, dass alle Einzelschritte berechenbar sind. Daf"ur wird die Berechenbarkeit
von $\chi_A$ ben"otigt. Zudem wird als bekannt vorausgesetzt, dass die diskrete
Arithmetik auf $\mathbb{N}_0$ und damit der Term $x+2y^2+17+11^x$ berechenbar ist.

\end{document}