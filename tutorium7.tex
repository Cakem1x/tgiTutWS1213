\documentclass[t]{beamer}
\usetheme[deutsch]{KIT}
\setbeamercovered{transparent}
\setbeamertemplate{navigation symbols}{}

\KITfoot{Tutoriumsmaterial von Joachim Priesner, Sebastian Ullrich und Max Wagner \hspace{2.5cm} Basierend auf den Folien von Simon Stroh und Moritz v. Looz}
\usepackage[utf8]{inputenc}
\usepackage{amsmath}
\usepackage{ifthen}
\usepackage{amssymb}
\usepackage{tikz}
\usepackage{ngerman}
\usepackage[normalem]{ulem}
\usetikzlibrary{automata}
\usenavigationsymbols


\title{Theoretische Grundlagen der Informatik}
\subtitle{Tutorium}
\author{Moritz von Looz, Simon Stroh}

\institute[ITI]{Institut für Theoretische Informatik}

\TitleImage[height=\titleimageht]{images/tmaschine.png}

\newcommand{\N}{\ensuremath{\mathbb{N}}}
\newcommand{\M}{\ensuremath{\mathcal{M}}}
\newcommand{\classP}{\ensuremath{\mathcal{P}}}
\newcommand{\classNP}{\ensuremath{\mathcal{NP}}}
\newcommand{\co}{\ensuremath{\mathsf{co\text{-}}}}
\newcommand{\pot}{\ensuremath{\mathcal{P}}}
\newcommand{\abs}[1]{\ensuremath{\left\vert #1 \right\vert}}
\newcommand{\menge}[2]{\ensuremath{\left\lbrace #1 \,\middle\vert\, #2 \right\rbrace}}
\newcommand{\ducttape}[1]{\vspace{#1}}
\newcommand{\neglit}[1]{\overline{#1\vphantom{x^a}}}
\newcommand{\recipe}{\raisebox{-.3cm}{\includegraphics[scale=.15]{images/chefs-cap.png}}\hspace{0.2cm}}
\newcommand{\opt}[1]{\ensuremath{\text{OPT}(#1)}}
\newcommand{\A}[1]{\ensuremath{\mathcal{A}(#1)}}
\renewcommand{\O}[1]{\ensuremath{\mathcal{O}(#1)}}
\newcommand{\msout}[1]{\text{\sout{\ensuremath{#1}}}}

\newcommand{\invincible}{\setbeamercovered{invisible}} %  "Yesss! I am invincible!!" (Boris Grishenko)
\newcommand{\vincible}{\setbeamercovered{transparent}}
\renewcommand{\solution}[1]{\invincible \pause #1 \vincible}
\newcommand{\micropause}{\\[8pt]}

% \@ifundefined{tikzset}{}{\tikzset{initial text=}} % Text "start" bei Startknoten unterdrücken
\tikzstyle{every node}=[thick]
\tikzstyle{every line}=[thick]

\newcommand{\tutnr}[1]{
  \subtitle{Tutorium #1}
	\begin{frame}
		\maketitle
	\end{frame}
}

\newcommand{\uebnr}[1]{
  \subtitle{Anmerkungen zum #1. Übungsblatt}
	\begin{frame}
		\maketitle
	\end{frame}
}

\begin{document}

\newcommand{\start}[3]
{
  \draw (#1*2,#2*2) node{$#3$};
  \draw (#1*2,#2*2) circle(0.4cm);
  \draw [->] (#1*2-0.9,#2) -- (#1*2-0.4,#2);
}
\newcommand{\final}[3]
{
  \draw (#1*2,#2*2) node{$#3$};
  \draw (#1*2,#2*2) circle(0.4cm);
  \draw (#1*2,#2*2) circle(0.32cm);
}
\newcommand{\startfinal}[3]
{
  \draw (#1*2,#2*2) node{$#3$};
  \draw (#1*2,#2*2) circle(0.4cm);
  \draw (#1*2,#2*2) circle(0.32cm);
  \draw [->] (#1*2-0.9,#2) -- (#1*2-0.4,#2);
}
\newcommand{\state}[3]
{
  \draw (#1*2,#2*2) node{$#3$};
  \draw (#1*2,#2*2) circle(0.4cm);
}
\newcommand{\tol}[4]
{
  \draw (#1+#3,#2*2) node[above]{$#4$};
  \draw [->] (#1*2-0.4,#2*2) -- (#3*2+0.4,#2*2);
}
\newcommand{\tor}[4]
{
  \draw (#1+#3,#2*2) node[above]{$#4$};
  \draw [->] (#1*2+0.4,#2*2) -- (#3*2-0.4,#2*2);
}
\newcommand{\tot}[4]
{
  \draw (#1*2,#2+#3) node[right]{$#4$};
  \draw [->] (#1*2,#2*2+0.4) -- (#1*2,#3*2-0.4);
}
\newcommand{\tob}[4]
{
  \draw (#1*2,#2+#3) node[right]{$#4$};
  \draw [->] (#1*2,#2*2-0.4) -- (#1*2,#3*2+0.4);
}
\newcommand{\totl}[5]
{
  \draw (#1+#3,#2+#4) node[above right]{$#5$};
  \draw [->] (#1*2-0.283,#2*2+0.283) -- (#3*2+0.283,#4*2-0.283);
}
\newcommand{\totr}[5]
{
  \draw (#1+#3,#2+#4) node[above left]{$#5$};
  \draw [->] (#1*2+0.283,#2*2+0.283) -- (#3*2-0.283,#4*2-0.283);
}
\newcommand{\tobl}[5]
{
  \draw (#1+#3,#2+#4) node[below right]{$#5$};
  \draw [->] (#1*2-0.283,#2*2-0.283) -- (#3*2+0.283,#4*2+0.283);
}
\newcommand{\tobr}[5]
{
  \draw (#1+#3,#2+#4) node[below left]{$#5$};
  \draw [->] (#1*2+0.283,#2*2-0.283) -- (#3*2-0.283,#4*2+0.283);
}
\newcommand{\rloopl}[3]
{
  \draw (#1*2-1,#2*2) node[left]{$#3$};
  \draw [->] (#1*2-0.35,#2*2-0.2) arc (-30:-320:0.32cm);
}
\newcommand{\rloopr}[3]
{
  \draw (#1*2+1,#2*2) node[right]{$#3$};
  \draw [->] (#1*2+0.35,#2*2+0.2) arc (150:-140:0.32cm);
}
\newcommand{\rloopt}[3]
{
  \draw (#1*2,#2*2+1) node[above]{$#3$};
  \draw [->] (#1*2-0.2,#2*2+0.35) arc (240:-50:0.32cm);
}
\newcommand{\rloopb}[3]
{
  \draw (#1*2,#2*2-1) node[below]{$#3$};
  \draw [->] (#1*2+0.2,#2*2-0.35) arc (60:-230:0.32cm);
}
\newcommand{\lloopl}[3]
{
  \draw (#1*2-1,#2*2) node[left]{$#3$};
  \draw [->] (#1*2-0.35,#2*2+0.2) arc (30:320:0.32cm);
}
\newcommand{\lloopr}[3]
{
  \draw (#1*2+1,#2*2) node[right]{$#3$};
  \draw [->] (#1*2+0.35,#2*2-0.2) arc (-150:140:0.32cm);
}
\newcommand{\lloopt}[3]
{
  \draw (#1*2,#2*2+1) node[above]{$#3$};
  \draw [->] (#1*2+0.2,#2*2+0.35) arc (-60:230:0.32cm);
}
\newcommand{\lloopb}[3]
{
  \draw (#1*2,#2*2-1) node[below]{$#3$};
  \draw [->] (#1*2-0.2,#2*2-0.35) arc (-240:50:0.32cm);
}
\include{amsmath}
\tutnr{7}

\section{Kolmogorow Komplexität}
\subsection{Kolmogorow Komplexität erklären}
\subsection{Kolmogorow Aufgabe}
\begin{frame}
	\frametitle{Kolmogorow Komplexität: Aufgabe (B7 A1)}
	\begin{enumerate}
		\item Beweisen Sie, dass $K(x)$ nicht berechenbar ist!
		\item Beweisen Sie, dass die Menge der nichtkomprimierbaren Strings $\mathcal{L}$
		nicht rekursiv aufz"ahlbar ist!
		\item Geben Sie eine m"oglichst gute obere Schranke f"ur die Kolmogorow-Komplexit"at von $0^n$ an!
		\item Geben Sie eine m"oglichst gute obere Schranke f"ur die Kolmogorow-Komplexit"at der
		Bin"ardarstellung der $n$-ten\\ Primzahl $p$ an!
		\item Sei $x$ ein Palindrom. Geben sie eine m"oglichst gute obere Schranke f"ur $K(x)$ an!
		\item Sei $\pi_n$ die Kreiszahl $\pi$ bis zur $n$-ten Nachkommastelle entwickelt. Geben Sie eine m"oglichst gute obere Schranke f"ur $\pi_n$ an.
	\end{enumerate}
\end{frame}

\section{Rekursionstheorem}
\subsection{Rekursionstheorem erklären}

\section{SELF-Maschine / Quines}
\subsection{SELF-Maschine / Quines erklären}

\section{Wiederholung einiger Begriffe}
\subsection{Wiederholung einiger Begriffe}
\begin{frame}
	\frametitle{Wiederholung einiger Begriffe}
	\begin{itemize}
		\item Quantoren
		\begin{itemize}
			\item Existenzquantor $\exists x$: \\ Aussage muss für mindestens ein x aus dem Universum gelten.
			\item Allquantor $\forall x$: Aussage muss für alle x aus dem Universum gelten.
			\item Vorsicht bei Schachtelung von Quantoren: \\ $\forall x \exists y: x = y$ ist etwas völlig anderes als $\exists y \forall x: x = y$.
		\end{itemize}
		\item Ein Universum ist die Menge über der man eine Aussage betrachtet.
		\item Eine Relation drückt aus, dass zwei Objekte zueinander in Beziehung stehen.
		\begin{itemize}
			\item Sei $R$ die Gleichheit, dann gilt $R(x, y) \Leftrightarrow x = y$.
		\end{itemize}
		\item Eine Theorie ist eine Menge $Th(U, R)$ induziert über dem Tupel $(U, R)$ mit einem Universum $U$ und einer Relation $R$. \\ 
		Eine Formel $\phi$ ist Element einer Theorie, falls sie in Bezug auf $U$ bzw. $R$ wahr ist.
		\begin{itemize}
			\item Sei $\phi = \forall x \exists y: R_1(x,y)$. Dann gilt $\phi \in Th(\mathbb{Z}, >)$ aber $\phi \notin Th(\mathbb{N}, >)$.
		\end{itemize}
	\end{itemize}
\end{frame}

\section{Weitere Aufgaben}
\subsection{B7 A2}
\begin{frame}
	\frametitle{Weitere Aufgaben: B7 A2}
	Geben Sie f"ur folgendende Formeln an ob diese in den besagten Theorien liegen
	\begin{enumerate}
		\item Ist $\phi_1 = \forall x \exists y \forall z: x + y = z$ in $\text{Th}(\mathbb{N,+})$?
		\item Ist $\phi_2 = \forall x \exists y \forall z \exists w: (x + z = w ) \wedge (x + y = w)$  in $\text{Th}(\mathbb{N},+)$?
		\item Ist $\phi_3 = \forall x \forall y \forall z \forall w \forall v \exists s: \neg(x + w = y) \vee \neg(y + v = z) \vee (x + s = z)$ in $\text{Th}(\mathbb{N},+)$?
		\item Sei $\text{Th}(\mathbb{N},<)$ die Theorie der nat"urlichen Zahlen mit der Relation "`echt kleiner"'. Zeigen Sie: $\text{Th}(\mathbb{N},<)$ ist entscheidbar.
	\end{enumerate}
\end{frame}
\subsection{B7 A3}
\begin{frame}
	\frametitle{Weitere Aufgaben: B7 A3}
	Geben Sie Modelle f"ur die folgenden pr"adikatenlogischen Formeln an! Geben Sie dazu
	jeweils ein Universum $\mathcal{U}$\\
	und eine Interpretation der Relationszeichen $R_i$ an!
	\begin{enumerate}
		\item $\phi_1 =$ \hspace*{0.2cm} $\forall \; x \; (R_1(x,x))$ \hspace*{5.3cm}[K1.1]\\
		\hspace*{0.85cm} $\wedge \forall \; x,y \; (R_1(x,y) \leftrightarrow R_1(y,x))$
		\hspace*{3.2cm} [K1.2]\\
		\hspace*{0.85cm} $\wedge \forall \; x,y,z \; ((R_1(x,y) \wedge R_1(y,z)) \rightarrow
		R_1(x,z))$ \hspace*{1cm} [K1.3]
		\item $\phi_2 =$ \hspace*{0.2cm} $\phi_1$\\
		\hspace*{0.85cm} $\wedge \forall \; x \; (R_1(x,x) \rightarrow \neg R_2(x,x))$
		\hspace*{3.25cm} [K2.1]\\
		\hspace*{0.85cm} $\wedge \forall \; x,y \; (\neg R_1(x,y) \rightarrow (R_2(x,y)
		\oplus R_2(y,x)))$ \hspace*{1cm} [K2.2]\\
		\hspace*{0.85cm} $\wedge \forall \; x,y,z \; ((R_2(x,y) \wedge R_2(y,z)) \rightarrow
		R_2(x,z))$ \hspace*{1cm}  [K2.3]\\
		\hspace*{0.85cm} $\wedge \forall \; x \; \exists \; y \, (R_2(x,y))$ \hspace*{4.8cm}
		[K2.4]
	\end{enumerate}
\end{frame}
\subsection{B6 A4}
\begin{frame}
	\frametitle{Weitere Aufgaben: B6 A4}
	Sei $A \subseteq \mathbb{N}_0$ eine entscheidbare Menge. Zeigen Sie, dass
	$ B := \{x+2y^2+17+11^x \; | \; x,y \in A\}$ entscheidbar ist!
\end{frame}

\section{Schluss}
\subsection{Schluss}
\begin{frame}
\frametitle{Bis zum nächsten Mal!}
\begin{center}
%TODO Neuer Comic!
  \includegraphics[width=1.59 \textheight]{images/halting.jpg}
\end{center}
\end{frame}

\input{includes/disclaimer}
\end{document}