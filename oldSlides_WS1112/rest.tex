\documentclass[t]{beamer}
\usetheme[deutsch]{KIT}
\setbeamercovered{transparent}
\setbeamertemplate{navigation symbols}{}

\KITfoot{Tutoriumsmaterial von Joachim Priesner, Sebastian Ullrich und Max Wagner \hspace{2.5cm} Basierend auf den Folien von Simon Stroh und Moritz v. Looz}
\usepackage[utf8]{inputenc}
\usepackage{amsmath}
\usepackage{ifthen}
\usepackage{amssymb}
\usepackage{tikz}
\usepackage{ngerman}
\usepackage[normalem]{ulem}
\usetikzlibrary{automata}
\usenavigationsymbols


\title{Theoretische Grundlagen der Informatik}
\subtitle{Tutorium}
\author{Moritz von Looz, Simon Stroh}

\institute[ITI]{Institut für Theoretische Informatik}

\TitleImage[height=\titleimageht]{images/tmaschine.png}

\newcommand{\N}{\ensuremath{\mathbb{N}}}
\newcommand{\M}{\ensuremath{\mathcal{M}}}
\newcommand{\classP}{\ensuremath{\mathcal{P}}}
\newcommand{\classNP}{\ensuremath{\mathcal{NP}}}
\newcommand{\co}{\ensuremath{\mathsf{co\text{-}}}}
\newcommand{\pot}{\ensuremath{\mathcal{P}}}
\newcommand{\abs}[1]{\ensuremath{\left\vert #1 \right\vert}}
\newcommand{\menge}[2]{\ensuremath{\left\lbrace #1 \,\middle\vert\, #2 \right\rbrace}}
\newcommand{\ducttape}[1]{\vspace{#1}}
\newcommand{\neglit}[1]{\overline{#1\vphantom{x^a}}}
\newcommand{\recipe}{\raisebox{-.3cm}{\includegraphics[scale=.15]{images/chefs-cap.png}}\hspace{0.2cm}}
\newcommand{\opt}[1]{\ensuremath{\text{OPT}(#1)}}
\newcommand{\A}[1]{\ensuremath{\mathcal{A}(#1)}}
\renewcommand{\O}[1]{\ensuremath{\mathcal{O}(#1)}}
\newcommand{\msout}[1]{\text{\sout{\ensuremath{#1}}}}

\newcommand{\invincible}{\setbeamercovered{invisible}} %  "Yesss! I am invincible!!" (Boris Grishenko)
\newcommand{\vincible}{\setbeamercovered{transparent}}
\renewcommand{\solution}[1]{\invincible \pause #1 \vincible}
\newcommand{\micropause}{\\[8pt]}

% \@ifundefined{tikzset}{}{\tikzset{initial text=}} % Text "start" bei Startknoten unterdrücken
\tikzstyle{every node}=[thick]
\tikzstyle{every line}=[thick]

\newcommand{\tutnr}[1]{
  \subtitle{Tutorium #1}
	\begin{frame}
		\maketitle
	\end{frame}
}

\newcommand{\uebnr}[1]{
  \subtitle{Anmerkungen zum #1. Übungsblatt}
	\begin{frame}
		\maketitle
	\end{frame}
}

\begin{document}


\section{13. Tutoriumsvorschlag}
\begin{frame}
Geben Sie einen zu folgendem NEA äquivalenten DEA an:

\begin{center}
 \includegraphics[width=5cm]{NEA}
\end{center}
\end{frame}

\begin{frame}
Beweisen Sie die $\mathcal{NP}$-Vollständigkeit des Problems HITTING SET:
\begin{quote}
  Gegeben eine Menge $M$ und eine Menge $T$ von Teilmengen von $M$,
  $K \in \mathbb{N}$. Gibt es eine Teilmenge $M' \subseteq M$ mit $|M'| \leq K$ so,
  dass $M'$ mindestens ein Element jeder Teilmenge $t \in T$ enthält?
\end{quote}
\textbf{Lösungshinweis:}
 Reduziere von VERTEX COVER (s. korrekte Variante auf Tutvorschlag 7 ;-) ).
\end{frame}

\begin{frame}
Zeigen Sie, dass es keinen absoluten Approximationsalgorithmus für die Optimierungsvariante von
INDEPENDENT SET gibt, falls $\mathcal{P} \neq \mathcal{NP}$.\footnote{vgl. Tutoriumsvorschlag 7}
\textbf{Lösungsskizze:}
\begin{itemize}
 \item Nimm an, es gäbe einen abs. Approx-Algo APX (mit konstanter Gütegarantie $k$) für INDEPENDENT SET und zeige, dass man damit INDEPENDENT SET auch optimal in Polynomialzeit lösen könnte
 \item Um eine Instanz $G$ von INDEPENDENT SET zu lösen, konstruiere einen Graphen $G'$, der aus $k+1$ Kopien von $G$ besteht
 \item Benutze APX, um in $G'$ ein INDEPENDENT SET $I$ zu finden und betrachte den Teilgraphen in $G'$, der die meisten Knoten aus $I$ enthält
 \item Zeige, dass $I$ eingeschränkt auf diesen Teilgraphen ein INDEPENDENT SET maximaler Größe ist. 
\end{itemize}
\end{frame}

\begin{frame}
Bringen Sie die folgende Grammatik in CNF:

$$S \rightarrow ASA \mid aB, \quad A \rightarrow B \mid S, \quad
B \rightarrow b \mid \varepsilon.$$
\end{frame}
\input{includes/disclaimer}
\end{document}
