\documentclass[t]{beamer}
\usetheme[deutsch]{KIT}
\setbeamercovered{transparent}
\setbeamertemplate{navigation symbols}{}

\KITfoot{Tutoriumsmaterial von Joachim Priesner, Sebastian Ullrich und Max Wagner \hspace{2.5cm} Basierend auf den Folien von Simon Stroh und Moritz v. Looz}
\usepackage[utf8]{inputenc}
\usepackage{amsmath}
\usepackage{ifthen}
\usepackage{amssymb}
\usepackage{tikz}
\usepackage{ngerman}
\usepackage[normalem]{ulem}
\usetikzlibrary{automata}
\usenavigationsymbols


\title{Theoretische Grundlagen der Informatik}
\subtitle{Tutorium}
\author{Moritz von Looz, Simon Stroh}

\institute[ITI]{Institut für Theoretische Informatik}

\TitleImage[height=\titleimageht]{images/tmaschine.png}

\newcommand{\N}{\ensuremath{\mathbb{N}}}
\newcommand{\M}{\ensuremath{\mathcal{M}}}
\newcommand{\classP}{\ensuremath{\mathcal{P}}}
\newcommand{\classNP}{\ensuremath{\mathcal{NP}}}
\newcommand{\co}{\ensuremath{\mathsf{co\text{-}}}}
\newcommand{\pot}{\ensuremath{\mathcal{P}}}
\newcommand{\abs}[1]{\ensuremath{\left\vert #1 \right\vert}}
\newcommand{\menge}[2]{\ensuremath{\left\lbrace #1 \,\middle\vert\, #2 \right\rbrace}}
\newcommand{\ducttape}[1]{\vspace{#1}}
\newcommand{\neglit}[1]{\overline{#1\vphantom{x^a}}}
\newcommand{\recipe}{\raisebox{-.3cm}{\includegraphics[scale=.15]{images/chefs-cap.png}}\hspace{0.2cm}}
\newcommand{\opt}[1]{\ensuremath{\text{OPT}(#1)}}
\newcommand{\A}[1]{\ensuremath{\mathcal{A}(#1)}}
\renewcommand{\O}[1]{\ensuremath{\mathcal{O}(#1)}}
\newcommand{\msout}[1]{\text{\sout{\ensuremath{#1}}}}

\newcommand{\invincible}{\setbeamercovered{invisible}} %  "Yesss! I am invincible!!" (Boris Grishenko)
\newcommand{\vincible}{\setbeamercovered{transparent}}
\renewcommand{\solution}[1]{\invincible \pause #1 \vincible}
\newcommand{\micropause}{\\[8pt]}

% \@ifundefined{tikzset}{}{\tikzset{initial text=}} % Text "start" bei Startknoten unterdrücken
\tikzstyle{every node}=[thick]
\tikzstyle{every line}=[thick]

\newcommand{\tutnr}[1]{
  \subtitle{Tutorium #1}
	\begin{frame}
		\maketitle
	\end{frame}
}

\newcommand{\uebnr}[1]{
  \subtitle{Anmerkungen zum #1. Übungsblatt}
	\begin{frame}
		\maketitle
	\end{frame}
}

\begin{document}

\uebnr{1}

\begin{frame}
	\begin{center}
	\includegraphics[height=0.85 \textheight]{images/gutti.jpg}
	\end{center}
	
	\textcolor{gray}{\tiny{http://de.wikipedia.org/w/index.php?title=Datei:Guttenberg-800.jpg}}
\end{frame}

\begin{frame}
	\frametitle{Häufige Fehler}		
		\begin{itemize}
			\item Entgegen Gerüchten der Übungsleiter schreibt man ''Ullrich'' mit zwei l :) .\pause
            \item Aufgabe 1: Mengengleichheit und Aussagenäquivalenz nicht verwechseln \pause
            \item Aufgabe 2: Nur Sprachen, nicht Wörter, können regulär sein.
            \item Aufgabe 2: Reguläre Ausdrücke kennen nur Vereinigung und Konkatenation, keinen Schnitt. \pause
            \item Aufgabe 3: Wenn systematische Durcharbeitung eines Verfahrens verlangt ist, führt kein Weg vorbei. \pause
            \item Aufgabe 4: $q$ liegt immer in $E(q)$.
            \item Aufgabe 4: bei $\epsilon$-Entfernung Schleifen und akzeptierende Zustände nicht vergessen
        \end{itemize}
\end{frame}
\begin{frame}
	\frametitle{Aufgabe 7: Allgemeines Schema}
	
	\begin{itemize}
        \item Annahme: L ist regulär.
		\item Sei $n$ dann wie im Pumping-Lemma \emph{(also innerhalb des Beweises beliebig!)}.
		\item Wähle $w=\text{\textbf{ passendes Wort } , sodass } w\in L, \abs{w} > n$
		\item Sei nun $w=uvx$ eine \emph{beliebige} Zerlegung mit $\abs{uv} \le n, v\neq \epsilon$
        \item $[$Dann haben $u, v, x$ die Form ... $]$
        \item Für $i=\textbf{eineZahl}$ ist $uv^ix \not\in L$, im Widerspruch zum Pumping-Lemma.
        \item Also ist L nicht regulär.
        \vspace{1cm}
        \item Manchmal muss $i$ je nach Zerlegung unterschiedlich gewählt werden \\ (auf dem Übungsblatt allerdings nicht).
	\end{itemize}
\end{frame}

\input{includes/disclaimer}
\end{document}
