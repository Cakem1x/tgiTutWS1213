\documentclass[t]{beamer}
\usetheme[deutsch]{KIT}
\setbeamercovered{transparent}
\setbeamertemplate{navigation symbols}{}

\KITfoot{Tutoriumsmaterial von Joachim Priesner, Sebastian Ullrich und Max Wagner \hspace{2.5cm} Basierend auf den Folien von Simon Stroh und Moritz v. Looz}
\usepackage[utf8]{inputenc}
\usepackage{amsmath}
\usepackage{ifthen}
\usepackage{amssymb}
\usepackage{tikz}
\usepackage{ngerman}
\usepackage[normalem]{ulem}
\usetikzlibrary{automata}
\usenavigationsymbols


\title{Theoretische Grundlagen der Informatik}
\subtitle{Tutorium}
\author{Moritz von Looz, Simon Stroh}

\institute[ITI]{Institut für Theoretische Informatik}

\TitleImage[height=\titleimageht]{images/tmaschine.png}

\newcommand{\N}{\ensuremath{\mathbb{N}}}
\newcommand{\M}{\ensuremath{\mathcal{M}}}
\newcommand{\classP}{\ensuremath{\mathcal{P}}}
\newcommand{\classNP}{\ensuremath{\mathcal{NP}}}
\newcommand{\co}{\ensuremath{\mathsf{co\text{-}}}}
\newcommand{\pot}{\ensuremath{\mathcal{P}}}
\newcommand{\abs}[1]{\ensuremath{\left\vert #1 \right\vert}}
\newcommand{\menge}[2]{\ensuremath{\left\lbrace #1 \,\middle\vert\, #2 \right\rbrace}}
\newcommand{\ducttape}[1]{\vspace{#1}}
\newcommand{\neglit}[1]{\overline{#1\vphantom{x^a}}}
\newcommand{\recipe}{\raisebox{-.3cm}{\includegraphics[scale=.15]{images/chefs-cap.png}}\hspace{0.2cm}}
\newcommand{\opt}[1]{\ensuremath{\text{OPT}(#1)}}
\newcommand{\A}[1]{\ensuremath{\mathcal{A}(#1)}}
\renewcommand{\O}[1]{\ensuremath{\mathcal{O}(#1)}}
\newcommand{\msout}[1]{\text{\sout{\ensuremath{#1}}}}

\newcommand{\invincible}{\setbeamercovered{invisible}} %  "Yesss! I am invincible!!" (Boris Grishenko)
\newcommand{\vincible}{\setbeamercovered{transparent}}
\renewcommand{\solution}[1]{\invincible \pause #1 \vincible}
\newcommand{\micropause}{\\[8pt]}

% \@ifundefined{tikzset}{}{\tikzset{initial text=}} % Text "start" bei Startknoten unterdrücken
\tikzstyle{every node}=[thick]
\tikzstyle{every line}=[thick]

\newcommand{\tutnr}[1]{
  \subtitle{Tutorium #1}
	\begin{frame}
		\maketitle
	\end{frame}
}

\newcommand{\uebnr}[1]{
  \subtitle{Anmerkungen zum #1. Übungsblatt}
	\begin{frame}
		\maketitle
	\end{frame}
}

\begin{document}

\tutnr{10}

\section{Chomsky-Hierarchie}
\subsection*{derp}
\frame{
\frametitle{Grammatiken}
\begin{block}{Definition}
 Eine Grammatik ist ein Regelsystem, mit dem sich die Wörter einer Sprache erzeugen lassen. Sie besteht aus vier Komponenten:
 \begin{itemize}
  \item ein endliches \textbf{Alphabet} $\Sigma$ (auch Terminale genannt)
  \item eine endliche Menge $V$ mit $V \cap \Sigma = \emptyset$ von \textbf{Variablen} (auch Nichtterminale genannt)
  \item ein \textbf{Startsymbol} $S \in V$
  \item eine endliche Menge von \textbf{Ableitungsregeln} $R$ (auch Produktionen genannt).
  Eine Ableitungsregel ist ein Paar ($l,r)$, wobei $l \in (V \cup \Sigma)^+$ und $r \in (V \cup \Sigma)^*$ ist.
  Wir schreiben meist $l \rightarrow r$.
 \end{itemize}
\end{block}
}

\frame{
\frametitle{Beispiel}
\begin{exampleblock}{Beispiel: Sprache der Palindrome}
\begin{align*}
V = &\{S\}\\
\Sigma = &\{0,1\}\\
R = &\{ S \rightarrow \varepsilon \mid 0 \mid 1,\\
       &S \rightarrow 0S0 \mid 1S1 \} 
\end{align*}
\end{exampleblock}
}

\frame{
\frametitle{Chomsky-Hierarchie}
Von Noam Chomsky definiert, um natürliche Sprachen formalisieren zu können. \\
Wir definieren Klassen von Sprachen, die von Grammatiken mit bestimmten Einschränkungen erzeugt werden:
\begin{itemize}
\item CH-3 Regulär/Rechtslinear
\item CH-2 Kontextfrei
\item CH-1 Kontextsensitiv/Längenbeschränkt
\item CH-0 Beliebig \only<2>{(rekursiv aufzählbar)}
\end{itemize}
Es gilt: $$CH\text{-}3 \subset CH\text{-}2 \subset CH\text{-}1 \subset CH\text{-}0$$
Frage: Welche Sprachen liegen nicht in CH-0 ?
\pause
}

\frame{
\frametitle{CH-0}
Rekursiv aufzählbare Sprachen
\begin{itemize}
\item Sprachen, die von einer beliebigen Grammatik erzeugt werden
\item Genau die Sprachen, die DTM \& NTM \emph{akzeptieren} können
\end{itemize}

}

\frame{
\frametitle{CH-1}
Kontextsensitive/längenbeschränkte Sprachen
\begin{itemize}
\item Sprachen, die von Grammatiken erzeugt werden, deren Ableitungen eine linke Seite besitzen, die höchstens so lang ist wie ihre rechte Seite. D.h. Produktionen machen das Wort nie kürzer.\\
$\Rightarrow$ entscheidbar
\item (Genau die Sprachen, die von linear beschränken Turingmaschinen erkannt werden)
\pause
\item $\rightarrow$ vgl. Klasse $\mathcal{NTAPE}$(n)
\end{itemize}
\pause
~\micropause
\begin{block}{Beispiel}
\begin{itemize}
\item $\{ a^nb^nc^n \mid n \in \N \}$
\item Syntaktisch korrektes C (typedef)
\item Die Sprache der wohlgeformten Latex-Dokumente
\end{itemize}
\end{block}
}

\frame{
\frametitle{CH-2}
Kontextfreie Sprachen
\begin{itemize}
\item Grammatiken, deren Ableitungsregeln auf der linken Seite immer aus genau einem Nichtterminalsymbol bestehen
\item Genau die Sprachen, die von Kellerautomaten erkannt werden (Vorgriff)
\end{itemize}
~\micropause
\begin{block}{Beispiele}
\begin{itemize}
\item $\{ a^nb^n \mid n \in \N \}$
\item Viele natürlichen Sprachen
\item Die Sprache der gültigen Klammerausdrücke
\item Die Sprache der regulären Ausdrücke
\end{itemize}
\end{block}
}

\frame{
\frametitle{CH-3}
Reguläre Sprachen
\begin{itemize}
\item Grammatiken, deren Ableitungsregeln ausschließlich die folgende Form haben:
$$A\rightarrow v \mbox{ mit } A \in V \mbox{ und } v = \varepsilon \mbox{ oder } v = aB \mbox{ mit } a \in \Sigma \mbox{,} B \in V$$
\item Genau die Sprachen, die von endlichen Automaten erkannt werden
\end{itemize}
~\micropause
\begin{block}{Beispiele}
\begin{itemize}
\item Alle endlichen Sprachen
\item Tokens ($\rightarrow$ vgl. Compilerbau)
\item Die Sprache der kontextfreien Grammatiken über festem Alphabet
\item Perl Regexes \emph{nicht}
\end{itemize}
\end{block}
}

\section{Aufgaben}
\subsection*{seufz}

\begin{frame}
\frametitle{Aufgabe}
Gegeben sei die Grammatik $G = (\Sigma, V, S, R)$ mit $\Sigma = \{a, b\}$,
$V = \{S, A\}$ und 
\[ R = \{S \rightarrow AA, \quad A \rightarrow AAA,
\quad A \rightarrow bA, \quad A \rightarrow Ab, \quad A \rightarrow a\}. \]

\begin{itemize}
  \item Welchen Typ in der Chomsky-Hierarchie hat $G$?

  \item Welche Zeichenketten aus $\Sigma^*$ k\"onnen in vier Schritten abgeleitet werden?

  \item Sei $\alpha = (b^*ab^*ab^*)^+$ ein regul\"arer Ausdruck.
        Zeige: $L(\alpha) \subseteq L(G)$.
\end{itemize}
\end{frame}


\begin{frame}
\frametitle{Aufgabe: CH-1-Grammatiken}
Konstruiere eine Typ-1-Grammatik, welche die Sprache

\begin{quote}
  $L = \{a^n b^n c^n \, | \, n \geq 1\}$
\end{quote}

erzeugt.

\invincible
\pause
\begin{block}{Lösung}
\ducttape{-1cm}
\begin{align*}
&\Sigma = \{a, b, c\} \\
&V = \{ S, S', \#, B, C \} \\
&S \rightarrow S' \\
&S' \rightarrow aS'\#C \mid aBC \\
&C\# \rightarrow \#C \\
&B\# \rightarrow BB \\
&B \rightarrow b \\
&C \rightarrow c
\end{align*}
\end{block}
\vincible
\end{frame}

\begin{frame}
\frametitle{Aufgabe: CH-2-Grammatiken}
Gib eine kontextfreie Grammatik an, die die Sprache
\begin{quote}
  $L = \{w \in \Sigma^* \, | \, w \text{ enth\"alt mehr Nullen als Einsen}\}$  
\end{quote}
\"uber dem Alphabet $\Sigma = \{0, 1\}$ beschreibt.

\invincible
\pause
\begin{block}{Lösung}
\begin{align*}
&\Sigma = \{ 0, 1 \} \\
&V = \{ S, \# \} \\
&S \rightarrow \#0\# \\
&\# \rightarrow 0\#1 \mid 1\#0 \mid \#\# \mid 0\# \mid \varepsilon
\end{align*}
\end{block}
\vincible
\end{frame}


\begin{frame}
\frametitle{Aufgabe: CH-3-Grammatiken}
Gib unter Anwendung des Verfahrens aus Satz 5.5 eine Grammatik an,
welche genau die Sprache erzeugt, die folgender DEA akzeptiert:

\begin{center}\includegraphics[scale=0.8]{./images/dea1.pdf}\end{center}
\end{frame}


\begin{frame}
\frametitle{Aufgabe}
\begin{enumerate}
\item Bestimme eine möglichst kurze Grammatik $G_R$ für reguläre Ausdrücke über dem Alphabet $\Sigma = \{0,1\}$ mit maximalem Chomsky-Typ.
\item Bestimme einen Ableitungsbaum für das Wort $1^* \cup (01)^*$. 
\item Ist die Grammatik eindeutig? Diskutiere, ob es eine eindeutige Grammatik für reguläre Ausdrücke geben kann, falls die angegebene Grammatik mehrdeutig ist.
\end{enumerate}

\invincible
\pause
\begin{block}{Lösung}
\ducttape{-1cm}
\begin{align*}
&\Sigma = \{ 0, 1, \vphantom{0}^\ast, \cup, \epsilon, \emptyset, (, ) \} \\
&V = \{ S, P, K \} \\
\\
&S \rightarrow P \cup S \mid P \\
&P \rightarrow KP \mid K \\
&K \rightarrow 0 \mid 1 \mid \epsilon \mid \emptyset \mid T^\ast \mid (S)
\end{align*}
\end{block}
\vincible
\end{frame}

\frame{
\frametitle{Probleme}
Es gibt einige interessante Probleme, mit deren Entscheidbarkeit, Komplexität bzw. Abgeschlossenheit wir uns noch befassen (jeweils für die Klassen)
\begin{itemize}
\item Das Wortproblem
\item Vereinigung
\item Schnitt
\item Komplement
\item Konkatenation
\item Kleenescher Abschluss (*-Operator)
\end{itemize}
}

\input{includes/disclaimer}
\end{document}
