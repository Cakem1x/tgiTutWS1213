\documentclass[t]{beamer}
\usetheme[deutsch]{KIT}
\setbeamercovered{transparent}
\setbeamertemplate{navigation symbols}{}

\KITfoot{Tutoriumsmaterial von Joachim Priesner, Sebastian Ullrich und Max Wagner \hspace{2.5cm} Basierend auf den Folien von Simon Stroh und Moritz v. Looz}
\usepackage[utf8]{inputenc}
\usepackage{amsmath}
\usepackage{ifthen}
\usepackage{amssymb}
\usepackage{tikz}
\usepackage{ngerman}
\usepackage[normalem]{ulem}
\usetikzlibrary{automata}
\usenavigationsymbols


\title{Theoretische Grundlagen der Informatik}
\subtitle{Tutorium}
\author{Moritz von Looz, Simon Stroh}

\institute[ITI]{Institut für Theoretische Informatik}

\TitleImage[height=\titleimageht]{images/tmaschine.png}

\newcommand{\N}{\ensuremath{\mathbb{N}}}
\newcommand{\M}{\ensuremath{\mathcal{M}}}
\newcommand{\classP}{\ensuremath{\mathcal{P}}}
\newcommand{\classNP}{\ensuremath{\mathcal{NP}}}
\newcommand{\co}{\ensuremath{\mathsf{co\text{-}}}}
\newcommand{\pot}{\ensuremath{\mathcal{P}}}
\newcommand{\abs}[1]{\ensuremath{\left\vert #1 \right\vert}}
\newcommand{\menge}[2]{\ensuremath{\left\lbrace #1 \,\middle\vert\, #2 \right\rbrace}}
\newcommand{\ducttape}[1]{\vspace{#1}}
\newcommand{\neglit}[1]{\overline{#1\vphantom{x^a}}}
\newcommand{\recipe}{\raisebox{-.3cm}{\includegraphics[scale=.15]{images/chefs-cap.png}}\hspace{0.2cm}}
\newcommand{\opt}[1]{\ensuremath{\text{OPT}(#1)}}
\newcommand{\A}[1]{\ensuremath{\mathcal{A}(#1)}}
\renewcommand{\O}[1]{\ensuremath{\mathcal{O}(#1)}}
\newcommand{\msout}[1]{\text{\sout{\ensuremath{#1}}}}

\newcommand{\invincible}{\setbeamercovered{invisible}} %  "Yesss! I am invincible!!" (Boris Grishenko)
\newcommand{\vincible}{\setbeamercovered{transparent}}
\renewcommand{\solution}[1]{\invincible \pause #1 \vincible}
\newcommand{\micropause}{\\[8pt]}

% \@ifundefined{tikzset}{}{\tikzset{initial text=}} % Text "start" bei Startknoten unterdrücken
\tikzstyle{every node}=[thick]
\tikzstyle{every line}=[thick]

\newcommand{\tutnr}[1]{
  \subtitle{Tutorium #1}
	\begin{frame}
		\maketitle
	\end{frame}
}

\newcommand{\uebnr}[1]{
  \subtitle{Anmerkungen zum #1. Übungsblatt}
	\begin{frame}
		\maketitle
	\end{frame}
}

\begin{document}

\newcommand{\start}[3]
{
  \draw (#1*2,#2*2) node{$#3$};
  \draw (#1*2,#2*2) circle(0.4cm);
  \draw [->] (#1*2-0.9,#2) -- (#1*2-0.4,#2);
}
\newcommand{\final}[3]
{
  \draw (#1*2,#2*2) node{$#3$};
  \draw (#1*2,#2*2) circle(0.4cm);
  \draw (#1*2,#2*2) circle(0.32cm);
}
\newcommand{\startfinal}[3]
{
  \draw (#1*2,#2*2) node{$#3$};
  \draw (#1*2,#2*2) circle(0.4cm);
  \draw (#1*2,#2*2) circle(0.32cm);
  \draw [->] (#1*2-0.9,#2) -- (#1*2-0.4,#2);
}
\newcommand{\state}[3]
{
  \draw (#1*2,#2*2) node{$#3$};
  \draw (#1*2,#2*2) circle(0.4cm);
}
\newcommand{\tol}[4]
{
  \draw (#1+#3,#2*2) node[above]{$#4$};
  \draw [->] (#1*2-0.4,#2*2) -- (#3*2+0.4,#2*2);
}
\newcommand{\tor}[4]
{
  \draw (#1+#3,#2*2) node[above]{$#4$};
  \draw [->] (#1*2+0.4,#2*2) -- (#3*2-0.4,#2*2);
}
\newcommand{\tot}[4]
{
  \draw (#1*2,#2+#3) node[right]{$#4$};
  \draw [->] (#1*2,#2*2+0.4) -- (#1*2,#3*2-0.4);
}
\newcommand{\tob}[4]
{
  \draw (#1*2,#2+#3) node[right]{$#4$};
  \draw [->] (#1*2,#2*2-0.4) -- (#1*2,#3*2+0.4);
}
\newcommand{\totl}[5]
{
  \draw (#1+#3,#2+#4) node[above right]{$#5$};
  \draw [->] (#1*2-0.283,#2*2+0.283) -- (#3*2+0.283,#4*2-0.283);
}
\newcommand{\totr}[5]
{
  \draw (#1+#3,#2+#4) node[above left]{$#5$};
  \draw [->] (#1*2+0.283,#2*2+0.283) -- (#3*2-0.283,#4*2-0.283);
}
\newcommand{\tobl}[5]
{
  \draw (#1+#3,#2+#4) node[below right]{$#5$};
  \draw [->] (#1*2-0.283,#2*2-0.283) -- (#3*2+0.283,#4*2+0.283);
}
\newcommand{\tobr}[5]
{
  \draw (#1+#3,#2+#4) node[below left]{$#5$};
  \draw [->] (#1*2+0.283,#2*2-0.283) -- (#3*2-0.283,#4*2+0.283);
}
\newcommand{\rloopl}[3]
{
  \draw (#1*2-1,#2*2) node[left]{$#3$};
  \draw [->] (#1*2-0.35,#2*2-0.2) arc (-30:-320:0.32cm);
}
\newcommand{\rloopr}[3]
{
  \draw (#1*2+1,#2*2) node[right]{$#3$};
  \draw [->] (#1*2+0.35,#2*2+0.2) arc (150:-140:0.32cm);
}
\newcommand{\rloopt}[3]
{
  \draw (#1*2,#2*2+1) node[above]{$#3$};
  \draw [->] (#1*2-0.2,#2*2+0.35) arc (240:-50:0.32cm);
}
\newcommand{\rloopb}[3]
{
  \draw (#1*2,#2*2-1) node[below]{$#3$};
  \draw [->] (#1*2+0.2,#2*2-0.35) arc (60:-230:0.32cm);
}
\newcommand{\lloopl}[3]
{
  \draw (#1*2-1,#2*2) node[left]{$#3$};
  \draw [->] (#1*2-0.35,#2*2+0.2) arc (30:320:0.32cm);
}
\newcommand{\lloopr}[3]
{
  \draw (#1*2+1,#2*2) node[right]{$#3$};
  \draw [->] (#1*2+0.35,#2*2-0.2) arc (-150:140:0.32cm);
}
\newcommand{\lloopt}[3]
{
  \draw (#1*2,#2*2+1) node[above]{$#3$};
  \draw [->] (#1*2+0.2,#2*2+0.35) arc (-60:230:0.32cm);
}
\newcommand{\lloopb}[3]
{
  \draw (#1*2,#2*2-1) node[below]{$#3$};
  \draw [->] (#1*2-0.2,#2*2-0.35) arc (-240:50:0.32cm);
}
\tutnr{2}

\section{Altlasten}
\subsection{Altlasten}

\begin{frame}
\frametitle{Automaten und ihre Fehlerzustände}
\begin{itemize}
	\item Ein vollständiger DEA benötigt einen expliziten Fehlerzustand.
	\item Ein DEA mit implizitem Fehlerzustand (alle nicht eingezeichneten Überführungen führen in den Fehlerzustand)
	\item Ein NEA wird i.d.R. nicht vollständig dargestellt ($\Rightarrow$ also kann Fehlerzustand auch implizit sein)
\end{itemize}
\end{frame}

\begin{frame}
\frametitle{Wdh. Übungsblatt}
\begin{itemize}
	\item Jeder muss eine eigene Abgabe anfertigen
	\item Lerngruppen bis 5 Personen
	\item Letzte Aufgabe (mit Stern) wird nicht gewertet
\end{itemize}
\end{frame}

\section{Semi-Thue-Systeme}
\subsection{Semi-Thue-Systeme}

\begin{frame}
\frametitle{Semi-Thue-Systeme}
Ein Semi-Thue-System besteht aus
\begin{itemize}
	\item einem nichtleeren Alphabet $A$
\end{itemize}
und
\begin{itemize}
	\item einer Produktionsmenge $P \subset \{A^* \rightarrow A^*\}$
\end{itemize}~\\
Beispiel:~\\
$A = \{a, b, c\}$~\\
$P = \{ab \rightarrow c, bc \rightarrow a, aa \rightarrow \lambda, cc \rightarrow \lambda\}$~\\~\\
Beispieleingaben:
\begin{description}
	\item[abc] $\Rightarrow$ cc $\Rightarrow$ $\lambda$~\\
	$\Rightarrow$ aa $\Rightarrow$ $\lambda$~\\
	\item[aab] $\Rightarrow$ b~\\
	$\Rightarrow$ ac
\end{description}
Produktionen sind nicht immer eindeutig.
\end{frame}

\section{Äquivalenzklassen}
\subsection{Äquivalenzklassen}
\begin{frame}
	\frametitle{Äquivalenzklassen}
	\begin{block} {Äquivalenz}
		\begin{itemize}
			\item Zwei Zustände sind äquivalent, wenn es für das Erreichen eines Endzustandes durch Abarbeiten eines Wortes $w$ unerheblich ist, aus welchem der beiden Zustände wir starten.
		\end{itemize}
	\end{block}
	\begin{block} {Definition}
  Zwei Zustände $p$ und $q$ eines deterministischen endlichen Automaten heißen \emph{äquivalent} ($p \equiv q$),
  wenn für alle Wörter $w\in\Sigma^*$ gilt:
  \[
   \delta(p, w)\in F \Leftrightarrow \delta(q, w)\in F
  \]
  Offensichtlich ist $\equiv$ eine Äquivalenzrelation. Mit $[p]$ bezeichnen wir die \emph{Äquivalenzklasse} der zu $p$ äquivalenten Zustände.
  \end{block}
\end{frame}

\subsection{Aufgabe 1}
\begin{frame}
	\frametitle{Aufgabe 1}
		Gegeben sei der folgende endliche Automat:\\
		$\mathcal{M} = (\mathcal{Q},\Sigma,\delta,s_0,\mathcal{F})$ mit
		$\mathcal{Q} = \{s_0,s_1,s_2,s_3,s_4\}, \Sigma = \{0,1\}, \mathcal{F} = \{s_4\}$
		und $\delta$ gegeben durch:
	\begin{center}
		\resizebox{4.5cm}{!} {%
		\begin{tikzpicture}[line width=1pt]
			\start{0}{0}{s_0}
			\totr{0}{0}{1}{1}{0}
			\tor{0}{0}{1}{1}
			\state{1}{1}{s_1}
			\tor{1}{1}{2}{0}
			\tob{1}{1}{0}{1}
			\state{1}{0}{s_2}
			\tor{1}{0}{2}{0}
			\rloopb{1}{0}{1}
			\state{2}{0}{s_3}
			\tot{2}{0}{1}{0}
			\draw [->] (4,-0.4) arc (0:-180:2cm);
			\draw (2,-2.4) node[below]{$1$};
			\final{2}{1}{s_4}
			\rloopr{2}{1}{0,1}
		\end{tikzpicture}%
		}
	\end{center}
	\begin{enumerate}
		\item Ist der gegebene endliche Automat deterministisch?
		\item Zeichnen Sie den Äquivalenzklassenautomaten!
		\item Geben Sie die Äquivalenzklassen der Zustände vom entstandenen Automaten an!
	\end{enumerate}
\end{frame}

\section{Minimierung}
\subsection{Minimierung}

\subsection{Aufgabe 2}
\begin{frame}
	\frametitle{Aufgabe 2}
	Gegeben sei ein nichtdeterministischer endlicher Automat (NEA):\\
	$\mathcal{M} = (\mathcal{Q},\Sigma,\delta,q_0,\mathcal{F})$ mit
	$\mathcal{Q} = \{q_0,q_1,q_2\}, \Sigma = \{a,b\}, \mathcal{F} = \{q_2\}$
	und $\delta$ gegeben durch:
	\begin{center}
		\resizebox{4.5cm}{!} {%
		\begin{tikzpicture}[line width=1pt]
			\start{0}{0}{q_0}
			\tor{0}{0}{1}{a}
			\state{1}{0}{q_1}
			\rloopt{1}{0}{a,b}
			\tor{1}{0}{2}{b}
			\final{2}{0}{q_2}
		\end{tikzpicture}%
		}
	\end{center}
	\begin{enumerate}
		\item Geben Sie einen entsprechenden deterministischen endlichen Automaten
		(DEA) an, der die gleiche\\
		Sprache akzeptiert! Benutzen Sie hierbei das Potenzmengenkonstruktionsverfahren!
		\item Ist der entstandene Automat vollständig? Wenn nicht, wie kann man den
		Automaten vervollständigen?\\
		Welche Mengen stellen bei dem Potenzmengenkonstruktionsverfahren einen
		Fehlerzustand dar?
	\end{enumerate}
\end{frame}

\section{Potenzmengenkonstruktion}
\subsection{Potenzmengenkonstruktion}
	
\subsection{Aufgabe 3}
\begin{frame}
	\frametitle{Aufgabe 3}
	Gegeben sei der folgende deterministische endliche Automat:\\
	$\mathcal{M} = (\mathcal{Q},\Sigma,\delta,q_0,\mathcal{F})$ mit
	$\mathcal{Q} = \{q_0,q_1,q_2,q_3,q_4,q_5\}, \Sigma = \{0,1\}, \mathcal{F} =
	\{q_3,q_5\}$ und $\delta$ gegeben durch:
	\begin{center}
		\resizebox{5.5cm}{!} {%
		\begin{tikzpicture}[line width=1pt]
			\start{0}{0}{q_0}
			\tobr{0}{0}{1}{-1}{0}
			\tor{0}{0}{1}{1}
			\state{1}{0}{q_1}
			\tor{1}{0}{2}{0}
			\rloopt{1}{0}{1}
			\state{2}{0}{q_2}
			\tor{2}{0}{3}{0}
			\final{3}{0}{q_3}
			\rloopt{3}{0}{1}
			\state{1}{-1}{q_4}
			\tor{1}{-1}{2}{0}
			\final{2}{-1}{q_5}
			\draw (5,-1) node[below right]{$1$};
			\draw [->] (4.283,-1.717) -- (5.717,-0.283);
		\end{tikzpicture}%
		}
	\end{center}
	\begin{enumerate}
		\item Vervollständigen Sie den Automaten, d.h. führen Sie einen Fehlerzustand ein!
		\item Minimieren Sie den vervollständigten Automaten!
	\end{enumerate}
\end{frame}

\subsection{Aufgabe 4}
\begin{frame}
	\frametitle{Aufgabe 4}
	Gegeben seien die folgenden beiden nichtdeterministischen endlichen Automaten:
	\begin{center}
		\resizebox{6.5cm}{!} {%
		\begin{tikzpicture}[line width=1pt]
			\startfinal{0}{0}{q_1}
			\rloopt{0}{0}{a}
			\draw [->] (0.4,0) arc (90:-90:1cm);
			\draw (1.4,-1) node[right]{$a,b$};
			\state{0}{-1}{q_2}
			\tot{0}{-1}{0}{b}
			\start{2}{0}{s_1}
			\tor{2}{0}{3}{\varepsilon}
			\tob{2}{0}{-1}{a}
			\final{3}{0}{s_2}
			\draw [->] (6,0.4) arc (0:180:1cm);
			\draw (5,1.4) node[above]{$a$};
			\state{2}{-1}{s_3}
			\draw (5,-1) node[below right]{$a,b$};
			\draw [->] (4.283,-1.717) -- (5.717,-0.283);
			\rloopb{2}{-1}{b}
		\end{tikzpicture}%
		}
	\end{center}
	Wandeln Sie diese mittels des Potenzmengenkonstruktionsverfahrens in
	deterministische endliche Automaten um!
\end{frame}

\section{Schluss}
\subsection{Schluss}

\begin{frame}
\frametitle{Bis zum nächsten Mal!}
\vspace{0.3cm}
\begin{center}
	\resizebox{11.85cm}{!} {%
	\includegraphics[height=0.8\textheight]{images/xkcd_804.png}%
	}
\end{center}
\end{frame}

\input{includes/disclaimer}
\end{document}
