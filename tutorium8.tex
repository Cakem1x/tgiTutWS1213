\documentclass[t]{beamer}
\usetheme[deutsch]{KIT}
\setbeamercovered{transparent}
\setbeamertemplate{navigation symbols}{}

\KITfoot{Tutoriumsmaterial von Joachim Priesner, Sebastian Ullrich und Max Wagner \hspace{2.5cm} Basierend auf den Folien von Simon Stroh und Moritz v. Looz}
\usepackage[utf8]{inputenc}
\usepackage{amsmath}
\usepackage{ifthen}
\usepackage{amssymb}
\usepackage{tikz}
\usepackage{ngerman}
\usepackage[normalem]{ulem}
\usetikzlibrary{automata}
\usenavigationsymbols


\title{Theoretische Grundlagen der Informatik}
\subtitle{Tutorium}
\author{Moritz von Looz, Simon Stroh}

\institute[ITI]{Institut für Theoretische Informatik}

\TitleImage[height=\titleimageht]{images/tmaschine.png}

\newcommand{\N}{\ensuremath{\mathbb{N}}}
\newcommand{\M}{\ensuremath{\mathcal{M}}}
\newcommand{\classP}{\ensuremath{\mathcal{P}}}
\newcommand{\classNP}{\ensuremath{\mathcal{NP}}}
\newcommand{\co}{\ensuremath{\mathsf{co\text{-}}}}
\newcommand{\pot}{\ensuremath{\mathcal{P}}}
\newcommand{\abs}[1]{\ensuremath{\left\vert #1 \right\vert}}
\newcommand{\menge}[2]{\ensuremath{\left\lbrace #1 \,\middle\vert\, #2 \right\rbrace}}
\newcommand{\ducttape}[1]{\vspace{#1}}
\newcommand{\neglit}[1]{\overline{#1\vphantom{x^a}}}
\newcommand{\recipe}{\raisebox{-.3cm}{\includegraphics[scale=.15]{images/chefs-cap.png}}\hspace{0.2cm}}
\newcommand{\opt}[1]{\ensuremath{\text{OPT}(#1)}}
\newcommand{\A}[1]{\ensuremath{\mathcal{A}(#1)}}
\renewcommand{\O}[1]{\ensuremath{\mathcal{O}(#1)}}
\newcommand{\msout}[1]{\text{\sout{\ensuremath{#1}}}}

\newcommand{\invincible}{\setbeamercovered{invisible}} %  "Yesss! I am invincible!!" (Boris Grishenko)
\newcommand{\vincible}{\setbeamercovered{transparent}}
\renewcommand{\solution}[1]{\invincible \pause #1 \vincible}
\newcommand{\micropause}{\\[8pt]}

% \@ifundefined{tikzset}{}{\tikzset{initial text=}} % Text "start" bei Startknoten unterdrücken
\tikzstyle{every node}=[thick]
\tikzstyle{every line}=[thick]

\newcommand{\tutnr}[1]{
  \subtitle{Tutorium #1}
	\begin{frame}
		\maketitle
	\end{frame}
}

\newcommand{\uebnr}[1]{
  \subtitle{Anmerkungen zum #1. Übungsblatt}
	\begin{frame}
		\maketitle
	\end{frame}
}

\begin{document}

\newcommand{\start}[3]
{
  \draw (#1*2,#2*2) node{$#3$};
  \draw (#1*2,#2*2) circle(0.4cm);
  \draw [->] (#1*2-0.9,#2) -- (#1*2-0.4,#2);
}
\newcommand{\final}[3]
{
  \draw (#1*2,#2*2) node{$#3$};
  \draw (#1*2,#2*2) circle(0.4cm);
  \draw (#1*2,#2*2) circle(0.32cm);
}
\newcommand{\startfinal}[3]
{
  \draw (#1*2,#2*2) node{$#3$};
  \draw (#1*2,#2*2) circle(0.4cm);
  \draw (#1*2,#2*2) circle(0.32cm);
  \draw [->] (#1*2-0.9,#2) -- (#1*2-0.4,#2);
}
\newcommand{\state}[3]
{
  \draw (#1*2,#2*2) node{$#3$};
  \draw (#1*2,#2*2) circle(0.4cm);
}
\newcommand{\tol}[4]
{
  \draw (#1+#3,#2*2) node[above]{$#4$};
  \draw [->] (#1*2-0.4,#2*2) -- (#3*2+0.4,#2*2);
}
\newcommand{\tor}[4]
{
  \draw (#1+#3,#2*2) node[above]{$#4$};
  \draw [->] (#1*2+0.4,#2*2) -- (#3*2-0.4,#2*2);
}
\newcommand{\tot}[4]
{
  \draw (#1*2,#2+#3) node[right]{$#4$};
  \draw [->] (#1*2,#2*2+0.4) -- (#1*2,#3*2-0.4);
}
\newcommand{\tob}[4]
{
  \draw (#1*2,#2+#3) node[right]{$#4$};
  \draw [->] (#1*2,#2*2-0.4) -- (#1*2,#3*2+0.4);
}
\newcommand{\totl}[5]
{
  \draw (#1+#3,#2+#4) node[above right]{$#5$};
  \draw [->] (#1*2-0.283,#2*2+0.283) -- (#3*2+0.283,#4*2-0.283);
}
\newcommand{\totr}[5]
{
  \draw (#1+#3,#2+#4) node[above left]{$#5$};
  \draw [->] (#1*2+0.283,#2*2+0.283) -- (#3*2-0.283,#4*2-0.283);
}
\newcommand{\tobl}[5]
{
  \draw (#1+#3,#2+#4) node[below right]{$#5$};
  \draw [->] (#1*2-0.283,#2*2-0.283) -- (#3*2+0.283,#4*2+0.283);
}
\newcommand{\tobr}[5]
{
  \draw (#1+#3,#2+#4) node[below left]{$#5$};
  \draw [->] (#1*2+0.283,#2*2-0.283) -- (#3*2-0.283,#4*2+0.283);
}
\newcommand{\rloopl}[3]
{
  \draw (#1*2-1,#2*2) node[left]{$#3$};
  \draw [->] (#1*2-0.35,#2*2-0.2) arc (-30:-320:0.32cm);
}
\newcommand{\rloopr}[3]
{
  \draw (#1*2+1,#2*2) node[right]{$#3$};
  \draw [->] (#1*2+0.35,#2*2+0.2) arc (150:-140:0.32cm);
}
\newcommand{\rloopt}[3]
{
  \draw (#1*2,#2*2+1) node[above]{$#3$};
  \draw [->] (#1*2-0.2,#2*2+0.35) arc (240:-50:0.32cm);
}
\newcommand{\rloopb}[3]
{
  \draw (#1*2,#2*2-1) node[below]{$#3$};
  \draw [->] (#1*2+0.2,#2*2-0.35) arc (60:-230:0.32cm);
}
\newcommand{\lloopl}[3]
{
  \draw (#1*2-1,#2*2) node[left]{$#3$};
  \draw [->] (#1*2-0.35,#2*2+0.2) arc (30:320:0.32cm);
}
\newcommand{\lloopr}[3]
{
  \draw (#1*2+1,#2*2) node[right]{$#3$};
  \draw [->] (#1*2+0.35,#2*2-0.2) arc (-150:140:0.32cm);
}
\newcommand{\lloopt}[3]
{
  \draw (#1*2,#2*2+1) node[above]{$#3$};
  \draw [->] (#1*2+0.2,#2*2+0.35) arc (-60:230:0.32cm);
}
\newcommand{\lloopb}[3]
{
  \draw (#1*2,#2*2-1) node[below]{$#3$};
  \draw [->] (#1*2-0.2,#2*2-0.35) arc (-240:50:0.32cm);
}
\include{amsmath}
\tutnr{8}

\section{Altlasten}
\subsection{Reduktion}

\section{SAT}
\subsection{SAT}
\begin{frame}
\frametitle{SAT (SATisfiability = Erfüllbarkeitsproblem)}
\begin{block}{Problem}
\textbf{Gegeben:} Formel in konjunktiver Normalform
\begin{itemize}
 \item Menge $U$ von Variablen
 \item Menge $C$ von Klauseln (Disjunktionen) über $U$
\end{itemize} $$ \underbrace{(\stackrel{\text{Literale}}{\stackrel{\downarrow}{\vphantom{\neglit{a}}a} \vee \stackrel{\downarrow}{\vphantom{\neglit{b}}b} \vee \stackrel{\downarrow}{\neglit{c}}})}_\text{Klausel} \wedge \underbrace{(b \vee c)}_\text{Klausel} \wedge \underbrace{(\neglit{a} \vee \neglit{b} \vee \neglit{c})}_\text{Klausel}$$

\textbf{Frage:} Existiert eine (alle Klauseln) erfüllende Variablenbelegung?
\end{block}
\end{frame}

\section{$\mathcal{P}, \mathcal{NP}$}
\subsection{Erklärung}
\begin{frame}
	\frametitle{$\mathcal{P}, \mathcal{NP}$}
	\ducttape{1cm}
	$\mathcal{P}$ ist die Klasse aller Sprachen, die von einer deterministischen Turingmaschine in Polynomialzeit erkannt werden.\\
	\ducttape{1cm}
	$\mathcal{NP}$ ist die Klasse aller Sprachen, die von einer \textbf{nicht}deterministischen Turingmaschine in Polynomialzeit erkannt werden.\\
	\ducttape{1cm}
	\textbf{Anmerkungen: }
	\begin{itemize}
		\item $\mathcal{P} \subseteq \mathcal{NP}$.
		\item Die Frage ob $\mathcal{P} = \mathcal{NP}$ gilt ist ein großes, ungeklärtes Problem.
	\end{itemize}
	\end{frame}

\begin{frame}
	\frametitle{Gedanklicher Trick für $\mathcal{NP}$}
	Ein Problem liegt in $\mathcal{NP}$ falls man eine mögliche Lösung in Polynomialzeit von einer \textbf{deterministischen} Turingmaschine verifizieren lassen kann.
	\begin{block}{Beispiel}
		Zeige: \emph{SAT} $\in \mathcal{NP}$
		\begin{enumerate}
			\item Es werden nichtdeterministisch alle möglichen Variablenbelegungen aufs Band geschrieben.
			\item Es gibt nun eine deterministische Turingmaschine welche die Variablenbelegung in Polynomialzeit überprüft.
		\end{enumerate}
	\end{block}
\end{frame}
\subsection{Aufgabe B8 A2}

\section{Graph Färbbarkeit}
\subsection{Erklärung}
\subsection{Beispiel}
\subsection{Aufgabe B8 A3}

\section{2-SAT}
\subsection{Erkläung}
\subsection{Beispiel}

\section{Weitere Aufgaben}
\subsection{2-COLOR auf 2-SAT Reduktion (B8 A4)}
\subsection{B8 A1}

\section{Schluss}
\subsection{Schluss}
\begin{frame}
\frametitle{Bis zum nächsten Mal!}
\begin{center}
	%TODO: Aktuellen Comic einfügen
	\includegraphics[width=1 \textheight]{images/xkcd_981.png}
\end{center}
\end{frame}

\input{includes/disclaimer}
\end{document}