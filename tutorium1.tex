\documentclass[t]{beamer}
\usetheme[deutsch]{KIT}
\setbeamercovered{transparent}
\setbeamertemplate{navigation symbols}{}

\KITfoot{Tutoriumsmaterial von Joachim Priesner, Sebastian Ullrich und Max Wagner \hspace{2.5cm} Basierend auf den Folien von Simon Stroh und Moritz v. Looz}
\usepackage[utf8]{inputenc}
\usepackage{amsmath}
\usepackage{ifthen}
\usepackage{amssymb}
\usepackage{tikz}
\usepackage{ngerman}
\usepackage[normalem]{ulem}
\usetikzlibrary{automata}
\usenavigationsymbols


\title{Theoretische Grundlagen der Informatik}
\subtitle{Tutorium}
\author{Moritz von Looz, Simon Stroh}

\institute[ITI]{Institut für Theoretische Informatik}

\TitleImage[height=\titleimageht]{images/tmaschine.png}

\newcommand{\N}{\ensuremath{\mathbb{N}}}
\newcommand{\M}{\ensuremath{\mathcal{M}}}
\newcommand{\classP}{\ensuremath{\mathcal{P}}}
\newcommand{\classNP}{\ensuremath{\mathcal{NP}}}
\newcommand{\co}{\ensuremath{\mathsf{co\text{-}}}}
\newcommand{\pot}{\ensuremath{\mathcal{P}}}
\newcommand{\abs}[1]{\ensuremath{\left\vert #1 \right\vert}}
\newcommand{\menge}[2]{\ensuremath{\left\lbrace #1 \,\middle\vert\, #2 \right\rbrace}}
\newcommand{\ducttape}[1]{\vspace{#1}}
\newcommand{\neglit}[1]{\overline{#1\vphantom{x^a}}}
\newcommand{\recipe}{\raisebox{-.3cm}{\includegraphics[scale=.15]{images/chefs-cap.png}}\hspace{0.2cm}}
\newcommand{\opt}[1]{\ensuremath{\text{OPT}(#1)}}
\newcommand{\A}[1]{\ensuremath{\mathcal{A}(#1)}}
\renewcommand{\O}[1]{\ensuremath{\mathcal{O}(#1)}}
\newcommand{\msout}[1]{\text{\sout{\ensuremath{#1}}}}

\newcommand{\invincible}{\setbeamercovered{invisible}} %  "Yesss! I am invincible!!" (Boris Grishenko)
\newcommand{\vincible}{\setbeamercovered{transparent}}
\renewcommand{\solution}[1]{\invincible \pause #1 \vincible}
\newcommand{\micropause}{\\[8pt]}

% \@ifundefined{tikzset}{}{\tikzset{initial text=}} % Text "start" bei Startknoten unterdrücken
\tikzstyle{every node}=[thick]
\tikzstyle{every line}=[thick]

\newcommand{\tutnr}[1]{
  \subtitle{Tutorium #1}
	\begin{frame}
		\maketitle
	\end{frame}
}

\newcommand{\uebnr}[1]{
  \subtitle{Anmerkungen zum #1. Übungsblatt}
	\begin{frame}
		\maketitle
	\end{frame}
}

\begin{document}

\tutnr{1}
\newcommand{\start}[3]
{
  \draw (#1*2,#2*2) node{$#3$};
  \draw (#1*2,#2*2) circle(0.4cm);
  \draw [->] (#1*2-0.9,#2) -- (#1*2-0.4,#2);
}
\newcommand{\state}[3]
{
  \draw (#1*2,#2*2) node{$#3$};
  \draw (#1*2,#2*2) circle(0.4cm);
}
\newcommand{\final}[3]
{
  \draw (#1*2,#2*2) node{$#3$};
  \draw (#1*2,#2*2) circle(0.4cm);
  \draw (#1*2,#2*2) circle(0.32cm);
}
\newcommand{\tol}[4]
{
  \draw (#1+#3,#2*2) node[above]{$#4$};
  \draw [->] (#1*2-0.4,#2*2) -- (#3*2+0.4,#2*2);
}
\newcommand{\tor}[4]
{
  \draw (#1+#3,#2*2) node[above]{$#4$};
  \draw [->] (#1*2+0.4,#2*2) -- (#3*2-0.4,#2*2);
}
\newcommand{\tot}[4]
{
  \draw (#1*2,#2+#3) node[right]{$#4$};
  \draw [->] (#1*2,#2*2+0.4) -- (#1*2,#3*2-0.4);
}
\newcommand{\tob}[4]
{
  \draw (#1*2,#2+#3) node[right]{$#4$};
  \draw [->] (#1*2,#2*2-0.4) -- (#1*2,#3*2+0.4);
}
\newcommand{\totl}[5]
{
  \draw (#1+#3,#2+#4) node[above right]{$#5$};
  \draw [->] (#1*2-0.283,#2*2+0.283) -- (#3*2+0.283,#4*2-0.283);
}
\newcommand{\totr}[5]
{
  \draw (#1+#3,#2+#4) node[above left]{$#5$};
  \draw [->] (#1*2+0.283,#2*2+0.283) -- (#3*2-0.283,#4*2-0.283);
}
\newcommand{\tobl}[5]
{
  \draw (#1+#3,#2+#4) node[below right]{$#5$};
  \draw [->] (#1*2-0.283,#2*2-0.283) -- (#3*2+0.283,#4*2+0.283);
}
\newcommand{\tobr}[5]
{
  \draw (#1+#3,#2+#4) node[below left]{$#5$};
  \draw [->] (#1*2+0.283,#2*2-0.283) -- (#3*2-0.283,#4*2+0.283);
}
\newcommand{\rloopl}[3]
{
  \draw (#1*2-1,#2*2) node[left]{$#3$};
  \draw [->] (#1*2-0.35,#2*2-0.2) arc (-30:-320:0.32cm);
}
\newcommand{\rloopr}[3]
{
  \draw (#1*2+1,#2*2) node[right]{$#3$};
  \draw [->] (#1*2+0.35,#2*2+0.2) arc (150:-140:0.32cm);
}
\newcommand{\rloopt}[3]
{
  \draw (#1*2,#2*2+1) node[above]{$#3$};
  \draw [->] (#1*2-0.2,#2*2+0.35) arc (240:-50:0.32cm);
}
\newcommand{\rloopb}[3]
{
  \draw (#1*2,#2*2-1) node[below]{$#3$};
  \draw [->] (#1*2+0.2,#2*2-0.35) arc (60:-230:0.32cm);
}
\newcommand{\lloopl}[3]
{
  \draw (#1*2-1,#2*2) node[left]{$#3$};
  \draw [->] (#1*2-0.35,#2*2+0.2) arc (30:320:0.32cm);
}
\newcommand{\lloopr}[3]
{
  \draw (#1*2+1,#2*2) node[right]{$#3$};
  \draw [->] (#1*2+0.35,#2*2-0.2) arc (-150:140:0.32cm);
}
\newcommand{\lloopt}[3]
{
  \draw (#1*2,#2*2+1) node[above]{$#3$};
  \draw [->] (#1*2+0.2,#2*2+0.35) arc (-60:230:0.32cm);
}
\newcommand{\lloopb}[3]
{
  \draw (#1*2,#2*2-1) node[below]{$#3$};
  \draw [->] (#1*2-0.2,#2*2-0.35) arc (-240:50:0.32cm);
}
\section{Organisatorisches}

\subsection{Organisatorisches}

\begin{frame}
	\frametitle{\texttt{whois tutor}}
	
	\begin{itemize}
		\item \textbf{Alexander Kwiatkowski} \\ alexander.kwiatkowski@gmx.net \\ Dienstag 17:30, SR -120
		\item \textbf{Michael Vollmer} \\ Michael@trollbu.de \\ Dienstag 17:30, SR -119
		\item \textbf{Matthias Holoch} \\ Matthias.Holoch@student.kit.edu \\ Donnerstag 8:00, SR -120
	\end{itemize}
\end{frame}

\begin{frame}
	\frametitle{Organisatorisches -- Zum Übungsbetrieb}
	\begin{itemize}
		\item \textbf{Abgabe:} \emph{Handschriftlich} in Gruppen
		\begin{itemize}
			\item Bis zu 5 Personen pro Gruppe
			\item Erste Abgabe legt die Gruppe fest
			\item Jede Person muss ein eigenes Blatt abgeben (mit Gruppenname falls vorhanden)
		\end{itemize}
		\item \textbf{Schein:} 
		\begin{itemize}
			\item Klausurbonus (1 Notenschritt)
			\item Bei min. 6 (von 7) Blättern 50\% Punkte
		\end{itemize}
		\item korrigierte Übungsblätter gibt es im Tutorium
		\begin{itemize}
			\item Bei Nichtabholung: Büro 274 Montags 14:00-15:00
		\end{itemize}
	\end{itemize}
\end{frame}
\begin{frame}
	\frametitle{Organisatorisches -- Zum Tutorium}
	\begin{itemize}
	\item Tutoriumsfolien
		\begin{itemize}
			\item \texttt{http://tinyurl.com/tgi1213}
		\end{itemize}
		\item E-Mail-Liste geht rum
		\item Stoff soll wiederholt werden
		\item Dabei Fokus auf Übungsbetrieb
		\item Fragen/Vorschläge/Anmerkungen willkommen!
	\end{itemize}
\end{frame}

\subsection*{Aufgabe 1}
\begin{frame}
	\frametitle{Aufgabe 1}
	\only<1>{
	Gegeben sei der folgende endliche Automat:\\
	$\mathcal{M} = (\mathcal{Q},\Sigma,\delta,S,\mathcal{F})$ mit
	$\Sigma = \{a,b\}$, $\mathcal{Q} = \{S,B,C,D\}$, $\mathcal{F} = \{B,C\}$
	und $\delta$ gegeben durch:
	}
	\only<2>{\vspace{-1.7 cm}}
	\begin{center}
		\resizebox{4cm}{!} {%
		\begin{tikzpicture}[line width=1pt]
			\start{0}{0}{S}
			\totr{0}{0}{1}{1}{a}
			\tobr{0}{0}{1}{-1}{b}
			\final{1}{1}{B}
			\rloopt{1}{1}{a}
			\draw (3,1) node[above right]{$b$};
			\draw [->] (2.283,1.717) -- (3.717,0.283);
			\final{1}{-1}{C}
			\draw (3,-1) node[below right]{$a$};
			\draw [->] (2.283,-1.717) -- (3.717,-0.283);
			\rloopb{1}{-1}{b}
			\state{2}{0}{D}
			\rloopr{2}{0}{a,b}
		\end{tikzpicture}%
		}
	\end{center}
	\only<2>{
	\begin{enumerate}
		\item Geben Sie die von diesem Automaten akzeptierte Sprache in einem regulären
		Ausdruck an!
		\item Um was für einen Automaten handelt es sich?
		\item Konstruieren Sie einen äquivalenten endlichen Automaten, der nur einen
		einzigen Endzustand besitzt!
		\item Geben Sie eine linkslineare Grammatik für die Sprache dieses Automaten an,
		die keine überflüssigen Nichtterminale und Regeln enthält!
	\end{enumerate}
	}
\end{frame}

\section{Chomsky-Hierarchie}
\subsection{Chomsky-Hierarchie}
\begin{frame}
	\frametitle{Chomsky-Hierarchie}
	\begin{itemize}
		\item \textbf{Chomsky Typ 3}
		\begin{itemize}
			\item Reguläre Sprachen (z.B. rechtslineare Sprachen)
			\item Reguläre Ausdrücke
			\item Endliche Automaten
		\end{itemize}
		\item \textbf{Chomsky Typ 2}
		\begin{itemize}
			\item $Ch3 \subseteq Ch2$
			\item Kontextfreie Sprachen
			\item Nichtdeterministische Kellerautomaten
			\item Programmiersprachen sind in der Regel Ch2
		\end{itemize}
		\item \textbf{Chomsky Typ 1}
		\begin{itemize}
			\item $Ch2 \subseteq Ch1$
			\item Kontextsensitiven Sprachen
			\item Nichtdeterministische, linear platzbeschränkte Turingmaschine
		\end{itemize}
		\item \textbf{Chomsky Typ 0}
		\begin{itemize}
			\item $Ch1 \subseteq Ch0$
			\item Semi-entscheidbare Sprachen (durch Turingmaschine)
		\end{itemize}
	\end{itemize}
\end{frame}
\subsection{Aufgabe 2}
\begin{frame}
	\frametitle{Aufgabe 2}
	\begin{enumerate}
		\item Formulieren Sie einen regulären Ausdruck über dem Alphabet $\Sigma =
		\{0,1\}$, der jedes beliebige Wort erfasst,\\
		wobei die vorletzte Ziffer $0$ sein soll!
		\item Geben Sie für diese Sprache den möglichst größten Chomsky-Typ und eine
		zugehörige Grammatik an!
		\item Geben Sie einen dazugehörigen Automaten an, der diese Sprache akzeptiert!
	\end{enumerate}
\end{frame}

\subsection{Aufgabe 3}
\begin{frame}
	\frametitle{Aufgabe 3}
	Gegeben sei der folgende endliche Akzeptor $\mathcal{M}$ mit dem Eingabealphabet
	$\Sigma = \{a,b,c,d\}$:
	\begin{center}
		\begin{tikzpicture}[line width=1pt]
			\start{0}{0}{q_0}
			\tor{0}{0}{1}{a,b}
			\state{1}{0}{q_1}
			\rloopt{1}{0}{a,b}
			\totr{1}{0}{3}{1}{c}
			\state{3}{1}{q_2}
			\rloopt{3}{1}{c}
			\draw (8,1) node[above right]{$d$};
			\draw [->] (6.283,1.717) -- (9.717,0.283);
			\final{5}{0}{q_3}
			\tol{5}{0}{1}{d}
		\end{tikzpicture}
	\end{center}
	\begin{enumerate}
		\item Welche Sprache $\mathcal{L}(\mathcal{M})$ wird von dem Akzeptor $\mathcal{M}$
		akzeptiert?
		\item Konstruieren Sie aus $\mathcal{M}$ eine rechtslineare Grammatik, die
		$\mathcal{L}(\mathcal{M})$ erzeugt!
	\end{enumerate}
\end{frame}

\subsection{Aufgabe 4}
\begin{frame}
	\frametitle{Aufgabe 4}
	Die Sprache $\mathcal{L}$ sei durch den regulären Ausdruck $(aa^*b^*)^*cc^*$
	definiert.
	\begin{enumerate}
		\item Geben Sie eine rechtslineare Grammatik $\mathcal{G}$ an, die $\mathcal{L}$
		erzeugt!
		\item Konstruieren Sie aus $\mathcal{G}$ einen endlichen Akzeptor, der $\mathcal{L}$
	akzeptiert!
	\end{enumerate}
\end{frame}

\section{Schluss}
\subsection{Schluss}

\begin{frame}
\frametitle{Bis zum nächsten Mal!}
\vspace{-0.5cm}
\begin{center}\includegraphics[height=0.8\textheight]{images/regular_expressions.png}\end{center}
\end{frame}

\input{includes/disclaimer}
\end{document}
