\documentclass[t]{beamer}
\usetheme[deutsch]{KIT}
\setbeamercovered{transparent}
\setbeamertemplate{navigation symbols}{}

\KITfoot{Tutoriumsmaterial von Joachim Priesner, Sebastian Ullrich und Max Wagner \hspace{2.5cm} Basierend auf den Folien von Simon Stroh und Moritz v. Looz}
\usepackage[utf8]{inputenc}
\usepackage{amsmath}
\usepackage{ifthen}
\usepackage{amssymb}
\usepackage{tikz}
\usepackage{ngerman}
\usepackage[normalem]{ulem}
\usetikzlibrary{automata}
\usenavigationsymbols


\title{Theoretische Grundlagen der Informatik}
\subtitle{Tutorium}
\author{Moritz von Looz, Simon Stroh}

\institute[ITI]{Institut für Theoretische Informatik}

\TitleImage[height=\titleimageht]{images/tmaschine.png}

\newcommand{\N}{\ensuremath{\mathbb{N}}}
\newcommand{\M}{\ensuremath{\mathcal{M}}}
\newcommand{\classP}{\ensuremath{\mathcal{P}}}
\newcommand{\classNP}{\ensuremath{\mathcal{NP}}}
\newcommand{\co}{\ensuremath{\mathsf{co\text{-}}}}
\newcommand{\pot}{\ensuremath{\mathcal{P}}}
\newcommand{\abs}[1]{\ensuremath{\left\vert #1 \right\vert}}
\newcommand{\menge}[2]{\ensuremath{\left\lbrace #1 \,\middle\vert\, #2 \right\rbrace}}
\newcommand{\ducttape}[1]{\vspace{#1}}
\newcommand{\neglit}[1]{\overline{#1\vphantom{x^a}}}
\newcommand{\recipe}{\raisebox{-.3cm}{\includegraphics[scale=.15]{images/chefs-cap.png}}\hspace{0.2cm}}
\newcommand{\opt}[1]{\ensuremath{\text{OPT}(#1)}}
\newcommand{\A}[1]{\ensuremath{\mathcal{A}(#1)}}
\renewcommand{\O}[1]{\ensuremath{\mathcal{O}(#1)}}
\newcommand{\msout}[1]{\text{\sout{\ensuremath{#1}}}}

\newcommand{\invincible}{\setbeamercovered{invisible}} %  "Yesss! I am invincible!!" (Boris Grishenko)
\newcommand{\vincible}{\setbeamercovered{transparent}}
\renewcommand{\solution}[1]{\invincible \pause #1 \vincible}
\newcommand{\micropause}{\\[8pt]}

% \@ifundefined{tikzset}{}{\tikzset{initial text=}} % Text "start" bei Startknoten unterdrücken
\tikzstyle{every node}=[thick]
\tikzstyle{every line}=[thick]

\newcommand{\tutnr}[1]{
  \subtitle{Tutorium #1}
	\begin{frame}
		\maketitle
	\end{frame}
}

\newcommand{\uebnr}[1]{
  \subtitle{Anmerkungen zum #1. Übungsblatt}
	\begin{frame}
		\maketitle
	\end{frame}
}

\begin{document}

\tutnr{1}

\section{Organisatorisches}

\subsection{Organisatorisches}

\begin{frame}
	\frametitle{\texttt{whois tutor}}
	
	\begin{itemize}
		\item \textbf{Alexander Kwiatkowski} \\ alexander.kwiatkowski@gmx.net \\ Dienstag 17:30, SR -120
		\item \textbf{Michael Vollmer} \\ Michael@trollbu.de \\ Dienstag 17:30, SR -119
		\item \textbf{Matthias Holoch} \\ Matthias.Holoch@student.kit.edu \\ Donnerstag 8:00, SR -120
	\end{itemize}
\end{frame}

\begin{frame}
	\frametitle{Organisatorisches -- Zum Übungsbetrieb}
	\begin{itemize}
		\item \textbf{Abgabe:} \emph{Handschriftlich} in Gruppen
		\begin{itemize}
			\item Bis zu 5 Personen pro Gruppe
			\item Erste Abgabe legt die Gruppe fest
			\item Jede Person muss ein eigenes Blatt abgeben (mit Gruppenname falls vorhanden)
		\end{itemize}
		\item \textbf{Schein:} 
		\begin{itemize}
			\item Klausurbonus (1 Notenschritt)
			\item Bei min. 6 (von 7) Blättern 50\% Punkte
		\end{itemize}
		\item korrigierte Übungsblätter gibt es im Tutorium
		\begin{itemize}
			\item Bei Nichtabholung: Büro 274 Montags 14:00-15:00
		\end{itemize}
	\end{itemize}
\end{frame}
\begin{frame}
	\frametitle{Organisatorisches -- Zum Tutorium}
	\begin{itemize}
	\item Tutoriumsfolien
		\begin{itemize}
			\item \texttt{http://tinyurl.com/tgi1213}
		\end{itemize}
		\item E-Mail-Liste geht rum
		\item Stoff soll wiederholt werden
		\item Dabei Fokus auf Übungsbetrieb
		\item Fragen/Vorschläge/Anmerkungen willkommen!
	\end{itemize}
\end{frame}

\section{Schluss}
\subsection{Schluss}

\begin{frame}
\frametitle{Bis zum nächsten Mal!}
\vspace{-0.5cm}
\begin{center}\includegraphics[height=0.8\textheight]{images/regular_expressions.png}\end{center}
\end{frame}

\input{includes/disclaimer}
\end{document}
